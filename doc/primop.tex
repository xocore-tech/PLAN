\documentclass[11pt]{article}

\usepackage{fancyhdr}
\pagestyle{empty}
\fancypagestyle{plain}{\pagestyle{empty}}

\usepackage{listings}
\usepackage[T1]{fontenc}
\usepackage{textcomp}
\usepackage{lmodern} % Latin Modern family (vectorized Computer Modern)
\usepackage{microtype}
\usepackage[paperwidth=210mm,paperheight=2000mm,margin=15mm]{geometry}
\usepackage{parskip}
\usepackage{needspace}
\usepackage[colorlinks=true,linkcolor=blue]{hyperref}

\lstset{
  basicstyle=\ttfamily\small,
  breaklines=true,
  frame=single,
  xleftmargin=1em,   % indent from left
}

\lstdefinelanguage{asm}{
  morecomment=[l]{\#},
}

\lstset{
  language=asm,
  basicstyle=\ttfamily\small,
  commentstyle=\color{black},
  tabsize=8,
  breaklines=false,
  captionpos=b
}

\lstdefinelanguage{Haskell}{
  morekeywords={case,of,if,then,else,data,type,class,where,do,let,in},
  sensitive=true,
  morecomment=[l]--,
  morecomment=[n]{\{-}{-\}},
  morestring=[b]",
}

\lstdefinestyle{haskellstyle}{
  backgroundcolor=\color{backcolour},
  commentstyle=\color{codegreen},
  keywordstyle=\color{blue},
  numberstyle=\tiny\color{codegray},
  stringstyle=\color{codepurple},
  basicstyle=\ttfamily\footnotesize,
  breaklines=true,
  numbers=left,
  numbersep=5pt,
  showstringspaces=false,
  tabsize=2,
}

\lstnewenvironment{asm}[1][] % optional args pass through
  {\lstset{language=asm,#1}}
  {}

\lstnewenvironment{asmdata}[1][] % optional args pass through
  {\lstset{language=asm,#1}}
  {}

\lstnewenvironment{clang}[1][]
                  {\lstset{language=C,#1}}
                  {}


\title{The Design of the PrimOp Calling Convention of PLAN}
\author{Sol}
\date{September 12, 2025}

\begin{document}

\maketitle
\begin{abstract}

This describes the primop calling convention and provides an in-depth
rational for the design.

\end{abstract}

\section{Introduction}

PLAN uses the following calling convention for primitive operations.
This document will discuss why this design was chosen, and what
important properties it has, which other options failed to provide.

\begin{lstlisting}
(<0> (0 i))           -> mk_pin(i)
(<0> (1 n a b))       -> mk_law(n,a,b)
(<0> (2 p l a z m o)) -> plan_case(p,l,a,z,m,o)
\end{lstlisting}

This convention replaces the more obvious convention used in earlier
designs.

\begin{lstlisting}
(<0> i)           -> mk_pin(i)
(<1> n a b)       -> mk_law(n,a,b)
(<2> p l a z m o) -> plan_case(p,l,a,z,m,o)
\end{lstlisting}

One advantage of the new design is that it makes the arity logic
simpler. Before, the arity of pinned nats was 3 for
\textless1\textgreater, 6 for \textless2\textgreater, and 1 for
everything else. With the new logic, the arity is always 1.

But more importantly, this design is forced by two core constraints:

\begin{enumerate}
\def\labelenumi{\arabic{enumi}.}
\item
  The need to have a shared runtime with a common semantics between PLAN
  and XPLAN.
\item
  The need to enable some basic optimizations when compiling laws which
  may not yet have been fully normalized (may contain references to lazy
  data).
\end{enumerate}

\section{XPLAN and PLAN}\label{xplan-and-plan}

A PLAN runtime is actually a runtime for two different languages at the
same time: PLAN and XPLAN. PLAN is a lazy, purely functional, frozen
language with a tiny set of built-in operations. XPLAN is an impure
functional language with a much larger, growing set of built-in
operations.

Extending the obvious calling convention to include XPLAN operations
would require changing the arity:

\begin{lstlisting}
(<3> x)   = inc(x)
(<4> x)   = dec(x)
(<5> x y) = add(x,y)
\end{lstlisting}

This creates fundamental incompatibilities between PLAN and XPLAN.

\begin{itemize}
\item
  A single execution engine cannot be used for both languages since, the
  graph reduction engine and an optimizing law compiler needs to be able
  to determine if an application is saturated or not, and this needs to
  know the arity.
\item
  The universe of possible evaluated values changes. For example, the
  following value can exist in XPLAN, but not in PLAN:
  \texttt{(\textless{}5\textgreater{}\ 1)}. Enforcing that such values
  never leak between contexts is not realistic.
\end{itemize}

Switching to the new convention eliminates all of the essential
differences between PLAN and XPLAN. PLAN is just an execution mode of
XPLAN where unlawful operations crash instead of being executed.

\section{Optimizing Laws}\label{optimizing-laws}

Another option that was considered, which avoids constructing and ADT
for each primop call, was the following:

\begin{lstlisting}
ARITY(<i:@>) = i+1

(<1> 0 i)           = MkPin(i)
(<3> 1 n a b)       = MkLaw(n,a,b)
(<6> 2 p l a z m o) = VCase(p,l,a,z,m,o)
\end{lstlisting}

This avoids the incompatibility between XPLAN and PLAN by using the
pinned nat to indicate the arity, and then having the first argument
identify the operation.

However, this interacts very poorly with support for laziness within
pins and laws.

In order to demonstrate this, let's look at the code for a wrapper
function which calls one of these primops.

\begin{lstlisting}
(Pin x)=(<1> 0 x)

Pin = <{"Pin" 1 [<1>[0] 1]}>
\end{lstlisting}

The important thing to note here is that the head of the body was
collapsed into the constant closure
\texttt{\textless{}1\textgreater{}{[}0{]}}. This is smaller and faster
than encoding it as
\texttt{{[}{[}\textless{}1\textgreater{}\ {[}0{]}{]}\ 1{]}}.

The problem is that laws only guarantee that the expression itself is
evaluated, including evaluating constants to WHNF. So, the law compiler
cannot assume that the closure parameter is available for inspection, it
could very well be a thunk:

\begin{lstlisting}
<1>[thunk]
\end{lstlisting}

The result is that we cannot reliably recognize calls to primops, which
destroys a number of very important optimizations.

In contrast, let's look at the law expression within the new
representation:

\begin{lstlisting}
(Pin x)=(<0> (0 x))

Pin = <{"Pin" 1 [<0> [[0] 1]]}>
\end{lstlisting}

As you can see, the pinned number, the constructor tag, and the arity of
the call are always available for such analysis.

\section{Additional Benefits}\label{additional-benefits}

There are some other slightly useful properties which fall out of this
design.  Let's start by noting that the the new convention is essentially
just an uncurried version of the old one:

\begin{lstlisting}
(<0> i)           -> (<0> (0 i))
(<1> n a b)       -> (<1> (0 n a b))
(<2> p l a z m o) -> (<2> (0 p l a z m o))
\end{lstlisting}

But because PLAN represents arrays using partially applied numbers, all of
our arrays include a zero tag, which is not used for anything.  Instead,
we can take advantage of this tag to group calls into ``modules''.

\begin{lstlisting}
(<0> i)           -> (<0> (0 i))
(<1> n a b)       -> (<0> (1 n a b))
(<2> p l a z m o) -> (<0> (2 p l a z m o))
\end{lstlisting}

Where \textless0\textgreater{} is used for the primitive PLAN values,
\textless1\textgreater{} is used for direct invocations of jets in
XPLAN, \textless2\textgreater{} is used for XPLAN effects, etc.

This has some nice advantages:

\begin{itemize}
\item
  Many fewer pinned nats. Every pinned nat requires a whole pin on the
  heap, which is a small but pointless waste of resources.
\item
  Each type of operation has its own ``namespace''. We can add new XPLAN
  effects without needing to renumber other types of effects.
\end{itemize}

\section{Avoiding Runtime Overhead}\label{avoiding-runtime-overhead}

The new convention is notably less efficient. Instead of simply running
the operation, every call to a primop must now allocate an ADT, and the
actual implementation of the primop must inspect this ADT, handle
edge-cases, and only \emph{then} run the operation. This is pointlessly
inefficient.

However, if we wrap each primop with a simple function:

\begin{lstlisting}
(MkLaw n a b)=(<0> (1 n a b))
\end{lstlisting}

It is extremely simple to recognize and optimize these wrappers, even in
an extremely dumb interpreter. These wrappers functions can then be
optimized to directly run the appropriate operations, which removes all
of the execution overhead.

\section{Conclusion}\label{conclusion}

The new primop calling maintains a common semantics between PLAN and
XPLAN by keeping the arity logic uniform, while still enabling a simple
interpreter to make low-overhead calls into primops. In addition, we
reduce the number of pointless pinned laws, and add the ability to
namespace our effects into categories.

\end{document}
