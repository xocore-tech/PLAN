\documentclass[11pt]{article}

\usepackage{fancyhdr}
\pagestyle{empty}
\fancypagestyle{plain}{\pagestyle{empty}}

\usepackage{listings}
\usepackage[T1]{fontenc}
\usepackage{textcomp}
\usepackage{lmodern} % Latin Modern family (vectorized Computer Modern)
\usepackage{microtype}
\usepackage[paperwidth=210mm,paperheight=2000mm,margin=15mm]{geometry}
\usepackage{parskip}
\usepackage{needspace}
\usepackage[colorlinks=true,linkcolor=blue]{hyperref}

\lstset{
  basicstyle=\ttfamily\small,
  breaklines=true,
  frame=single,
  xleftmargin=1em,   % indent from left
}

\lstdefinelanguage{asm}{
  morecomment=[l]{\#},
}

\lstset{
  language=asm,
  basicstyle=\ttfamily\small,
  commentstyle=\color{black},
  tabsize=8,
  breaklines=false,
  captionpos=b
}

\lstdefinelanguage{Haskell}{
  morekeywords={case,of,if,then,else,data,type,class,where,do,let,in},
  sensitive=true,
  morecomment=[l]--,
  morecomment=[n]{\{-}{-\}},
  morestring=[b]",
}

\lstdefinestyle{haskellstyle}{
  backgroundcolor=\color{backcolour},
  commentstyle=\color{codegreen},
  keywordstyle=\color{blue},
  numberstyle=\tiny\color{codegray},
  stringstyle=\color{codepurple},
  basicstyle=\ttfamily\footnotesize,
  breaklines=true,
  numbers=left,
  numbersep=5pt,
  showstringspaces=false,
  tabsize=2,
}

\lstnewenvironment{asm}[1][] % optional args pass through
  {\lstset{language=asm,#1}}
  {}

\lstnewenvironment{asmdata}[1][] % optional args pass through
  {\lstset{language=asm,#1}}
  {}

\lstnewenvironment{clang}[1][]
                  {\lstset{language=C,#1}}
                  {}


\title{The Wisp Language}
\author{Benjamin Summers}

\begin{document}
\maketitle
\tableofcontents
\begin{abstract}

This document describes the design and implementation of the Wisp
language, an extremely small lisp for PLAN which is used to bootstrap
Sire.

\end{abstract}

\section{Why Boot with Wisp instead of Sire?}

Sire has proven to be a pretty wonderful language.  However, despite
being very small, it has been very difficult to serialize to a seed
without specialized runtime system support for serialization.

The root problem is that PLAN functions have a lot of shared structure,
and the serializer must process *all* of that structure, which ends up
being exponentially large.

In the full production setting, this type of problem is solved using
the memoized \lstinline|PinHash| jet, which allows us to detect sharing
within the pin DAG.  However, this is a fairly complex features, and I
don't want to depend on that in order to bootstrap a seed.

The solution is then to just radically decrease the size, and have a
another, even tinier language in which Sire itself is implemented.

This language can also take advantage of XPLAN in order to get direct
access to primops, without needing to define all of the jets.  This also
makes it possible to run it without jets.

The quine process will be something like:

\begin{itemize}
\item
    \lstinline|./plan wisp.seed <wisp.wisp| to create a new Wisp seed.
\item
    \lstinline|./boot wisp.seed <boot.wisp| to bootstrap a new caching
    Sire REPL.
\end{itemize}

The full bootstrap process will look something like this:

\begin{itemize}
\item Use Wisp to rebuild itself (to confirm the seed).
\item Use Wisp to build the bootstrapping Sire runtime.
\item Use Wisp to build a basic online PLAN compiler with jet matching.
\item Use Sire to build Sire (in the PLAN sandbox, with jet matching)
\item Now we have a working, sandboxed Sire environment.
\end{itemize}

In addition to bootstrapping, Wisp should be useful for defining a suite
of basic runtime system tests.  Since it is a much smaller language,
there are a lot fewer moving parts.

\section{What Features Will Wisp lack?}

These Sire features will not be included:

\begin{itemize}
\item Rex Syntax
\item Printing
\item Error Messages
\item Error Handling generally
\item Modules
\item Inlining
\item Lambda Lifting
\item Let-Optimization
\item Nested lets
\item Pinned Lambdas
\end{itemize}

Sire needs these features because it needs to be expressive enough to
define jets, and to be able to be able to be evolve itself into a fairly
nice language through macros.

However, for just bootstrapping, we only need very basic macros (mostly
just destructuring and pattern matching).  So instead, Wisp will basically
just be an s-expression syntax for PLAN, with a namespace of globals
and a very dumb macro system.

\section{Wisp Syntax}

Core Syntax:

\begin{verbatim}
    ; comments
    exprs :=
         | expr
         | expr exprs
    body :=
         | (@ sym expr) body
         | expr
    args :=
         | sym
         | sym args
    expr :=
        | sym ; attempt to read number if undefined
        | "str"
        | (exprs)
\end{verbatim}

Syntactic Sugar:

\begin{verbatim}
    [..]       -> (BRACKETS ..)
    {..}       -> (CURLIES ..)
    <..>       -> (BRACES ..)
    sym(exprs) -> (sym (exprs))
\end{verbatim}

Pre-Defined Values:

\begin{verbatim}
    PIN = {"PIN" 1 [<0> [[0] 1]]}
\end{verbatim}

Pre-Defined Macros:

\begin{verbatim}
    (function (sym args) body) ->  compute a constant law
    (defmacro sym expr)        ->  binds a macro, returns 0
    (defvalue sym expr)        ->  binds a value, returns 0
    (= (f args) body)          ->  (defvalue f (PIN (function (f args) body)))
    (= nm body)                ->  (defvalue f body)
    (CURLIES x y b)            ->  (function (_ x y) b)
    (CURLIES x y -> b)         ->  (function (_ x y) b)
    (CURLIES f x y => b)       ->  (function (f x y) b)
\end{verbatim}

\section{Wisp State}

The Wisp state is threaded through all macro expansions, and it is just:

\begin{itemize}
\item The next free number (for gensym).
\item A binary search tree of bindings.
\end{itemize}

A binding is a name, a macro value, and an actual value.

\section{Reading Wisp}

\subsubsection{Expression Representation}

\begin{verbatim}
    foo     = %foo      ; symbol
    ()      = 0         ; empty expression
    (x y z) = (0 x y z) ; expression
    "foo"   = (1 %foo") ; constant value
\end{verbatim}

\subsubsection{Reading Expressions}

We characterize expressions by looking up an enum in an array and
switching on it.

\begin{itemize}
\item Zero?  Crash.
\item Space?  Try again.
\item Double quote?  Read a string.
\item Symbol character?  Read a symbol.
\item '('? Read a paren expression.
\item '['? Read a bracket expression.
\item '<'? Read a pointy expression.
\item '{'? Read a curly expression.
\item Comment character?  Read a comment.
\end{itemize}

Reading non-paren expression is the same as reading a paren expression
except with a different termination character and with the result wrapped
in a dummy expression.

\begin{asm}
(= (readParen buf read)   (readNested buf read ")" 0)
(= (readBracket buf read) (Sap "BRACED" (readNested buf read ")" 0)))
(= (readCurly buf read)   (Sap "CURLY"  (readNested buf read "}" 0)))
(= (readPointy buf read)  (Sap "POINTY" (readNested buf read ">" 0)))
\end{asm}

\subsubsection{Reading comments}

This also uses a table.

\begin{itemize}
\item zero? crash.
\item newline? re-enter the reader.
\item anything else? recur.
\end{itemize}

\subsubsection{Reading Symbols}

\begin{itemize}
\item If the next character is a symbolchar, recur.
\item If the next character is a nesting character, read, wrap, return.
\item Otherwise return.
\end{itemize}

\subsubsection{Nested Expressions}

\begin{itemize}
\item Space?  Repeat
\item Term?  Return
\item Otherwise, read a value, collect it, repeat.
\end{itemize}

\begin{asm}
(= (readNested b0 read end acc)
  (@ b1 (eat b0))
  (@ r  (read b1))
  (@ b2 (Hd r))
  (If (Eq end (Word8 b1)) acc
    (Sap (readNested b2 read end)
      (Sap acc (Ix0 r)))))
\end{asm}


\section{Parsing Wisp}

Once we have read a Wisp s-expression, we

\begin{itemize}
\item Expand all macros.
\item Resolve symbols to global bindings.
\item Turn all expressions into applications.
\item Run \lstinline|Force|
\end{itemize}

This is the entire system.  This is just the bare minimum to get
from the language of text into the language of PLANs.  This should be
\textbf{tiny}.

\begin{enumerate}

\item
    The reader is very small
\item
    The parsing logic is trivial.
\item
    The namespace environment is small: just lookup/insert into an
    unbalanced BST.
\item
    Compiling lambdas without lifting or let-optimization to laws is
    very, very little code.
\item
    XPLAN means that we do not need to define our own primitives.
\item
\end{enumerate}

And yet, it is still expressive enough to add basic syntax like [tuples],
destructuring, and pattern matching on ADTs.

\section{The Environment}

Just two trivial operations, lookup and insert.

\begin{asm}
(= (getenv key x[k v m l r])
  (And x
    (Case3 (Compare key k)
       (getenv key l)
       x
       (getenv key r))))

(= (putenv key val mac x[k v m l r])
  (Ifz x [key val mac 0 0]
    (Case3 (Compare key k)
      (Up 3 (putenv key val mac l) x)
      [key val mac l r]
      (Up 4 (putenv key val mac r) x))))
\end{asm}


\subsubsection{Loading ADT Parameters}

The body parameter and capture symbols are used to construct a law for the
branch, which is then called with each parameter and with the captures.
Wisp does not have lambda lifting, so all captures must be explicit.

\section{Compiling Functions}

The only remaining piece of complexity is the built-in function macro:

\begin{asm}
bind = (@ var val)
(function (self args..) bind.. expr)
\end{asm}

Without doing any validation, we parse into a minimal structure:

\begin{verbatim}
    self :: Text
    args :: Row Text
    binds :: (Text, Sexp)
    body :: Sexp
\end{verbatim}

We can now produce a environment for mapping names in the local scope
to numbers.

Next, we run macro-expansion on all expressions.  Finally, we resolve
symbols to local references or constant values (globals).

The result is a n-ariy tree where values are either constant values or
local references.

\begin{verbatim}
Expr :=
    | VAL Plan
    | LOC Nat
    | APP Expr Expr..
Function :=
    name   :: Nat
    args   :: Row Sym
    lets   :: Row (Sym, Expr)
    body   :: Expr
    locals :: Map Sym Nat
\end{verbatim}

From here, we just need to compile the result to law, which is trivial.
The name is the name from the input, the arity is the length of args.
Each expression is transformed into a law expression, and the lets
are chained.

That's all!


\section{Pattern Matching and Destructuring}

These wont be features of Wisp itself, but we need to walk through this
to be sure that this construct works, since building a Sire compiler in
Wisp would be extremely difficult without this.

\subsection{The case macro}

Taking advantage of the fact that ADT constructors are now just numbers,
we can implement a case macro with very little code.

\begin{asm}
constr = symbol
branch = ((constr params..) (captures..) body)
top = (case expr (branch..) fallback)
\end{asm}

Lookup each constr in the environment and find it's value.  If it's not
an atom, crash.  Otherwise, insert into a table of handlers.

Any gaps in the sequence of handler keys gets a nat.  Generate a switch
via Ix.

\subsection{Destructuring}

Destructuring should be implemented by replacing the built-in
implementation of the `Function` macro with one which understands
destructuring.

The logic doesn't change that much.  Each argument and let-binding that
includes a destructuring will turn into a series of let-bindings (and
seqs), so the result still compiles to a law in a straight-forward way.

This feature is why we need gensym, so we will need to generate new
symbols for unnamed patterns.

The syntax supports both \listinline|!a| and \lstinline|![a b c]| syntax,
though the former requires inspecting symbols to see if they start with
an exclaimation.

Instead of \lstinline|foo@[a b c]| we can use the
\lstinline|foo[a b c]|, which is a short-hand for
\lstinline|(foo [a b c])|.

\end{document}
