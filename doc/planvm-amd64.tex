\documentclass[11pt]{article}

\usepackage{fancyhdr}
\pagestyle{empty}
\fancypagestyle{plain}{\pagestyle{empty}}

\usepackage{listings}
\usepackage[T1]{fontenc}
\usepackage{textcomp}
\usepackage{lmodern} % Latin Modern family (vectorized Computer Modern)
\usepackage{microtype}
\usepackage[paperwidth=210mm,paperheight=2000mm,margin=15mm]{geometry}
\usepackage{parskip}
\usepackage{needspace}
\usepackage[colorlinks=true,linkcolor=blue]{hyperref}

\lstset{
  basicstyle=\ttfamily\small,
  breaklines=true,
  frame=single,
  xleftmargin=1em,   % indent from left
}

\lstdefinelanguage{asm}{
  morecomment=[l]{\#},
}

\lstset{
  language=asm,
  basicstyle=\ttfamily\small,
  commentstyle=\color{black},
  tabsize=8,
  breaklines=false,
  captionpos=b
}

\lstdefinelanguage{Haskell}{
  morekeywords={case,of,if,then,else,data,type,class,where,do,let,in},
  sensitive=true,
  morecomment=[l]--,
  morecomment=[n]{\{-}{-\}},
  morestring=[b]",
}

\lstdefinestyle{haskellstyle}{
  backgroundcolor=\color{backcolour},
  commentstyle=\color{codegreen},
  keywordstyle=\color{blue},
  numberstyle=\tiny\color{codegray},
  stringstyle=\color{codepurple},
  basicstyle=\ttfamily\footnotesize,
  breaklines=true,
  numbers=left,
  numbersep=5pt,
  showstringspaces=false,
  tabsize=2,
}

\lstnewenvironment{asm}[1][] % optional args pass through
  {\lstset{language=asm,#1}}
  {}

\lstnewenvironment{asmdata}[1][] % optional args pass through
  {\lstset{language=asm,#1}}
  {}

\lstnewenvironment{clang}[1][]
                  {\lstset{language=C,#1}}
                  {}


\title{PLAN Virtual Machine for AMD64 Implementation Manual}
\author{Benjamin Summers and Elliot Glaysher}

\begin{document}
\maketitle
\tableofcontents
\begin{abstract}

  This document explains the implementation of the AMD64 version of the PLAN
  Virtual Machine. This literate document contains the assembly code with
  documentation and commentary, while also presenting higher level whys. A
  reader should understand the memory layout and registers conventions used by
  the native runtime system.

\end{abstract}
\section{Formatting Conventions}

\subsection{Assembly Code}

Assembly code uses tabs for indentation, and tab between the instruction
and the arguments.  In the rare cases where the instruction is too fat
for this (e.g. \lstinline|lock cmpxchg|), just do whatever you feel like.

Code is written in two columns with pseudocode on the left, aligned to
column 45 using spaces.  The pseudocode is immensely helpful as a quick
reference to describe what is happening on each line, and incredibly
helpful when stepping through code in a debugger.

For aesthetic reasons, every line of assembly should have a \# at
column 45.

More complex explanations should be written in prose, using LaTeX instead
of comments.

\begin{lstlisting}
label:                                      # label:
	mov	rax, rdi                    #   res = x
	add	rax, rsi                    #   res += y
	ret                                 #   return res
\end{lstlisting}

\subsection{C Code}

C code is used for tests and for subsystems which are still being
prototyped.  Writing assembly is relatively painless, but refactoring
it is incredibly time consuming.

C code should be indented using tabs, and aligned using spaces (but
avoid aligning things).

\section{Register Conventions}

\subsection{System V Amd64 Register Conventions}

Just for reference, here are the normal conventions for this platform:

\begin{verbatim}
    ARGS: rdi rsi rdx rcx r8 r9
    RETURN: rax, rdx
    TEMPORARY: rax, r10, r11
    PRESERVED: rbx rbp r12-r15
    SPECIAL: rsp rbp rip
\end{verbatim}

These conventions are not respected, but are still used as the default
conventions on which the actual conventions are based.

\subsection{Global Register Allocations}

Except in specific contexts which need access to a lot of different
registers and don't need to allocate, these registers are always pinned
to global variables:

\begin{verbatim}
    r12 -- scratch buffer, ignored by GC.
    r13 -- pointer to the first word immediately after the heap.
    r14 -- pointer to the next free word on the heap.
    r15 -- pointer to topmost item of the shadow stack.
\end{verbatim}

While it is still possible to use r12 in contexts which require a lot
of registers, a lot of macros clobber r12, so you must be careful.

\subsection{Calling Pins and Laws}

Pins and Laws share a single calling convention, which is similar to the
normal amd64 but extended with self-reference, and with no callee-saved
registers.

\begin{verbatim}
    SELF: rax -- self reference (pin or law)
    ARGS: rdi rsi rdx rcx r8 r9
    OUT: rax
\end{verbatim}

TODO: It may be worthwhile to extend the set of arguments used for
registers here, since we clobber all registers anyways.

\subsection{Calling Thunks}

Thunks are evaluated by calling (or jumping to) the code pointer stored
in the first slot of a thunk.  Thunks contain all of the information
they need in order to be evaluated, and they clobber all registers. The
calling convention is just:

\begin{verbatim}
    SELF: rax
    OUT: rax
\end{verbatim}

\subsection{Calling Asm From C}

We use some C code for debugging tools, and for some things that we are
still prototyping.  But C code can leave registers in invalid states,
so we need to use adapters in order to correctly call into the runtime
system from C.

\begin{asm}
.global zzcal2
zzcal2:	jmp zcall2

.global zcall0, zcall1, zcall2, zcall3, zcall4, zcall5

zcall0: xor	rsi, rsi
zcall1: xor	rdx, rdx
zcall2: xor	rcx, rcx
zcall3: xor	r8d, r8d
zcall4: xor	r9d, r9d
zcall5:	sub	rsp, 32
	mov	[rsp], r12                  # save r12, rbx, rbp
	mov	[rsp+8], rbx
	mov	[rsp+16], rbp
	mov	rax, rdi                    # self=arg1 (function pointer)
	mov	rdi, rsi                    # arg1=arg2
	mov	rsi, rdx                    # arg2=arg3
	mov	rdx, rcx                    # arg3=arg4
	mov	rcx, r8                     # arg4=arg5
	mov	r8, r9                      # arg5=arg6
	xor	r9d, r9d                    # clear r9
	xor	r10d, r10d                  # clear r10
	xor	r11d, r11d                  # clear r11
	xor	r12d, r12d                  # clear r12
	xor	ebx, ebx                    # clear rbx
	xor	ebp, ebp                    # clear rbp
	call	rax                         # Call into asm routine
	mov	r12, [rsp]
	mov	rbx, [rsp+8]
	mov	rbp, [rsp+16]               # restore r12, rbx, rbp
	add	rsp, 32
	ret
\end{asm}


\section{Convenience Macros}

\subsection{pdrop}

\begin{asm}
.macro pdrop n
	.set bytes, 8 * \n
	add	r15, bytes
.endm
\end{asm}

Drops a certain number of items from the shadow stack.

\subsection{ppush}

Pushes zero or more registers to the shadow stack.

\begin{asm}
.macro ppush regs:vararg
	.ifnc "\regs", ""

	.set n\@, 0
	.irp r, \regs
	.set n\@, n\@ + 1
	.endr

	.set total, 8 * n\@

	sub     r15, total

	.set offset\@, 0
	.irp r, \regs
		mov	qword ptr [r15 + offset\@], \r
		.set offset\@, offset\@ + 8
	.endr
	.endif
.endm
\end{asm}

\subsection{ppop}

Restores a sequence of registers from the shadow stack.  Note that the
order corresponds with the order using for \lstinline|ppush|.  If you
\lstinline|ppush a, b, c|, you also \lstinline|ppop a, b, c|

\begin{asm}
.macro ppop regs:vararg
	.ifnc "\regs", ""

	.set n\@, 0
	.irp r, \regs
	.set n\@, n\@ + 1
	.endr

	.set total, 8 * n\@

	.set offset\@, 0
	.irp r, \regs
		mov	\r, qword ptr [r15 + offset\@]
		.set offset\@, offset\@ + 8
	.endr

	add     r15, total

	.endif
.endm
\end{asm}

\subsection{Block Saves}

These are only used in the division logic, in order to push multiple
values to the C stack.  This should probably eventually be replaced with
a generic macro like ppush and ppop.

\begin{asm}
.macro pop_r8_to_rdi
	pop	r8
	pop	rcx
	pop	rdx
	pop	rsi
	pop	rdi
.endm
\end{asm}

\begin{asm}
.macro push_rdi_to_r8
	push	rdi
	push	rsi
	push	rdx
	push	rcx
	push	r8
.endm
\end{asm}

\subsection{sub0}

sub0 is unsigned subtraction with a floor at 0.  If the subtraction
would underflow, the result is zero.

\begin{asm}
.macro sub0 a, b
	cmp	\a, \b
	cmovb	\a, \b
	sub	\a, \b
.endm
\end{asm}

\subsection{cap}

\lstinline|cap(r,max)| just computes \lstinline|r=min(r,max)| in order
to keep r within a certain bounds.  Both inputs must be in registers.

\begin{asm}
.macro cap r, max
	cmp	\r, \max
	cmova	\r, \max
.endm
\end{asm}

\subsection{jzero}

Jump to a label if a register is equal to zero.

\begin{asm}
.macro jzero reg, label
	test	\reg, \reg
	jz	\label
.endm
\end{asm}

\subsection{btw}

\begin{asm}
.macro btw r1                               # bits_to_words(x) = (x+63)>>6
	add	\r1, 63
	shr	\r1, 6
.endm
\end{asm}

\subsection{swapq}

Using two temporary registers \lstinline|r1, r2|, swap the values behind
two locations in memory.

\begin{asm}
.macro swapq r1, r2, m1, m2
	mov	\r1, \m1
	mov	\r2, \m2
	mov	\m1, \r2
	mov	\m2, \r1
.endm
\end{asm}

\subsection{swapstk}

Using two temporary registers \lstinline|r1, r2|, swap the two to values
on the stack.

\begin{asm}
.macro swapstk r1, r2
	swapq	\r1, \r2, [r15], [r15+8]
.endm
\end{asm}

\section{Pointer Tagging}

Numbers that fit within 63-bits are stored unboxed, directly in the value.
A high bit being set indicates that the value is a heap reference.

Heap references are represented as normal pointer but with tagging data
stored in the high 16 bits.  The following conventions are used:

\begin{verbatim}
    u63 0nnnnnnnnnnnnnnn - 0x0000..0x7fff (n=nat data)
    NAT 1000001000000000 - 0x8200
    PIN 11000100000000cm - 0xC40x (c=hascrc32, m=ismegapin)
    LAW 1100100000000000 - 0xC800
    CLZ 11010000ttttzzzz - 0xD0tz (t=tag, z=size)
    THK 1110000000000000 - 0xE000
\end{verbatim}

the m flag indicates that a pin is a megapin, which is a pinned pin.
Megapins cache information about all of their subpins.

the c flag indicates that the pin has been hashed using crc32 (which is
directly supported by the CPU).  Pins that have this information are
bigger, having the hash data append to the end of the structure.

If the closure tag information is 0b1111, that indicates that the tag
is too big and must be loaded from the closure itself.

If the closure size is 0b0000, that indicates that the size is too big
and must be loaded from the closure itself.  0 is chosen because 0 is
not a valid closure size.

\subsection{Essential Properties}

This scheme works well because it maintains a number of essential
properties:

\begin{enumerate}
\item
The first property is that direct numbers are encoded directly in the
normal way, without any need to encode or decode them.  This also makes
for very tight fast-paths for numeric primops.
\item
The second property is that all of the information needed in order to
determine the type is available in the high byte.  This makes it possible
to perform comparisons on it, after roll with al, r12b, etc.  It also
make it very visible in hex-printouts.
\item
The third property, because the actual representations are ordered
by type, we can recognize ranges of possible types with a comparison.
For example, if the high-bit is greater than or equal to the smallest
possible closure value, then the result is either a thunk or a closure.
Similarly, anything smaller than a pin is a number; either direct or
indirect.
\item
The fourth property is that each heap reference can be determined by
checking a specific bit.  This only works if we already know that the
reference is not a direct number, but if we do know that we can do the
determination with a single `bt` instruction.
\item
The fifth property, is that each type of heap reference has a different
number of leading zeros, so we can use lzcnt to convert from a heap
reference to an enum of all possible heap types.
\item
The sixth property is that, the metadata associated with a type also
lives in the second highest byte, which has the same advantages as having
the type live in the top-most byte.
\end{enumerate}

\subsection{Operations on Heap References}

\subsubsection{Extracting the Raw Pointer}

Converting a tagged value into a pointer just requires two shifts to
zero out the high bits.

\begin{asm}
.macro ptr r                                # r = PTR(r)
	shl	\r, 16                      # lsh clears high bits
	shr	\r, 16                      # rsh recovers original shape
.endm

.macro ptr_ r, from                         # r = FROM(from)
	mov	\r, 48
	bzhi	\r, \from, \r		    # mask the bottom 48 bits
.endm

.macro dref r                               # r = *PTR(r)
	ptr	\r
	mov	\r, [\r]
.endm

.macro dref_ r from                         # r = *PTR(from)
	mov	\r, \from
	dref	\r
.endm

.macro refo r, o                           # r = PTR(r)[i]
	ptr	\r
	lea	\r, [\r + \o]
.endm

.macro refo_ r, src, o                     # r = PTR(src)[i]
	mov	\r, \src
	refo	\r, \o
.endm

.macro drefo r, o                           # r = PTR(r)[i]
	ptr	\r
	mov	\r, [\r + \o]
.endm

.macro drefo_ r, src, o                     # r = PTR(src)[i]
	mov	\r, \src
	drefo	\r, \o
.endm

.macro clzix_ r, src, i
	.set	ix, 8+(\i * 8)
	drefo_	\r, \src, ix
.endm

.macro drefi r, i                           # r = PTR(r)[i*8]
	ptr	\r
	mov	\r, [\r + 8 * \i]
.endm

.macro drefi_ r, src, i                     # r = PTR(str)[i*i]
	mov	\r, \src
	ptr	\r
	mov	\r, [\r + \i * 8]
.endm
\end{asm}

\subsubsection{Extracting an Type Enum}

In cases where we want to use a jump-table to switch on each possible
type instead of just branching, we need to convert from a heap-reference
into an enum:

\begin{verbatim}
    thunk=0, closure=1, law=2, pin=3, nat=4
\end{verbatim}

The implementation works by taking advantage of the fact that each heap
reference type has a unique number of zeros.

This ends up not being used very often, since there are only a few cases
where we want to handle each possible type in different way.

\begin{asm}
.macro tytag r                            # tytag(r):
	shl	\r, 2                     #   r <<= 2
	lzcnt	\r, \r                    #   r = clz(r)
.endm
.macro tytag_ out, from                   # tytag_(out,from):
	mov	\out, \from               #   out = inp
	tytag	\out                      #   out = tytag(out)
.endm
\end{asm}

\subsubsection{Direct Numbers}

In order to determine if a number is a direct atom or a heap reference,
we can examine the high bit, something which the architecture has direct
support for.

\lstinline|jdirect| jumps to a label if the the register is a direct
number, and \lstinline|jheap| jumps if it is anything else.

\begin{asm}
.macro jdirect reg, label                   # jump if direct
	test	\reg, \reg                  # examine register
	jns	\label                      # not signed == high bit clear
.endm
.macro jheap reg, label                     # jump if indirect
	test	\reg, \reg                  # examine register
	js	\label                      # signed == high bit set
.endm
\end{asm}

\subsubsection{Numbers}

Since direct and indirect numbers are both smaller values than everything
else, as long as the high byte is smaller than the smallest byte possible
for a pin, then the value is a number.

\lstinline|jinat| jumps to a label if the register is a nat,
\lstinline|jnnat| jumps if it's not.  Both clobber r12.

\lstinline|ncast| casts an already-evaluated register into a nat, clobbering
r11 and r12.

\begin{asm}
.macro jinat reg, lab                       # jump if nat
	rorx	r12, \reg, 56               # rorx to get high byte
	cmp	r12b, 0xC4                  # smaller than a pin?
	jb	\lab                        # then jump to the label
.endm
.macro jnnat reg, lab                       # jump if not nat
	rorx	r12, \reg, 56               # rorx to get high byte
	cmp	r12b, 0xC4                  # big enough to be a pin?
	jae	\lab                        # then jump to the label
.endm
.macro ncast reg                            # cast to nat
	xor	r11d, r11d                  # r11=0
	rorx	r12, \reg, 56               # r12 = (reg>>56)
	cmp	r12b, 0xC4                  # if (r12 >= 0xC4)
	cmovae	\reg, r11                   #   reg=0
.endm
\end{asm}


\subsubsection{Thunks, App, Laws}

Thunks are the largest values, so we can identify them by checking if
their high bit is within a certain range.

\lstinline|jithk| jumps t a label if the value is thunk, and
\lstinline|jnthk| jumps if it is not.

\lstinline|jiclzt| jumps t a label if the value is thunk or a closure,
and \lstinline|jnclzt| jumps if it is anything else.


\begin{asm}
.macro jithk reg, lab
	rorx	r12, \reg, 56
	cmp	r12b, 0xE0
	jae	\lab
.endm
.macro jnthk reg, lab
	rorx	r12, \reg, 56
	cmp	r12b, 0xE0
	jb	\lab
.endm
.macro jiclzt reg, lab                      # Jump if closure or thunk
	rorx	r12, \reg, 56
	cmp	r12b, 0xD0
	jae	\lab
.endm
.macro jnclzt reg, lab
	rorx	r12, \reg, 56
	cmp	r12b, 0xD0
	jb	\lab
.endm
.macro jiclz reg, lab                       # Jump if closure.
	rorx	r12, \reg, 56
	cmp	r12b, 0xD0
	je	\lab
.endm
.macro jnclz reg, lab                       # Jump if not closure.
	rorx	r12, \reg, 56
	cmp	r12b, 0xD0
	jne	\lab
.endm
.macro jipin reg, lab                       # Jump if closure.
	rorx	r12, \reg, 56
	cmp	r12b, 0xC4
	je	\lab
.endm
.macro jnpin reg, lab                       # Jump if not closure.
	rorx	r12, \reg, 56
	cmp	r12b, 0xC4
	jne	\lab
.endm
.macro jilaw reg, lab                       # Jump if closure.
	rorx	r12, \reg, 56
	cmp	r12b, 0xC8
	je	\lab
.endm
.macro jnlaw reg, lab                       # Jump if not closure.
	rorx	r12, \reg, 56
	cmp	r12b, 0xC8
	jne	\lab
.endm
\end{asm}

There are also slightly more efficient variants which can be used if we
know that the input is not a direct number:

\subsubsection{Laws}

\begin{asm}
.macro _jiclz reg, lab # Jump if closure (assuming it is not a nat)
	bt	\reg, 60
	jc	\lab
.endm

.macro _jilaw reg, lab # Jump if law (assuming it is not a nat)
	bt	\reg, 59
	jc	\lab
.endm

.macro _jnlaw reg, lab # Jump if not law (assuming it is not a nat)
	bt	\reg, 59
	jnc	\lab
.endm
\end{asm}

\subsubsection{Specific Closure Shapes}

Because the tag and size information of closures are available in the tag
for small values, we can recognize specific closures shapes by comparing
the high 16 bits to specific values.

This is especially useful when generating optimized code for functions
which switch various properties of ADTs.

But it's also useful in a few particular cases within the runtime system
itself, since we need to be able to do this in order to identify patterns
within law bodies.

\begin{asm}
.macro jnkal reg, lab                       # Jump if not closure.
	mov	r12, \reg
	shr	r12, 48
	cmp	r12w, 0xD002
	jne	\lab
.endm
\end{asm}

\begin{asm}
.macro jnquo reg, lab                       # Jump if not closure.
	mov	r12, \reg
	shr	r12, 48
	cmp	r12w, 0xD001
	jne	\lab
.endm
\end{asm}

\subsection{Asserts and Debugging}

We are a complex virtual machine with complex internal invariants. So we have
several macros which do nothing if the source was compiled with
the \lstinline|NDEBUG| preprocessor flag, but otherwise perform light validity
checking on pointers and objects to catch invariant violations.

These checks should be light enough that they can be left in a release binary
and precise enough that they will never false positive.

TODO: Actually transition everything over to .S files which do C preprocessor
expansion so we can hide these on release builds.

\subsubsection{Assertion Failure}

We jump to here to trap in the debugger if things have gone wrong.

\begin{asm}
assert_fail:
	ud2
\end{asm}

\subsubsection{Assert Buddy List Pointer}

Asserts that reg is within the buddy mmap, including the prelude to the buddy
heap that includes the bucket control. This is used for asserting validity of
the list pointers at runtime.

\begin{asm}
.macro ablptr reg                             # assert buddy list pointer
	cmp	\reg, buddy_mmap_ptr
	jb	assert_fail
	cmp	\reg, base_end_ptr
	jae	assert_fail
.endm
\end{asm}

\subsubsection{No 0xBADBEEFBADBEEF Values}

Validates that the register does not contain the sentinel 0xBADBEEFBADBEEF value.

TODO: An imm64 is kind of heavyweight. Can we just check the lower word and
then jump on low word to the rest of the check somehow?

\begin{asm}
.macro nobeef reg
	movabs	r12, 0xBADBEEFBADBEEF
	cmp	\reg, r12
	je	assert_fail
.endm
\end{asm}

\section{Heap Object Layout}

The word immediately preceding each heap object is a GC record which
includes it's size and type.  This is needed by the first-generation GC
because Cheney's algorithm needs to do a linear walk of the heap.

At the moment, this size is *also* used to encode the size of numbers and
closures, but in the future these will be stored separately.  Because of
this, the GC size is stored in bits, not words.

The GC header also has a separate type for NF closures and WHNF closures,
which is used to cache \lstinline|Force| operations to avoid duplicate work.

\subsection{GC Headers}

The GC header is always the word right before the one pointed at.
It contains a type and a bit-size, which is always a multiple of 64 for
things besides nats.

\begin{asm}
.macro getgc r                              # gcheader(r) = PTR(r)[-8]
	ptr	\r
	mov	\r, [\r - 8]
.endm

.macro gcsz hdr                             # Size of allocation box
	add	\hdr, 16128
	shr	\hdr, 14
.endm

.macro gcsz_ reg, hdr
	mov	\reg, \hdr
	gcsz	\reg
.endm

.macro gcbits hdr
	shr	\hdr, 8
.endm
\end{asm}

We define a number of wrapper macros to formalize the usage patterns
for extracting data from this word in different ways.

\begin{asm}
.macro bitsiz r
	getgc	\r
	gcbits	\r
.endm

.macro bitsiz_ r, from
	mov	\r, \from
	bitsiz	\r
.endm

.macro bytsiz r
	bitsiz	\r
	add	\r, 7
	shr	\r, 3
.endm

.macro bytsiz_ r, from
	mov	\r, \from
	bytsiz	\r
.endm

.macro dbitsz r
	xchg	\r, r12
	mov	\r, 64
	lzcnt	r12, r12
	sub	\r, r12
.endm

.macro dbitsz_ r, from
	lzcnt	r12, \from
	mov	\r, 64
	sub	\r, r12
.endm

.macro dbytsz_ r, from
	dbitsz_	\r, \from
	add	\r, 7
	shr	\r, 3
.endm

.macro wordsz r
	getgc	\r
	gcsz	\r
.endm

.macro wordsz_ r, from
	mov	\r, \from
	wordsz	\r
.endm

.macro slots_ r, from
	mov	\r, \from
	slots	\r
.endm

.macro slots r                              # wordsize, assuming not a nat
	getgc	\r
	shr	\r, 14
.endm

.macro clzsz r                              # Sz(x) = slots(x)-1
	slots	\r
	dec	\r
.endm

.macro clzsz_ r, from
	mov	\r, \from
	clzsz	\r
.endm

\end{asm}

\subsection{Extracting Data from Heap Objects}

Getting the tag of a closure is just a dereference.

\begin{asm}
.macro clzhd_ r, from
	dref_	\r, \from
.endm

.macro ix0 r
	drefo	\r, 8
.endm

.macro ix0_ r, from
	drefo_	\r, \from, 8
.endm

.macro ix1 r
	drefo	\r, 16
.endm
.macro ix1_ r, from
	drefo_	\r, \from, 16
.endm
.macro unpin r
	dref	\r
.endm
.macro unpin_ r, from
	dref_	\r, \from
.endm
.macro name r
	dref	\r
.endm
.macro name_ r, from
	dref_	\r, \from
.endm
.macro arity r
	drefo	\r, 8
.endm
.macro arity_ r, from
	drefo_	\r, \from, 8
.endm
.macro body r
	drefo	\r, 16
.endm
.macro body_ r, from
	drefo_	\r, \from, 16
.endm
\end{asm}

\subsection{Big Numbers}
TODO
\subsection{Closures}
TODO
\subsection{Thunks}
TODO
\subsection{Liquid Pins}
TODO
\subsection{Frozen Pins}
TODO
\subsection{Mega Pins}
TODO
\subsection{Claim and Reserve}
TODO
\subsection{Pending Changes}
Once we add support for non-normalized pins and laws on the 1st-gen heap,
we will need to have a NF flag for pins and laws as well.

TODO: explicit nat and closure size to enable in-place shrinking.

\section{Allocation}

\subsection{mkbuffer}

This creates a new bignat directly on GC2 in order to prevent it from
moving around.  This allows us to get pointers into nats for use during
syscalls, and for casting to function pointers in order to generate code
for pins and laws.

This takes a single argument: the number of bytes that we will need.

The process here is that we allocate one byte more than needed and set
the high byte to 1 (to create a BAR of the right size).

\lstinline|buddy_malloc| does not zero the memory, so we do that.  It also
does not set the actual bit-size, and we need that, so we overwrite the
GC header it creates with our own.

\begin{asm}
.global mkbuffer
mkbuffer:                                   # mkbuffer: (rdi=size)
	mov	r12, rdi                    #   r12=bytes
	mov	rbx, 8                      #
	cmp	rdi, rbx                    #   size = max(size, 8)
	cmovb	rdi, rbx                    #
	shr	rdi, 3                      #   words = bytes/8
	inc	rdi                         #   words++
	mov	rbx, rdi                    #   rbx=words
	xor	esi, esi                    #   tag=0
	call	buddy_malloc                #   buddy_malloc() (rax=ptr)
	mov	rsi, rax                    #   copy pointer to rsi
	mov	rcx, rbx                    #   count=words
	mov	rdi, rax                    #   dst=ptr
	xor	eax, eax                    #   value=0
	rep	stosq                       #   memcopy
	mov	byte ptr [rsi+r12], 1       #   set high byte
	mov	rax, 0x8200000000000000     #   result=NATMASK
	or	rax, rsi                    #   result |= ptr
	shl	r12, 3                      #
	inc	r12                         #   bits = (bytes*8)+1
	shl	r12, 8                      #   hdr = (bits, tag=0)
	mov	[rsi-8], r12                #   replace header w/ correct sz
	ret                                 #   return result
\end{asm}

TODO: make sure this is actually collected propertly.

TODO: make sure that GC2 correctly handles an actual bit-size, not just
a word-size wrapped in a bit-size.  If not, we can just provide a fake
GC header as the first word within this allocation.  We are going to
have to do that eventually, once we include the mark and the intrinsic
linked list.

\subsection{mkword}

\lstinline|mkword| takes any 64-bit number, and returns a valid PLAN
value.  The result will either be a direct number, or a bignum depending
on the size.

Because this is often used to finalize a result, this uses the
unconventional calling convention of having both the argument and the
return value in rax.  Only r8 is clobbered.

\begin{asm}
mkword:                                     # mkword: (rax=word)
	test	rax, rax                    #   if high bit is set
	js	mkword64                    #     tailcall mkword64(word)
	ret                                 #   return
\end{asm}

\subsection{mkword64}

\lstinline|mkword64| is similar to \lstinline|mkword| except that the
input is required to have it's high bit set, and so the output is always
a single-word bignum.

Because this is often used to finalize a result, this uses the
unconventional calling convention of having both the argument and the
return value in rax.  Only r8 is clobbered.

Note that the GC header is set to \lstinline|0x4000| which is tag=0,
size=64: \lstinline|0x4000 = 64<<8 + 0|.  It is always this size, because
all valid single-word heap-allocated nats are exactly 64-bits wide.

The pointer tag is 0x8200000000000000, which is just the bit-pattern
for NAT with no metadata.

\begin{asm}
mkword64:                                   # mkword64: (rax=word)
	lea	r8, [r14+16]                #   tmp = hp+2
	cmp	r8, r13                     #   if (tmp > heapend)
	ja	mkword64.gc                 #     goto gc
	xchg	r8, r14                     #   ptr=hp, hp=tmp
	mov	qword ptr [r8], 0x4000      #   gc record
	add	r8, 8                       #   ptr++
	mov	[r8], rax                   #   *ptr = word
	mov	rax, 0x8200000000000000     #   result = mask
	or	rax, r8                     #   result |= word
	ret                                 #   return result
mkword64.gc:                                # gc:
	not	rax                         #   hide 64bit word via negation
	ppush	rdi                         #   save rdi
	mov	rdi, 2                      #   need two words
	call	gc                          #   gc()
	ppop	rdi                         #   restore rdi
	not	rax                         #   restore 64bit word
	jmp	mkword64                    #   try again
\end{asm}

\subsection{mkdouble}

This allocates a two-word NAT, allocating a heap object where necessary.

Note that both input arguments may be in GC-unsafe states, since unpacked
word64s can look like heap references to the GC.  We need to be very
careful to hide these from GC and we need to make sure they are cleared
before we return.

This follows the unconventional calling convention where the arguments
are in rax and rdx, and the return value is in rax.

\begin{asm}
mkdouble: # rax=lo, rdx=hi                  # mkdouble128:
	test	rdx, rdx                    #   if (rdx == 0)
	jz	mkword                      #     return mkword(rax)
mkdouble128:                                # mkdouble128:
	lea	rsi, [r14+8*3]              #
	cmp	rsi, r13                    #   if (heap_next > heap_end)
	ja	mkdouble128.gc              #     then gc
	xchg	rsi, r14                    #   swap (heap_next, rsi)
	mov	[rsi+8], rax                #   rsi[1] = low_word
	mov	[rsi+16], rdx               #   rsi[2] = high_word
	lzcnt	rdx, rdx                    #
	mov	rax, 128                    #
	sub	rax, rdx                    #   rax = 128 - clz(high_word)
	shl	rax, 8                      #
	mov	[rsi], rax                  #   RECORD (t=nat/0 sz=bitsz)
	add	rsi, 8                      #   res = &gc_ptr[1]
	mov	rax, 0x8200000000000000     #
	or	rax, rsi                    #   rax=(res | 0x8200000000000000)
	ret                                 #   return (rax+rdx both safe)
mkdouble128.gc:                             #
	push	rax                         #   Hide 64-bit words on c stack
	push	rdx                         #
	xor	rax, rax                    #
	xor	rdx, rdx                    #
	ppush	rdi                         #
	mov	rdi, 3                      #   needed words
	call	gc                          #
	ppop	rdi                         #
	pop	rdx                         #
	pop	rax                         #
	jmp	mkdouble128                 #
\end{asm}

\section{Numeric Utilities}

\subsection{openat}

Given any PLAN value, coerces the input into a number (non-numeric values are
cast to 0), and returns a pointer to the u64[] containing the number along with
it's bit-width (special-casing 64 if 0).

This routine is used for numeric operations, and it solves tow problems at
once:

\begin{enumerate}
\item
Handle the edge-case of a non-numeric input being passed to a operation on
natural numbers. (All PLAN jets handle this by casting non-nat value to 0).
\item
Converts direct and indirect natural numbers into a uniform representation,
which makes it possible to use the same code for all combinations of
representations (indirect - indirect), (direct - indirect), (indirect -
direct), and (direct - direct).
\end{enumerate}

There is one special cases, which is a bit of a hack: When the input is zero,
we actually return 64 for the bit-size.  In practice, we will always use the
bit-size to compute the word-size of the resulting buffer, which is always one,
even if the input is zero.

\begin{asm}
opennat: # rdi=plan value,rsi=buffer	    # openat:
	jdirect	rdi, opennat.direct
opennat.indirect:
	bitsiz_	rax, rdi		    # rax = bitsize(rdi/bitnat)
	ptr_	rdx, rdi		    # rdx = ptr(rdi/ref)
	ret
opennat.direct:
	mov	[rsi], rdi
	mov	rdx, rsi	            # *tmp=word, rdx=temp
	mov	rax, 64         	    # rax=64
	test	rdi, rdi
	jz	opennat.direct_zero_sized
	lzcnt	rcx, rdi
	sub	rax, rcx
opennat.direct_zero_sized:
	ret
\end{asm}

\subsection{unpack}

Unpacks the top two items of the plan stack into the following standardized
parts of a stack frame. This function has the unusual calling convention that
it reads the top two items from the shadow stack, and then unpacks those items
into the caller's stack frame:

\begin{table}[h]
\begin{tabular}{|l|l|}\hline
Offset & Description               \\\hline
-8     & x pointer tag             \\
-16    & x qword buffer for direct \\
-24    & x bitsize                 \\
-32    & x buffer                  \\
-40    & x buffer word size        \\
-48    & y pointer tag             \\
-56    & y qword buffer for direct \\
-64    & y bitsize                 \\
-72    & y buffer                  \\
-80    & y buffer word size        \\\hline
\end{tabular}
\end{table}

\begin{asm}
unpack:                                     # unpack:
	mov	rdi, [r15+8]                #   arg1
	mov	[rbp-8], rdi                #   rbp[-8] = arg1
	mov	rsi, [r15]                  #   arg2
	mov	[rbp-48], rsi               #   rbp[-48] = arg2
	lea	rsi, [rbp-16]               #
	call	opennat                     #
	mov	[rbp-24], rax               #
	mov	[rbp-32], rdx               #
	btw	rax                         #
	mov	[rbp-40], rax               #
	mov	rdi, [rbp-48]               #
	lea	rsi, [rbp-56]               #
	call	opennat                     #
	mov	[rbp-64], rax               #   rbp[-64] = arg2.bits
	mov	[rbp-72], rdx               #
	btw	rax                         #
	mov	[rbp-80], rax               #   rbp[-80] = arg2.words
	ret
\end{asm}

\section{Pinning}

Pinning is our system for taking a tree of values up to the next pin boundary
and compacting it into a single contiguous segment, recording metadata about

\subsection{Hash Table}

Because pins can be arbitrarily large, we must define data structures that can
grow during pin construction to perform the size calculations, pointer mapping,
and pointer deduplication. We implement a simple linear probing hash table that
maps a qword to a qword, where the entire structure can be directly freed at
the end.

\begin{verbatim}
    typedef struct { uint64_t key, val; } entry_t;
    typedef struct {
   	uint64_t lg;      // exponent of the power of 2 size of table
   	uint64_t mask;    // mask for current size to apply to hashes
   	uint64_t count;   // number of items, tracked for resizing.
   	entry_t tbl[];    // flexible array
    } ht_t;
\end{verbatim}

\subsubsection{ht\_create}

We create a hash table by passing in the exponent of the power of two desired
capacity of the table.

\begin{asm}
.global ht_create
ht_create:				    # ht_create(rdi=lg):
	mov	rcx, rdi		    #   move to rcx fro shl
	mov	rsi, 1			    #   base 1
	shl	rsi, rcx		    #   size = 1 << lg
	ppush	rdi, rsi, rcx		    #   save registers
	mov	rdi, rsi		    #   bytes = size
	shl	rdi, 1			    #   bytes *= 2 (keys+vals)
	add	rdi, 3			    #   bytes += 3 (header words)
	mov	rsi, 0x008		    #   tag=Hash Table
	call	buddy_malloc		    #   buddy_malloc -> rax=buffer
	ppop	rdi, rsi, rcx		    #   restore registers
	mov	[rax], rdi		    #   h->lg = lg;
	mov	rcx, rsi		    #   mask = power of two size
	dec	rcx			    #   mask-- (now a mask of 1s)
	mov	[rax+8], rcx		    #   h->mask = size - 1
	inc	rcx			    #   restore size
	mov	qword ptr [rax+16], 0	    #   h->count = 0
	mov	rsi, rax		    #   save return value
	lea	rdi, [rsi+24]		    #   rdi = beginning of kv buffer
	shl	rcx, 1			    #   rcx = size * 2
	xor	rax, rax		    #   set value to 0
	rep	stosq			    #   zero out table memory
	mov	rax, rsi		    #   restore return value
	ret				    #   return
\end{asm}

\subsubsection{htkey}

Given the base pointer and an index, write the key value at idx. We do things
this way because it's cleaner at the call site to operate on masking the index.

\begin{asm}
.macro htkey base, idx, out
	mov	\out, \idx
	shl	\out, 4		# idx * 16; can't rdx*16 inside mov.
	mov	\out, [\base + \out + 24] # base->tbl[idx].key
.endm
\end{asm}

\subsubsection{htkpos}

Given the base pointer and an index, write the pointer to the key at idx.

\begin{asm}
.macro htkpos base, idx, out
	mov	\out, \idx
	shl	\out, 4		# idx * 16; can't rdx*16 inside mov.
	lea	\out, [\base + \out + 24] # base->tbl[idx].key
.endm
\end{asm}

\subsubsection{htval}

Like \lstinline|htkey|, but points to the value.

\begin{asm}
.macro htval base, idx, out
	mov	\out, \idx
	shl	\out, 4		# idx * 16; can't rdx*16 inside mov.
	mov	\out, [\base + \out + 32] # base->tbl[idx].val
.endm
\end{asm}

\subsubsection{ht\_has}

Returns the index at which a key exists, or -1 if key isn't found.

\begin{asm}
.global ht_has
ht_has:					    # ht_has(rdi=ht*,rsi=key):
	mov	rax, -1			    #   initialize hash to -1
	crc32	rax, rsi		    #   crc32 on key value as hash
	mov	r8, [rdi + 8]		    #   r8 = h->mask
	and	rax, r8			    #   idx = hash & h->mask
ht_has.loop:				    # loop:
	htkey	rdi, rax, rcx		    #   rcx = hdi->tbl[rax].key
	test	rcx, rcx		    #   if key is zero
	jz	ht_has.not_found	    #   then item not found
	cmp	rcx, rsi		    #   if key is requested key
	je	ht_has.return		    #   then found and return idx
	inc	rax			    #   idx++
	and	rax, r8			    #   use mask to loop around
	jmp	ht_has.loop		    #   loop
ht_has.not_found:			    # not_found:
	mov	rax, -1			    #   set return value to -1
ht_has.return:				    # return:
	ret				    #   return
\end{asm}

\subsubsection{ht\_put}

Sets \lstinline|key| to \lstinline|value|, overwriting the current value if one
is set.

When we put a value in a hash table, the first thing we have to do is make sure
that the hash table is less than 70\% full. If it is, we invoke the resizing
behaviour, which doubles the size and rehashes everything.

\begin{asm}
ht_put:					    # ht_put(rdi=ht*, rsi=key, rdx=val):
	mov	rax, [rdi + 16]		    #   rax = tbl->count
	inc	rax			    #   add 1 for new count
	cvtsi2sd xmm0, rax		    #   convert to double in xmm0
	mov	rax, [rdi + 8]		    #   rax = tbl->mask
	inc	rax			    #   rax is now capacity
	cvtsi2sd xmm1, rax		    #   convert to double in xmm1
	divsd   xmm0, xmm1                  #   xmm0 = (h->count++)/(h->mask++)
	movsd	xmm1, qword ptr [hash_table_load_factor]
	ucomisd xmm0, xmm1		    #   if percent < load_factr
	jbe     ht_put.has_valid_hash_table #   jump if not greater
	ppush	rdi, rsi, rdx		    #   save registers
	call	ht_resize		    #   resize the table
	ppop	rdi, rsi, rdx		    #   restore registers
	mov	rdi, rax		    #   rdi is now resized table
ht_put.has_valid_hash_table:		    # has_valid_hash_table:
	mov	rax, -1			    #   initialize crc32
	crc32	rax, rsi		    #   hash the key
	mov	r8, [rdi + 8]		    #   load mask
	and	rax, r8			    #   idx = hash & h->mask
ht_put.loop:				    # loop:
	mov	r9, rax			    #   r9 = idx
	shl	r9, 4			    #   r9 = idx * 16
	lea	r9, [rdi + r9 + 24]	    #   r9 = key location
	mov	r10, [r9]		    #   r10 = key value
	test	r10, r10		    #   if key value is null
	jz	ht_put.increment	    #   then increment count and store
	cmp	r10, rsi		    #   if key value is search key
	je	ht_put.store		    #   then store here w/o increment
	inc	rax			    #   idx++
	and	rax, r8			    #   overflow by mask if needed
	jmp	ht_put.loop		    #   goto loop
ht_put.increment:			    # increment:
	mov	r8, [rdi + 16]		    #   get current h->count
	inc	r8			    #   increment
	mov	[rdi + 16], r8		    #   set updated h->count
ht_put.store:				    # store:
	mov	[r9], rsi		    #   current location = key
	mov	[r9 + 8], rdx		    #   current location++ = val
	ret				    #   return
\end{asm}

We also have to store the floating point value for 0.7 in a place in rodata so
it's loadable.

\begin{asmdata}
.section .rodata
.align 8
hash_table_load_factor:
	.double	0.7
.section .data
\end{asmdata}

\subsubsection{ht\_resize}

Resizing the hash table is incrementing the logarithmic size by 1 and then
copying each item over.

\begin{asm}
ht_resize:				    # ht_resize(rdi=ht*):
	push	rdi			    #   save input hash table
	mov	rdi, [rdi]		    #   new_lg = ht->lg
	inc	rdi			    #   new_lg++
	call	ht_create		    #   rax = ht_create(new_lg)
	pop	rdi			    #   restore input hash table
	mov	r8, [rax + 8]		    #   newh->mask
	mov	r10, [rdi + 16]		    #   oldh->count
	mov	[rax + 16], r10		    #   newh->count = oldh->count
	mov	rsi, [rdi + 8]		    #   get old mask as terminator
	inc	rsi			    #   rsi = old table size
	mov	rdx, 0			    #   i = 0
ht_resize.outer_loop:			    # loop:
	htkey	rdi, rdx, rcx		    #   rcx = oldh->tbl[i].key
	test	rcx, rcx		    #   if key is zero
	jz	ht_resize.continue	    #   then try next item
	mov	r10, -1			    #   initialize crc32 hashing
	crc32	r10, rcx		    #   hash the key
	and	r10, r8			    #   dst = crc32(0, k) & newh->mask
ht_resize.inner_loop:			    # inner_loop:
	htkey	rax, r10, r11		    #   r11 = newh->tbl[dst].key
	test	r11, r11		    #   if dst's key is zero
	jz	ht_resize.inner_loop_target #   then place item here
	inc	r10			    #   dst++
	and	r10, r8			    #   dst = (dst + 1) & newh->mask
	jmp	ht_resize.inner_loop	    #   retry next slot
ht_resize.inner_loop_target:		    # inner_loop_target:
	htkpos	rax, r10, r11		    #   r11 = &(newh->tbl[dst].key)
	htkpos	rdi, rdx, rcx		    #   rcx = &(oldh->tbl[i].key)
	mov	r10, [rcx]		    #   read key
	mov	[r11], r10		    #   write key
	mov	r10, [rcx + 8]		    #   read value
	mov	[r11 + 8], r10		    #   write value
ht_resize.continue:			    # continue:
	inc	rdx			    #   i++
	cmp	rdx, rsi		    #   if i < size of old table
	jb	ht_resize.outer_loop	    #   then lop
	push	rax			    #   save new table
	call	buddy_free		    #   free old table
	pop	rax			    #   restore new table
	ret				    #   return
\end{asm}

\section{Evaluation}

\subsection{Macros}

\subsubsection{EVAL}

EVAL Causes target to be evaluated, result in target, clobbers
rax and r12.  tosave registers are preserved

\begin{asm}
.macro EVAL target, tosave:vararg
	mov	rax, \target
	jnthk	rax, 777f
	ppush	\tosave
	dref_	r12, rax
	call	r12
	mov	\target, rax
	ppop	\tosave
777:
.endm
\end{asm}

\subsubsection{ERAX}

This is a specialized version of EVAL, causes rax to be evaluated,
clobbers r12.  ALl of the registers in tosave are preserved

\begin{asm}
.macro ERAX tosave:vararg
	jnthk	rax, 777f
	ppush	\tosave
	dref_	r12, rax
	call	r12
	ppop	\tosave
777:
.endm
\end{asm}

\subsubsection{JRAX}

JRAX evaluates and returns rax (uses a tail-call if rax is a thunk).

\begin{asm}
.macro JRAX
	jnthk	rax, 777f
	dref_	r12, rax
	jmp	r12
777:
	ret
.endm
\end{asm}

\subsubsection{FORCE}

This evaluates the target to normal form.

\begin{asm}
.macro FORCE target, tosave:vararg
	ppush	\tosave
	mov	rax, \target
	call	force.newabi
	mov	\target, rax
	ppop	\tosave
.endm
\end{asm}

\subsubsection{ENAT}

ENAT evaluates a register and converts it into nat, clobbing r11
and r12.

\begin{asm}
.macro ENAT reg, tosave:vararg
	EVAL	\reg, \tosave
	ncast	\reg
.endm
\end{asm}


\subsection{Thunk Executioners}

\subsubsection{xdone}

xdone is used for thunks which have already been evaluated, it just
returns the cached values, which is in the first slot.

Note that the garbage collector automatically recognizes thunks of this
form, ignores any extra slots, and shrinks the result.

\begin{asm}
xdone:
	drefo	rax, 8
	ret
\end{asm}

\subsubsection{xhole}

This i used for thunks which are in the process of being evaluated.
This makes it possible to detect cases where an evaluation depends on
it's own result.

TODO: Should GC also shrink these down to one word, and not treat any
of the slots as live?

\begin{asm}
xhole:
	mov	rdi, 1 # stdout
	lea	rsi, [loopstr]
	mov	rdx, 9
	call	syscall_write
	mov	rsi, 1
	jmp	syscall_exit_group
loopstr: .string "<<loop>>\n"
\end{asm}

\subsubsection{xvar}

This is an executioner for a dummy thunk which wraps another value with
may also be a thunk.  This is just used to handle certain edge cases
that come up when implementing LETREC.

\begin{asm}
xvar:
	ppush	rax
	drefo	rax, 8
	ERAX
	ppop	rdi           # rdi=thunk
	ptr	rdi           # rdi=PTR(thunk)
	mov	[rdi+8], rax
	lea	rsi, [xdone]
	mov	[rdi], rsi
	ret
\end{asm}

\subsubsection{xhead}

\lstinline|xhead| is a specialized executioner for computing the Init
of a closure.  The size of the closure must be at least 2.

This is used during the implementation of Open primop, so that pattern
matching on a large closure doesn't need to materialize the head unless
we actually use it.

For example, if the input is \lstinline|{xhead, 3[4 5 6]}| then the
thunk will be replaced with \lstinline|{xdone, 3[4 5]}| and the return
value will be \lstinline|3[4 5]|.

\begin{asm}
xhead:
	ppush	rax                         # save thunk
	drefo	rax, 8                      # rax=(a b)
	clzsz_	rcx, rax                    # n = rax.size
	ppush	rax                         # save oldclz
	dref_	rdi, rax                    # arg1 = oldclz.ptr
	lea	rsi, [rcx-1]                # arg2 = n-1
	call	closure                     # rax = newclz
	ppop	rsi                         # src = restore(oldclz)
	ptr	rsi                         # src = oldclz.ptr
	ptr_	rdi, rax                    # dst = newclz.ptr
	mov	r8b, [rsi-8]                # r8 = oldclz.heaprecord.type
	mov	[rdi-8], r8b                # newclz.heaprecord.type = r8
	rep	movsq                       # memcpy(newclz.ptr, oldclz.ptr, n)
	ppop	rdi                         # rdi = restore thunk
	ptr	rdi                         # rdi = thunk.ptr
	mov	[rdi+8], rax                # thunk[1] = newclz
	lea	rsi, [xdone]
	mov	[rdi], rsi                  # thunk[0] = xdone
	ret                                 # return newclz
\end{asm}

\subsubsection{xunknown}

This is a wrapper around xunknownnoupdate which performs thunk update.

\begin{enumerate}
\item
    First, the thunk is updated to be a blackhole by replacing the
    executioner, which will cause evaluation loops to crash.
\item
    Then, we call xunknownnoupdate to do the actual evaluation.
\item
    Then we update the thunk again to by replacing the executioner with
    xdone and assigning the first slot to the result.
\end{enumerate}

\begin{asm}
xunknown:
	lea	rdx, [xhole]
	ptr_	r10, rax
	mov	[r10], rdx
	ppush	rax
	call	xunknownnoupdate
	ppop	r10               # thunk
	ptr	r10
	lea	rdx, [xdone]
	mov	[r10], rdx
	mov	[r10+8], rax
	ret
\end{asm}

\subsubsection{xunknownnoupudate}

Some TODOs:

\begin{itemize}
\item Branch on the size hand small sizes using registers (not stack).
\item Specialized entry-points for small sizes.
\end{itemize}

\begin{asm}
xunknownnoupdate:
thapple:
	ptr_	r8, rax        # r8 = ptr thunk
	mov	rcx, [r8-8]    # fetch RECORD
	shr	rcx, 14        # heapsz (in words)
	dec	rcx            # size = heapsz-1 (ignoring fp)
	cmp	rcx, 2
	je	thapple1
	cmp	rcx, 3
	je	thapple2
	cmp	rcx, 4	       # TODO: use a jump table (for arities up to 6)
	je	thapple3
	mov	r11, rcx
	mov	r9, rcx
	shl	r9, 3          #
	sub	r15, r9        # sp -= size (*8 for bytes)
	lea	rsi, [r8+8]    # src = &thunk[1]
	mov	rdi, r15       # dst = stack
	rep	movsq          # copy
	lea	rdi, [r11-1]
	jmp	apply          # return apply(args)
thapple1:
	mov	rax, [r8+8]
	mov	rdi, [r8+16]
	jmp	apply1
thapple2:
	mov	rax, [r8+8]
	mov	rdi, [r8+16]
	mov	rsi, [r8+24]
	jmp	apply2
thapple3:
	mov	rax, [r8+8]
	mov	rdi, [r8+16]
	mov	rsi, [r8+24]
	mov	rdx, [r8+32]
	jmp	apply3
\end{asm}

\section{PLAN vs XPLAN Primops}

PLAN itself only offers three primops: Pin, Law, and Open.

However, the runtime itself implements XPLAN, which greatly extends
this set of operations, as well as providing a number of mechanisms for
interacting the the host operating system and for directly poking the
process itself.

When running PLAN code, access to the lawful subset of these operations
is done by doing jet matching in the online compiler, but in XPLAN these
are invoked directly as primitives.

\section{Numeric Primops}

Numeric operations need to handle both direct numbers and indirect
numbers, and they also need to evaluate their inputs if they are thunks.

The most common case for these operations is to be called with small
numbers, and the architecture supports many of these operations natively.
Because of this, these are generally organized as a fast path and a slow
path, where the fast path attempts to perform the machine instruction
with just enough extra machinery to detect the edge-cases and fallback
to a more general, slower version of the routine.

Also, there is usually a general purpose entry-point which accepts any
PLAN value and a specialized version that expects to be given evaluated
numbers.  The idea here, is that an optimizer will often be optimizing
chains of numeric operations, and it can use these specialized routines
to avoid all of the overhead around handling evaluation thunks and casting.

\subsection{Nil}

We expose a quick check for zero equality, returning one if the input value is
zero.

\begin{asm}
opnil:                                      # opnil(rdi=val):
	EVAL	rdi                         #   evaluate
fastnil:                                    # fastnil:
	xor	eax, eax                    #   set return value to 0
	mov	edx, 1                      #   possible return value of 1
	test	rdi, rdi                    #   is val 0?
	cmovz	eax, edx                    #   set return value to 1 if 0
	ret                                 #   return
\end{asm}

\subsection{ToBit}

The opposite of Nil is ToBit, which casts a value to 0 or 1.

\begin{asm}
optobit:                                    # optobit(rdi=val):
	EVAL	rdi                         #   evaluate
fasttobit:                                  # fasttobit:
	xor	eax, eax                    #   set return value to 0
	mov	edx, 1                      #   possible return value of 1
	test	rdi, rdi                    #   is val not zero?
	cmovnz	eax, edx                    #   set return value to 1 if not 0
	ret                                 #   return
\end{asm}

\subsection{Nat}

Casts the input to a natural number.

\begin{asm}
opnat:                                      # opnat(rdi=n):
	mov	rax, rdi                    #   set to rax for evaluation
	ERAX                                #   evaluate rax
	ncast	rax                         #   cast result to nat
	ret                                 #   return
\end{asm}

\subsection{Increment}

The increment logic offers a number of entry-points with different
invariants.

\begin{itemize}
\item
  \lstinline|opinc|: The actual increment primop, which takes a single
  argument which can be any PLAN value.

\item
  \lstinline|fastinc|: The same, but with the precondition that the
  argument is in WHNF and is a number.

  This works by using the \lstinline|add| instruction to increment, which
  sets the sign bit if high bit of the result is set.  That condition
  handles two cases at once: the input is indirect, and the input is
  direct, but overflows into 64 bits.  We don't have to worry about
  overflow here, because (-1) is not a valid PLAN value.

\item
  \lstinline|slowinc|: Handles the case where the result of the increment
  overflows into a bignum.

  The preconditions here are that rax must be rdi+1, and that the high
  bit of rax is set.  So, if the high bit is unset on rdi, then we know
  that the specific values is 0x8000000000000000.  Otherwise, we are
  incrementing a bignum.

\item
  \lstinline|bufinc|: Handles an increment of a bignum where it is
  known that there is enough space to hold the result (propagates the
  carry-bit).

\end{itemize}
\begin{asm}
.global fastinc
opinc:
	ENAT	rdi                         # evaluate+cast
fastinc:
	mov	rax, rdi                    # result = input
	add	rax, 1                      # result += 1
	js	slowinc                     # if indirect, slowinc
	ret                                 # fast path: return (x+1)
slowinc:
	jdirect	rdi, mkword64               # if direct(x) return mkword64(x+1)
	xor	eax, eax                    # clear invalid register state
	call	reservecopy                 # rax/ptr, rdx/sz = reservecopy(x)
	mov	r10, rdx                    # r10 = wordsz
	mov	rdi, rax                    # arg = resultptr
	call	bufinc                      # bufinc
	lea	rdi, [r10+1]                # arg = wordsz+1
	jmp	claim                       # return claim(arg)
bufinc.loop:                                # rax:Ptr[Word] -> (clobbers rax)
	add	rax, 8                      # rax = &rax[1]
bufinc:
	add	qword ptr [rax], 1          # *rax += 1
	jc	bufinc.loop                 # if overflow, repeat
	ret                                 # return
\end{asm}

\subsection{Decrement}

Some notes on \lstinline|bufdec|:

This does an in-place decrement of a buffer.

Input: rax: pointer to the data in x.

Output: Rax is clobbered, There is no return value.

Invariant: The buffer must encode a non-zero number, otherwise this
will clobber memory outside the buffer (note the lack of a size input).

\begin{asm}
.global fastdec
opdec:
	ENAT	rdi                         # Evaluate + cast
fastdec:
	jheap	rdi, slowdec                # If indirect, slowpath
	xor	eax, eax                    # result = 0
	dec	rdi                         # rdi--
	cmovns	rax, rdi                    # if unsigned, result=rdi
	ret                                 # return result
slowdec:
	call	reservecopy                 # rax=pt rdx=sz {rdi rsi rcx}
	call	bufdec                      # ptr > () {rax}
	mov	rdi, rdx                    # arg1 = sz
	jmp	claim                       # rax=Val {rax rdi rsi r8}
bufdec:
	sub	qword ptr [rax], 1          # (*buf)--
	lea	rax, [rax+8]                # buf++
	jc	bufdec                      # if underflow, loop
	ret                                 # otherwise return
\end{asm}

\subsection{Addition}

\subsubsection{opadd}

Add is the general purpose version which works with any PLAN input.
Note that we evaluate the second argument first, since that's how the
jet definition works out.

\begin{asm}
opadd:
	ENAT	rsi, rdi                    # evaluate y + cast (saving x)
	ENAT	rdi, rsi                    # evaluate x + cast (saving y)
	# fallthrough to fastadd
\end{asm}

\subsubsection{fastadd}

fastadd is the fast path for addition, and it has the pre-condition
that both arguments are already natural numbers.  It attempts to do
direct addition, but falls back to the slow path in all cases where that
isn't enough.  There are four potential cases here:

\begin{itemize}
\item Both inputs are direct, and the result is direct.
\item Both inputs are direct, but result is too big.
\item One of the inputs is direct, and the other is indirect.
\item Both of the inputs are indirect.
\end{itemize}

We want to return in the first case and fallback in the other cases.
In all of the other cases, either the addition will overflow, or the
high bit will be set on the result.

\begin{asm}
.global fastadd
fastadd:                                    # rdi=x rsi=y
	mov	rax, rdi                    #   res = x
	add	rax, rsi                    #   res += y
	jc	slowadd                     #   overflow?
	js	fastadd.maybeoverflow       #   high bit set?
	ret                                 #   return
fastadd.maybeoverflow:                      # rax=res rdi=x rsi=y
	mov	rdx, rdi                    #
	or	rdx, rsi                    #   if direct(x|y):
	jns	mkword64                    #     tailcall mkword64(res)
	# fallthrough to slowadd
\end{asm}

\subsubsection{slowadd}

In this case, we know that at least one input is indirect.

We can handle both direct+indirect cases using the same code.  One of
the cases just swaps registers and then jumps to the other case.

In the direct+indirect case: we make a copy of the bignat iwth
\lstinline|reservecopy|, call \lstinline|bufadd1|, and then finalize
the result with \lstinline|claim|.

in the indirect+indirect case, we need to make sure that the first
paramter is longer, then copy the first parameter into a result buffer,
and perform the addition using \lstinline|bufadd|.

Note that in both cases \lstinline|reservercopy| reserves and zeros
an extra word for us, which we need for the overflow case, and
\lstinline|claim| handles shrinking the result if that ends up not
being used.

\begin{asm}
slowadd:
	xor	eax, eax                    # clear invalid register state
	jdirect	rdi, slowadd.directx        # if x is direct, goto directx
	jdirect	rsi, slowadd.directy        # if y is direct, goto directy
slowadd.slowpath:
	wordsz_	r8, rdi                     # r8 = x.sz
	wordsz_	r9, rsi                     # r9 = y.sz
	cmp	r9, r8                      # if (y.sz > x.sz)
	mov	rax, rdi                    #
	cmova	rdi, rsi                    #     swap x and y
	cmova	rsi, rax                    #
	mov	r10, r8                     #
	cmova	r8, r9                      #     swap xsz, ysz
	cmova	r9, r10                     #
	ppush	rsi                         # save y
	call	reservecopy                 # rax=buf rdx=xsz
	mov	r11, rdx                    # stash xsz in r11
	ppop	rdi                         # restore y
	mov	rcx, r9                     # count = y.sz
	ptr	rdi                         # input = y.buf
	call	bufadd                      # bufadd(buf, y.buf, count)
	lea	rdi, [r11+1]                # may have grown by one via carry.
	jmp	claim                       # construct result.
slowadd.directx:                            # rdi=word rsi=big
	xchg	rdi, rsi                    #   swap big/word
slowadd.directy:                            # rdi=big rsi=word
	mov	r11, rsi                    #   r11=word (prevent clobber)
	call	reservecopy                 #   rdi=bignat
	mov	rdi, rax                    #   rdi=ptr
	mov	rsi, r11                    #   rsi=word
	call	bufadd1                     #   bufadd1(ptr, word)
	lea	rdi, [rdx+1]                #   rdi = xsz+1
	jmp	claim                       #   return claim(rdi)
\end{asm}

\subsubsection{bufadd}

bufadd adds two bignats, mutating the first one in-place.  The second
argument cannot be wider than the first, and the first must contain
enough space to handle an overflow bit.

\begin{verbatim}
Input: rdi = output buffer (*u64)
Input: rsi = add buffer (*u64)
Input: rdx = size of add buffer in words (u64)
Outputs: none
Clobbers: rcx and rax
\end{verbatim}

This just loops over the words of the add buffer and adds each word to
the output buffer while propagating a carry bit forward.  If the carry
bit is set at the end, we call \lstinline|bufinc| to propagate the final
bit forward.

\begin{asm}
bufadd:                                     # rax=dst rdi=src rcx=n
	clc                                 #   clear carry
1:                                          # loop:
	mov	r8, [rdi]                   #   load src word
	adc	[rax], r8                   #   dst[i] += src[i]+CF; sets CF
	lea	rax, [rax+8]                #   rax++
	lea	rdi, [rdi+8]                #   rdi++
	dec	rcx                         #   n--
	jnz	1b                          #   if (n != 0) goto loop
	jc	bufinc                      #   if (carry) goto bufinc(rax)
	ret                                 #   return
\end{asm}

\subsubsection{bufadd1}

bufadd1 adds a single word to a bignat, mutating the bignat in-place.
The bignat must have enough space to deal with a potential overflow bit.

\begin{verbatim}
Input: rdi: is the buffer
Input: rsi: the word to add
Outputs: none
Clobbers: rax
\end{verbatim}

This works by just doing the add, setting rax to the next word, and
then calling bufinc if the addition overflowed.  Note that this has
a very minimal register footprint.  We take arguments in rdi and rsi,
but we only clobber rax.

\begin{asm}
bufadd1:                                    # rdi=dst rsi=word
	add	[rdi], rsi                  # Perform the addition in-place
	lea	rax, [rdi+8]                # nx = &buffer[1]
	jc	bufinc                      # if (overflow) tailcall bufinc(nx)
	ret                                 # return
\end{asm}

\subsection{Subtract}

\subsubsection{opsub}

\begin{asm}
opsub:
	ENAT	rdi, rsi
	ENAT	rsi, rdi
	# fallthrough to fastsub
\end{asm}

\subsubsection{fastsub}

\begin{asm}
.global fastsub
fastsub:
	mov	rax, rdi
	or	rax, rsi  # Careful!  rax value is gc-unsafe in the slow path.
	js	slowsub
	xor	rdx, rdx
	sub	rdi, rsi
	cmovc	rdi, rdx
	mov	rax, rdi
	ret
\end{asm}

\subsubsection{slowsub}

\begin{asm}
slowsub:
	xor	eax, eax # Clear GC-unsafe register
	jdirect	rsi, slowsub.yword
	mov	r10, rdi             # stash rdi/rsi in unclobbered regs
	mov	r11, rsi
	call	compare              # compare arguments
	mov	rdi, r10
	mov	rsi, r11             # restore rdi/rsi
	cmp	rax, 2
	jne	slowsub.zero         # if not a>b, result is 0
	call	reservecopy          # rax = rBuf, rdx = a.wordSz
	mov	r10, rdx             # save size
	ptr_	rdi, r11
	wordsz_	rcx, r11
	call	bufsub               # bufsub(rBuf, PTR(b), b.wordSz)
	mov	rdi, r10             # restore size
	jmp	claim                # return claim(resSz)
slowsub.zero:
	xor	eax, eax
	ret
slowsub.yword:
	mov	r11, rsi
	call	reservecopy
	mov	rdi, rax
	mov	rsi, r11
	call	bufsub1
	mov	rdi, rdx
	jmp	claim
\end{asm}

\subsubsection{bufsub1}

\begin{asm}
bufsub1: # rdi/a:Ptr[Word] rsi/b:Word (clobbers rax)
	sub	[rdi], rsi
	jc	bufsub1.underflow
	ret
bufsub1.underflow:
	lea	rax, [rdi+8]
	jmp	bufdec
\end{asm}

\subsubsection{bufsub}

This performs an in-place subtraction of two indirect atoms:

\begin{verbatim}
    x = (x - y)

    ARG rax: pointer to the data in x.
    ARG rdi: pointer to the data in y.
    ARG rcx: number of words in y.

    Four registers are clobbered: rax, rdi, rcx, and r8.

    There is no return value, just the mutation of the data behind rax.
\end{verbatim}

There is an important precondition: x must always be larger than y.

The *actual value* of x must be larger than the value of y, not just
the buffer size.  This precondition enables us to avoid needing to futz
around with the size of x.

Note that the \lstinline|dec,jnz| sequence is equivalent to the
\lstinline|loop| instruction, but that is much slower on some machines.

\begin{asm}
bufsub:                                     # bufsub: (rax/x rdi/y rcx/yWords)
	clc                                 #   Clear carry
bufsub.loop:                                # loop:
	mov	r8, [rdi]                   #   r8 = y[i]
	sbb	[rax], r8                   #   x[i] -= r8
	lea	rdi, [rdi+8]                #
	lea	rax, [rax+8]                #
	dec	rcx                         #   i--
	jnz	bufsub.loop                 #   if (i != 0) loop
	jc	bufdec                      #   If underflow, decrement
	ret                                 #   Otherwise return.
\end{asm}

\subsection{Multiply}

\subsubsection{opmul}

TODO: is this evaluation order correct?

\begin{asm}
opmul:
	ENAT	rdi, rsi                    #
	ENAT	rsi, rdi                    #
	# fallthrough to fastmul            #
\end{asm}

\subsubsection{fastmul}

We don't have any special tricks for mul, except that the actual machine
operation produces two words, so there is no overflow case.

We just check if either input is indirect by using OR and then checking
the high bit.

\begin{asm}
.global fastmul
fastmul:                                    # fastmul:
	mov	rax, rdi                    #   tmp = arg1
	or	rax, rsi                    #   tmp |= arg2
	js	slowmul                     #   if either indirect, slowmul
	mov	rax, rdi                    #   a = x
	mov	rdx, rsi                    #   b = y
	mul	rdx                         #   rax, rdx = a*b
	jmp	mkdouble                    #   return mkdouble(rax, rdx)
\end{asm}

\subsubsection{slowmul}

\lstinline|slowmul| has the precondition that at least on argument
is indirect.

TODO: can rawreserve be used instead of reserve?

TODO: can we avoid needing to stash things on the stack before reserve
by moving things into different registers?

\begin{asm}
slowmul:
	xor	eax, eax                    # Clear out gc-unsafe register rax
	jdirect	rdi, slowmul.directx
	jdirect	rsi, slowmul.directy
slowmul.indirect:
	mov	rcx, rsi                    # rcx   = y (n)
	mov	rsi, rdi                    # rsi   = x (m)
	wordsz_	r8, rsi                     # nx/r8 = x.wordsz (msz)
	wordsz_	r9, rcx                     # ny/r9 = y.wordsz (nsz)
	lea	rdi, [r8+r9]                # rsz   = nx + ny
	push	rdi                         # save rsz
	ppush	rcx                         # save y
	call	reserve                     # Clobbers rax, rdi, rcx
	ppop	rcx                         # restore y
	ptr	rcx                         # rcx = n (y.buf)
	ptr	rsi                         # rsi = m (x.buf)
	lea	rdi, [r14+8]                # rdi = &hp[1]
	call	bufmul                      # call bufmul
	pop	rdi                         # restore rsz
	jmp	claim                       # return claim(rsz)
slowmul.directx:
	xchg	rdi, rsi
slowmul.directy:
	xor	eax, eax                    # result = 0
	cmp	rsi, 1
	ja	1f                          # if (y <= 1):
	cmove	rax, rdi                    #     if y=1, result=x
	ret                                 #     return result
1:	mov	rdx, rdi                    # rdx=xref (avoid clobber)
	wordsz	rdi
	mov	r8, rdi                     # r8=xsz   (avoid clobber)
	inc	rdi                         # reserve(wordsz(x) + 1)
	call	reserve                     # Clobbers rdi, rcx, rax
	ptr_	r9, rdx                     # r9 = xbuf
\end{asm}

\subsubsection{bufmul1}

\begin{asm}
bufmul1:                                    # bufmul1:
	xor	ecx, ecx                    #   rcx/carry = 0
	xor	edi, edi                    #   rdi/i     = 0
bufmul1.loop:                               # loop:
	mov	rax, rsi                    #   lo    = yword
	mul	qword ptr [r9+8*rdi]        #   hi,lo = lo*xbuf[i]
	add	rax, rcx                    #   lo    = lo+carry
	adc	rdx, 0                      #   hi    = hi+cf
	mov	rcx, rdx                    #   carry = hi
	inc	rdi                         #   i++
	mov	[r14+8*rdi], rax            #   hp[i] = lo
	cmp	rdi, r8                     #   if (i < xsz)
	jb	bufmul1.loop                #     goto loop
	inc	rdi                         #   i++
	mov	[r14+8*rdi], rcx            #   hp[i] = carry
	jmp	claim                       #   tailcall claim(i)
\end{asm}

\subsubsection{bufmul}

\begin{verbatim}
# Input: rdi -- out
# Input: rsi -- m (xptr)
# Input: rcx -- n (yptr)
# Input: r8  -- msz
# Input: r9  -- nsz
\end{verbatim}

\begin{asm}
bufmul:
	push	r12
	push	r13
	xor	r10, r10                  # i=0
	xor	r13, r13                  # carry=0
bufmul.iloop:
	xor	r11, r11                  # j=0
bufmul.jloop:
	# clc                             # clear carry (pointless?)
	mov	rax, [rsi+8*r10]          # lo = m[i]
	mul	qword ptr [rcx+8*r11]     # hi:lo = lo*n[j]
	lea	r12, [r10+r11]            # i+j
	add	rax, r13                  # lo += carry
	adc	rdx, 0                    # hi += hi+cf
	add	[rdi+8*r12], rax          # out[i+j] += lo
	adc	rdx, 0                    # rdx += cf
	mov	r13, rdx                  # carry=hi
	inc	r11                       # j++
	cmp	r11, r9                   # if (j < nsz)
	jb	bufmul.jloop              #     continue
	inc	r12                       # (i+j)++
	add	[rdi+8*r12], r13          # out[i+j] += carry
	xor	r13, r13                  # carry=0
	inc	r10                       # i++
	cmp	r10, r8                   # if i < mSz
	jb	bufmul.iloop              #     continue outer
	pop	r13                       # pop r13
	pop	r12                       # pop r12
	ret                               # return
\end{asm}

\subsection{Div}
\subsubsection{opdiv}

The actual operation.  It just evaluates+casts and calls fastdiv.

\begin{asm}
.global opdiv
opdiv: # rdi=x rsi=y
	ENAT	rdi, rsi
	ENAT	rsi, rdi
	# fallthrough to fastdiv
\end{asm}

\subsubsection{fastdiv}

This is the fastpath for direct inputs.

We only have to check x, because if x is direct and y is indirect, then
we can pretend that y is direct because we will still get the right
result. Proof:

\begin{verbatim}
    if x<y then x/y = 0
        (a property of integer division)

    if direct(x) && !direct(y) then x<y
        (indirect numbers are bigger)

    forall a b. if direct(a) && !direct(b), then rawword(b) > a.
        (because the raw word of b has the high bit set, but it is never
        set on direct numbers)

    Therefore forall a b. if direct(a) then a/b == a/rawword(b).
\end{verbatim}

We take advantage of this to avoid switching on the shape of b, and just
always use it's raw register form.

TODO: One divide by zero we should error out with a resource exhaustion
error, but that isn't a concept yet.

\begin{asm}
fastdiv:                     # fastdiv: (rdi=x, rsi=y)
	jheap	rdi, slowdiv #   if indirect, goto slowdiv
	test	rsi, rsi     #   if (y != 0)
	jnz	1f           #     then okay
	ud2                  #     else crash (divide by zero)
1:	xor	edx, edx     #   high word = 0
	mov	rax, rdi     #   low word = x
	div	rsi          #   rax,rdx = x/y
	ret
\end{asm}

\subsubsection{slowdiv}

\begin{verbatim}
# Precondition: rdi is an indirect nat.
# Precondition: rsi is a nat
# Precondition: rsi is non-zero.
#
# Case one: rsi is direct, but rdi is indirect.
#
# Case two: both rdi and rsi are indirect.
#
# In case two, if (rsi >= rdi), the result is 0.
\end{verbatim}

\begin{asm}
slowdiv: # rsi=x rdi=y -> rax=(x/y)
	jdirect	rsi, slowdiv.y_direct
	mov	r8, rdi
	mov	r9, rsi
	call	compare    # clobbers rax, rdi, rsi, rcx
	mov	rdi, r8
	mov	rsi, r9
	cmp	rax, 2     # if x>y goto x_greater
	je	slowdiv.x_greater
	ret                # if x<y return 0 else return 1
slowdiv.y_direct:
slowdiv.x_greater:
	ppush	rsi, rdi
	push	rbp
	mov	rbp,rsp
	add	rsp, -120
	call	unpack
	mov	rdi, [rbp - 24]
	sub	rdi, [rbp - 64]
	mov	[rbp - 88], rdi
	inc	rdi
	btw	rdi
	mov	[rbp - 96], rdi
	mov	rsi, [rbp - 40]
	add	rdi, rsi
	add	rdi, rsi
	call	reserve
	mov	[rbp - 104], rax
	mov	rdi, [rbp - 96]
	lea	rsi, [rax + rdi * 8]
	mov	[rbp - 112], rsi
	add	rdi, [rbp - 40]
	lea	rsi, [rax + rdi * 8]
	mov	[rbp - 120], rsi
	call	unpack
	mov	rdi, [rbp - 112]
	mov	rsi, [rbp - 32]
	mov	rcx, [rbp - 40]
	rep	movsq
	mov	rdi, [rbp - 120]
	mov	rsi, [rbp - 72]
	mov	rdx, [rbp - 80]
	mov	rcx, [rbp - 88]
	call	buflsh
	mov	rdi, [rbp - 112]
	mov	rsi, [rbp - 120]
	mov	rdx, [rbp - 40]
	mov	rcx, [rbp - 104]
	mov	r8, [rbp - 88]
	call	divloop
	mov	rdi, [rbp - 96]
	call	claim # TODO drop+leave AND THEN tail-call into claim
	pdrop	2
	leave
	ret
\end{asm}

\subsubsection{divloop}

\begin{asm}
divloop:
	cmp	r8, 0
	js	divloop.done
	push_rdi_to_r8
	call	divstep.gte
	pop_r8_to_rdi
	test	rax, rax
	jz	divloop.endloop
	push_rdi_to_r8
	mov	rax, rdi
	mov	rdi, rsi
	mov	rcx, rdx
	call	bufsub
	pop_r8_to_rdi
	mov	r9, r8
	shr	r9, 6
	mov	r10, r8
	and	r10, 63
	mov	r11, [rcx + 8*r9]
	bts	r11, r10
	mov	[rcx + 8*r9], r11
divloop.endloop:
	push_rdi_to_r8
	mov	rdi, rsi
	mov	rsi, rdx
	call	bufshr1
	pop_r8_to_rdi
	mov	rdx, rax
	dec	r8
	jmp	divloop
divloop.done:
	ret
divstep.gte:
	dec	rdx
divstep.gte_loop:
	mov	rax, [rdi + rdx*8]
	cmp	rax, [rsi + rdx*8]
	jb	divstep.gte_cannot_subtract
	ja	divstep.gte_can_subtract
	dec	rdx
	cmp	rdx, 0
	jge	divstep.gte_loop
divstep.gte_can_subtract:
	mov	rax, 1
	ret
divstep.gte_cannot_subtract:
	mov	rax, 0
	ret
\end{asm}

\subsection{Left Shift}

\subsubsection{oplsh}

\begin{asm}
oplsh:                                      # oplsh: (rdi=x rsi=shift)
	ENAT	rdi, rsi                    #   eval+cast x
	ENAT	rsi, rdi                    #   eval+cast shift
	# fallthrough to fastlsh
\end{asm}

\subsubsection{fastlsh}

fastlsh uses a single branch to determine if it can use the `lsh`
operation safely.  If the shift is less than the number of leading zeros
on the input, then it is safe to use the fast path.  This handles all
of the edge-cases at once.

\begin{itemize}
\item
    If the input is indirect, it will have no leading zeros.  Since zeros
    is not greater than anything, this always exits the fast-path.
\item
    If the shift is indirect, then the number will be much greater than
    64, the maximum number of leading zeros.
\item
    If the shift would overflow into two words, then the shift would be
    greater than the number of leading zeros, exiting the fast path.
\end{itemize}

\begin{asm}
.global fastlsh
fastlsh:                                    # fastlsh: (rdi=x, rsi=shift)
	lzcnt	r8, rdi                     #   num high zeros
	cmp	rsi, r8                     #   if (shift >= lzcount)
	jae	slowlsh                     #     goto slowlsh
	mov	rcx, rsi                    #   count = shift
	mov	rax, rdi                    #   res = x
	shl	rax, cl                     #   res <<= x
	ret                                 #   return res
\end{asm}

\subsubsection{slowlsh}

If we fall off of the fast path, we need take a closer look at our inputs.
If the input is indirect, or the \lstinline|(shift >= 64)| then we are
dealing with the bignum variant.  Otherwise, we can the double-precision
variant for a 128-bit (or less) result.

\begin{asm}
slowlsh:                                    # slowlsh: (rdi=x, rsi=shift)
	jheap	rdi, biglsh                 #   indirect nat? slowlsh
	cmp	rsi, 64                     #   if (shift >= 64)
	jae	biglsh                      #     goto biglsh
	mov	rcx, rsi                    #   count=shift
	xor	edx, edx                    #   hi=0
	mov	rax, rdi                    #   lo=x
	shl	rax, cl                     #   rax = new lo
	shld	rdx, rdi, cl                #   rdx = new hi
	jmp	mkdouble                    #   goto mkdouble(rdx,rax)
\end{asm}

\subsubsection{biglsh}

\begin{asm}
biglsh:                                     # biglsh:
	ppush	rsi, rdi                    #
	push	rbp                         #
	mov	rbp,rsp                     #
	add	rsp, -96                    #
	call	unpack                      #
	mov	rax, [rbp-48]               #   load shift
	test	rax, rax                    #   if (shift==0)
	jz	biglsh.zero                 #     goto zero
	mov	rax, [rbp-64]               #   load shift.bits
	cmp	rax, 63                     #   if (shift.bits > 63)
	ja	biglsh.huge                 #     goto huge
	mov	rdi, [rbp-24]               #
	add	rdi, [rbp-48]               #
	btw	rdi                         #
	mov	[rbp-88], rdi               #
	call	reserve                     #
	mov	[rbp-96], rax               #
	call	unpack                      #
	mov	rdi, [rbp-96]               #
	mov	rsi, [rbp-32]               #
	mov	rdx, [rbp-40]               #
	mov	rcx, [rbp-48]               #
	call	buflsh                      #
	mov	rdi, [rbp-88]               #
	call	claim                       #
	pdrop	2                           #
	leave                               #
	ret                                 #
biglsh.zero:                                #
	pdrop	2                           #
	mov	rax, [rbp - 8]              #
	leave                               #
	ret                                 #
biglsh.huge:                                #
	ud2                                 #
\end{asm}

\subsubsection{buflsh}

\begin{asm}
buflsh:
	push	rbx
	push	r12
	mov	r9, rcx
	shr	r9, 6
	and	rcx, 63
	test	rcx, rcx
	jz	buflsh.word_only
	test	r9, r9
	jz	buflsh.basic
	xor	r8, r8
buflsh.zero_loop:
	cmp	r8, r9
	jge	buflsh.basic
	mov	qword ptr [rdi], 0
	add	rdi, 8
	inc	r8
	jmp	buflsh.zero_loop
buflsh.basic:
	xor	r8, r8
	xor	r9, r9
	mov	rbx, 64
	sub	rbx, rcx
	mov	r12, rcx
buflsh.loop:
	cmp	r8, rdx
	jge	buflsh.finalize
	mov	r10, [rsi + r8*8]
	mov	r11, r10
	mov	rcx, r12
	shl	r11, rcx
	or	r11, r9
	mov	[rdi + r8*8], r11
	mov	r9, r10
	mov	rcx, rbx
	shr	r9, rcx
	inc	r8
	jmp	buflsh.loop
buflsh.finalize:
	cmp	r9, 0
	je	buflsh.return
	mov	[rdi + r8*8], r9
buflsh.return:
	pop	r12
	pop	rbx
	ret
buflsh.word_only:
	lea	rdi, [rdi + r9*8]
	mov	rcx, rdx
	rep	movsq
	pop	r12
	pop	rbx
	ret
\end{asm}

\subsection{Read a Bit}

The \lstinline|Test| operation reads the value of a specific bit in
a nat.

The basic implementation outline is fairly simple: if it is direct,
use \lstinline|bt|.  If it is indirect, fetch the appropriate words and
then use \lstinline|bt|.

However, also note that we never have to check if the index is indirect
or not, because indirect values will always be out of bounds, and so
will the actual register value.  Pretending that the index is always
direct always gives the right results.

Another thing to note is that we do not need the out-of-bound check for
the direct case.  Since the high bit is always zero, simply capping
the index to a maximum value of 63 always gives the right result.

The indirect case could jump to \lstinline|optest.direct|, but the
capping logic is unnecessary there, since the computed bit-index is
always in range.

\begin{asm}
optest:                                      # optest: rdi=i rsi=nat
	ENAT	rdi, rsi                     #   eval+cast
	ENAT	rsi, rdi                     #   eval+cast
	jheap	rsi, optest.indirect         #   if rsi.big goto indirect
optest.direct:                               # direct: rdi=i rsi=nat
	mov	rdx, 63                      #
	cmp	rdi, rdx                     #   i = min(i, 63)
	cmova	rdi, rdx                     #
optest.directknown:                          # directknown
	xor	eax, eax                     #   res=0
	bt	rsi, rdi                     #   check bit
	mov	rdx, 1                       #   1
	cmovc	rax, rdx                     #   if set, res=1
	xor	esi, esi                     #   wipe nat (in case invalid)
	ret                                  #   return res
optest.indirect:                             # indirect:
	bitsiz_	rcx, rsi                     #   rcx=bits
	cmp	rdi, rcx                     #   if (i >= bits)
	jae	ret0                         #     return 0
	mov	rdx, rdi                     #
	shr	rdx, 6                       #   wordix = bitix/64
	ptr	rsi                          #   nat = PTR(nat)
	mov	rsi, [rsi+rdx*8]             #   nat = nat[wordix]
	and	rdi, 63                      #   idx = idx % 64
	jmp	optest.directknown           #   goto directknown
\end{asm}

\subsection{Write a Bit}

The \lstinline|Set| operation writes a 1 to a specific bit of the
given number.

We rely on the fact that indirect bignums always have the high bit set in
order to have a fast path with a single branch.  This works by setting
the min(i, 63) bit.  If the high bit was already set (indirect case),
or the result will not fit in 63 bits (i >= 63), then the high bit of
the result will be set, triggering the slow path.  Otherwise, we can just
return this result.

If we leave the fast path, then the result is guaranteed to be indirect.

Note that doing a \lstinline|bts| on registers is much faster than
doing it on memory because the memory variant has atomicity guarantees
which we don't need, hence the explicit load and store.

\begin{asm}
opset:                                      # opset: rdi=i rsi=x
	ENAT	rdi, rsi                    #   eval+cast i
	ENAT	rsi, rdi                    #   eval+cast x
fastset:                                    #
	mov	r8, 63                      #
	mov	r9, rdi                     #
	cmp	r9, r8                      #
	cmova	r9, r8                      #   tmp = min(i, 63)
	mov	rax, rsi                    #
	bts	rax, r9                     #   res = bts(x, tmp)
	test	rax, rax                    #   if (high bit set)
	js	slowset                     #     enter then slow path
	ret                                 #   return result
slowset:                                    # slowset: rdi=i rsi=x
	xor	eax, eax                    #   clear rax to avoid bad gc state
	mov	r8, rdi                     #   r8/i (avoid clobber)
	mov	r9, rsi                     #   r9/x (avoid clobber)
	lea	rsi, [r8+64]
	shr	rsi, 6                      #   minsz = ((i+1)+63)/64
	mov	rdi, r9
	call	reservecopyflex             #   rax/ptr rdx/wid
	mov	rdi, r8
	shr	rdi, 6                      #   wix = i / 64
	movzx	rsi, r8b                    #   bix = i % 64
	mov	r12, [rax+8*rdi]            #   tmp = buf[wix]
	bts	r12, rsi                    #   tmp = bts(tmp, bix)
	mov	[rax+8*rdi], r12            #   buf[wix] = tmp
	xor	r12, r12                    #   clear tmp (could be invalid)
	mov	rdi, rdx
	jmp	claim                       #   return claim(wid)
\end{asm}

\subsection{Clear a Bit}

The \lstinline|Clear| operations writes a 0 to a specific bit of the given
number.

For the direct case, the high bit is always zero, so clearing that
has no effect.  Thus, we can achive the desired result by doing
\lstinline|btr(word, max(63, index))|.  This also works correctly if
the index is indirect.

The slow path can produce a direct number iff we're looking to set the 63rd bit
of a 1 word indirect number.

\begin{asm}
.global fastclear
opclear:                                    # opclear: rdi=i rsi=x
	ENAT	rdi, rsi                    #   eval+cast i
	ENAT	rsi, rdi                    #   eval+cast x
fastclear:                                  # fastclear:
	jheap	rsi, slowclear              #   If x is indirect, slowpath
	mov	rax, rsi                    #   result = x
	mov	rsi, 63                     #   rsi=63
	cmp	rdi, rsi                    #   if (i > 63)
	cmova	rdi, rsi                    #     i=63
	btr	rax, rdi                    #   clear bit
1:	ret                                 #   return result
slowclear:                                  # slowclear: rdi=i rsi=x
	mov	rax, rsi                    #   result = x
	bitsiz	rsi                         #   n = bitsiz(x)
	cmp	rdi, rsi                    #   if (i >= n) return
	jae	1b
	mov	r8, rdi                     #   r8/i (avoid clobber)
	mov	r9, rax                     #   r9/x (avoid clobber)
	mov	rdi, rax
	call	reservecopy                 #   rax/ptr rdx/wid
	mov	rdi, r8                     #
	shr	rdi, 6                      #   wix = i / 64
	movzx	rsi, r8b                    #   bix = i % 64
	mov	r12, [rax+8*rdi]            #   tmp = buf[wix]
	btr	r12, rsi                    #   tmp = btr(tmp, bix)
	mov	[rax+8*rdi], r12            #   buf[wix] = tmp
	xor	r12, r12                    #   clear tmp (could be invalid)
	mov	rdi, rdx                    #
	jmp	claim                       #   return claim(wid)
\end{asm}


\subsection{Read a Byte}

\begin{asm}
opload8:
	ENAT	rdi, rsi
	ENAT	rsi, rdi
fastload8:
	mov	rcx, rdi # TODO: reallocate registers
	mov	rax, rsi
	jheap	rax, fastload8.indirect
	cmp	rcx, 8
	jae	fastload8.zero
	shl	rcx, 3             # bits to drop
	shr	rax, rcx
	movzx	rax, al
	ret
fastload8.indirect:
	bitsiz_	r10, rax           # r10 = bitsize(rax/bignat)
	add	r10, 7
	shr	r10, 3             # bytesz
	cmp	rcx, r10           # if (i >= sz) return 0
	jae	fastload8.zero
	ptr	rax                # rax = natptr
	mov	al, [rax + rcx]    # rax = natptr[i]
	movzx	rax, al
	ret
fastload8.zero:
	xor	rax, rax
	ret
\end{asm}


\subsection{Write a Byte}

The \lstinline|Store8| operation sets a specific byte of a number.

\begin{verbatim}
    = (Wipe o w n)       | Sub n (Lsh (Cut o w n) o)
    = (Edit o w v n)     | Add (Wipe o w n) (Lsh (Trunc w v) o)
    = (Store o w v n)    | Edit mul8-o mul8-w v n
    = (Store8 o v n)     | Store o 1 v n
\end{verbatim}

Just do this in a simple way.

\begin{itemize}
\item Calculate the minimum size (min = 1+(i/8)).
\item Do a reservecopyflex.
\item Write the byte.
\item Claim the result.
\end{itemize}

This will correctly handle direct numbers, though not in the most
efficient way possible.

\begin{asm}
.global faststore8
opstore8:                                   # opstore8: rdi/i rsi/b rdx/n
	ENAT	rdi, rsi, rdx               #  eval+cast index
	ENAT	rsi, rdi, rdx               #  eval+cast byte
	ENAT	rdx, rsi, rdi               #  eval+cast nat
faststore8:                                 # faststore8:
	mov	r8, rdi                     #   r8/i
	mov	r9, rsi                     #   r9/b
	mov	r10, rdx                    #   r10/n
	mov	rdi, r10                    #   x = n
	mov	rsi, r8                     #
	shr	rsi, 3                      #
	inc	rsi                         #   y = (i/8)+1
	call	reservecopyflex             #   buf, wid = reservecopyflex(x,y)
	jdirect	r9, 1f                      #   if (indirect(r9))
	dref	r9                          #     r9 = r9.lsw()
1:	mov	[rax+r8], r9b               #   buf[i] = (u8) b
	xor	r9w, r9w                    #   clear r9
	mov	rdi, rdx                    #   size = wid
	jmp	claim                       #   return claim(wid)
\end{asm}

\subsection{Write a 64-Bit Word}

The \lstinline|Store64| operation sets a specific word of a number.

This will correctly handle direct numbers, though not in the most
efficient way possible.

In order to calculuate the minimum word-size of the result, we need to be
careful to correctly handle the case where the write is not word aligned.
Here is the formula with some examples:

\begin{verbatim}
    last_byte = index+7
    min_words = (last_byte/8)+1.
    formula   = ((index+7)/8)+1

    0 -> 1 = 1+((0+7)/8) ; at index 0, we need one word
    1 -> 2 = 1+((1+7)/8) ; at index 1, we need two
    8 -> 2 = 1+((7+8)/8) ; at index 8, still only two
    9 -> 2 = 1+((7+9)/8) ; at index 9, we need three
\end{verbatim}

\begin{asm}
.global faststore64
opstore64:                                  # opstore64: rdi/i rsi/w rdx/n
	ENAT	rdi, rsi, rdx               #   eval+cast index
	ENAT	rsi, rdi, rdx               #   eval+cast word
	ENAT	rdx, rsi, rdi               #   eval+cast nat
faststore64:                                # faststore64:
	mov	r8, rdi                     #   r8/i
	mov	r9, rsi                     #   r9/w
	mov	r10, rdx                    #   r10/n
	mov	rdi, r10                    #   x = n
	lea	rsi, [r8+7]                 #
	shr	rsi, 3                      #   ; y = min word size
	inc	rsi                         #   y = ((i+7)/8)+1
	call	reservecopyflex             #   buf, wid = reservecopyflex(x,y)
	jdirect	r9, 1f                      #   if (indirect(word))
	dref	r9                          #   word = word.lsw()
1:	mov	[rax+r8], r9                #   buf[i] = word
	xor	r9w, r9w                    #   wipe r9 register
	mov	rdi, rdx                    #   size = wid
	jmp	claim                       #   return claim(wid)
\end{asm}

\subsection{Write a 64-Bit Word In-Place}

TODO: Consider changing the semantics of store64Uniq so that it will
refuse to grow the result (only keeping bits which are in range).

\subsection{Read a Byte Range}

"Load is just reserve, memcpy, claim except that you need to cap the size
of the copy of it if oob."

Load reads a byte-range of a Nat into another Nat.

Some notes on the handling of edge cases:

\begin{itemize}
\item
   If the offset is indirect, then the actual register value will have
   the high bit set, which is a value that is much larger than the biggest
   possible Nat, producing a zero result.  Since this is the correct
   result for all possible inputs, we don't have to specifically handle
   this case.
\item
   Similarly, since we only read bytes which actually exist in the input
   nat, a massive read width is just fine.  Using the same trick as the
   last point, we can treat indirect words as indirect words, since the
   register value will always be large enough so that the read-size
   needs to be capped to what's actually available.
\end{itemize}

Some notes on edge-cases in the direct path:

\begin{itemize}
\item
   \lstinline|bzhi| preserves all bits if given a
   \lstinline|count >= 64|, which is the right behavior.  However,
   we still need to cap the width to a maximum of 8 bytes, otherwise
   the conversion to bits could overflow.
\item
    Large offsets must be explicitly handled for the same reason as well
    (or else the conversion from bits to bytes could overflow).

    \lstinline|shr| also has weird behavior if we used with a
    \lstinline|count >= 64|, but the above bounds check means that we
    don't have to worry about that, since the largest bit-offset we will
    be shifting is 56.
\end{itemize}

\begin{asm}
.global opload
opload:				# opload: rdi=offset, rsi=width, rdx=nat
	ENAT	rdi, rsi, rdx
	ENAT	rsi, rdi, rdx
	ENAT	rdx, rdi, rsi
	# fallthrough to fastload
\end{asm}

\begin{asm}
.global fastload
fastload:                       # fastload: rdi=off rsi=wid rdx=num
	jheap	rdx, fastload.indirect
fastload.direct:
	mov	rax, 8
	cmp	rdi, rax        # if (offset >= 8)
	jae	ret0            #   return 0 (avoids lsh overflow)
	cmp	rsi, rax
	cmova	rsi, rax        # width = min(width, 8)  (avoids lsh overflow)
	shl	rdi, 3		# offset bytes to bits
	shl	rsi, 3		# width bytes to bits
	mov	rax, rdx        # result = nat
	mov	rcx, rdi        #
	shr	rax, rcx        #   result >>= offset
	bzhi	rax, rax, rsi	#   extract first n bits from rax.
	ret                     #
fastload.indirect:
	## Calculate capped size.
	wordsz_	rax, rdx	# rax = wordsz(rdx)
	shl	rax, 3		# bytes = words*8
	sub	rax, rdi	# rax=size(rdx) after offset
	jbe	ret0            # if (offset >= bytes) return 0
	cmp	rax, rsi
	cmovb	rsi, rax	# rsi=min(sizeof(nat), offset)

	ppush	rdi, rsi, rdx
	mov	rdi, rsi	# rdi=width (in bytes)
	add	rdi, 7
	shr	rdi, 3		# rdi=width (in words)
	call	reserve
	ppop	rdi, rsi, rdx

	xchg	rdx, rsi        # rdx/width rsi/nat
	mov	rcx, rdx	# count = width
	ptr	rsi             # src = PTR(input nat)
	add	rsi, rdi	# src += offset
	mov	rdi, rax	# dst = rax from rawreserve
	rep	movsb           # memcopy

	mov	rdi, rdx        # rdi = bytesize
	add	rdi, 7
	shr	rdi, 3		# rdi=width (in words)
	jmp	claim
\end{asm}

\subsection{Fast Copy of a Byte-Range}

\lstinline|Snore| copies a range of an input nat into a range of an
output nat.

This is a load-bearing operation, because it is used in many places
for building up bigger numbers by combining many different small and
large numbers.  For example, in serialization, printing, and operations on
packed arrays.  Because of this, each of the four cases (direct/indirect *
input/output) have fairly well-optimized code paths.

\subsubsection{opsnore}

As always, the fully general entry-point needs to evaluate and cast all
of the inputs.

\begin{asm}
opsnore:                                    # opsnore:
	ENAT	rdi, rsi, rdx, rcx, r8      #   eval+cast rdi/iof
	ENAT	rsi, rdi, rdx, rcx, r8      #   eval+cast rsi/bof
	ENAT	rdx, rdi, rsi, rcx, r8      #   eval+cast rdx/wid
	ENAT	rcx, rdi, rsi, rdx, r8      #   eval+cast rcx/inp
	ENAT	r8, rcx, rdi, rsi, rdx      #   eval+cast r8/buf
	# fallthrough to fastsnore
\end{asm}

\subsubsection{fastsnore}

Here, we know that all of our inputs are evaluated numbers.

First we check if the output buffer is direct or indirect.  If it is
direct, we calculate the bar width of the output minus the output offset
and constrain the copy-width to be no larger than that.

And then we branch to handle the case of a copy from a small input to
a small output.

\begin{asm}
.global fastsnore
fastsnore:                                  # fastsnore:
	mov	rax, r8                     #   result = buffer
	jheap	r8, bufsnore                #   indirect buffer -> bufsnore
	dbytsz_	r10, r8                     #   r10 = buffer size
	dec	r10                         #   r10 = size of writable portion
	sub0	r10, rsi                    #   r10 = writable size after offset
	cap	rdx, r10                    #   rdx/wid = safe write-size
	jheap	rcx, funnelsnore            #   indirect input -> funnelsnore
	# fallthrough to smolsnore
\end{asm}

\subsubsection{smolsnore}

When we are copying from a small input into a small buffer we don't need
any branches or memory operations.  Out-of-bounds slices are handled
using a conditional move at the end.

First, we calculate the size of the input and constrain the copy-width
to always be in-bounds (ignoring trailing zeros).

Next, we use \lstinline|shrx & bzhi| to extract the relevant bits from
the input.

Then we use \lstinline|bzhi & shlx & andn| to zero out the destination
bits from the output.

And finally, we use \lstinline|shlx & or| in order to write the input
bits into the output.

If the copy-width was zero, some of these computations might have produced
non-sense, but that is harmless.  We simply ignore what the result of
this computation and return the result unchanged.

\begin{asm}
smolsnore:                                  # smolsnore:
	dbytsz_	r10, rcx                    #   r10 = input size
	sub0	r10, rdi                    #   r10 = readable size after offset
	cap	rdx, r10                    #   rdx = safe read-size
smolsnore.bounded:                          # bounded: (used by funnelsnore)
	shl	rdx, 3                      #   rdx = wid in bits
	shl	rdi, 3                      #   rdi = iof in bits
	shl	rsi, 3                      #   rsi = bof in bits
	shrx	rcx, rcx, rdi               #   shr to apply input offset
	bzhi	rcx, rcx, rdx               #   bzhi to extract relevant bits
	mov	r9, -1                      #   r9/mask = -1
	bzhi	r9, r9, rdx                 #   r9/mask = (1 << wid) - 1
	shlx	r9, r9, rsi                 #   r9/mask = mask << bof
	andn	r9, r9, rax                 #   tmp = buf & ~mask
	shlx	rcx, rcx, rsi               #   input <<= bof
	or	r9, rcx                     #   tmp |= (input << bof)
	test	rdx, rdx                    #   if (wid != 0)
	cmovnz	rax, r9                     #     res = tmp
	ret                                 #   return res
\end{asm}

\subsubsection{funnelsnore}

Copying from an indirect input into a direct buffer is the same as
copying from a direct input into a direct buffer except that we have to
load the relevant chunk memory first.

We need to be careful not to read past the end of the input, but it is
safe to read a few bytes before the beginning of the input (since there
is always a GC header there).

The first thing this code does is to cap the size to always be in bounds,
we have to do this even though the destination is smaller than the input
because the slice could be at the end of the input.

Then, we need to do an early return if the resulting width is zero,
because otherwise we might do an out-of-bounds read (given a huge input
offset, for example).

Then, get a pointer into the input data, advance it by the input offset,
and the slide it back so that when we do the read, the sliced bytes are
the high bytes of the result.

Then we shift this back, removing all of the irreevant bytes from
the input, and we set the input offset to zero.

At this point, we have all of the input data in the low bits of a
direct words.  And we know that everything is in-bounds for both input
and output.  Here, the logic is the same as the direct+direct case,
so we just jump into that code.

\begin{asm}
	# rdi/iof rsi/bof rdx/wid rcx/inp rax/r8/buf
	# buf is direct, inp is indirect
funnelsnore:
	bytsiz_	r9, rcx                     #   max  = inp.bytes
	sub0	r9, rdi                     #   max -= input-offset
	cap	rdx, r9                     #   wid = min(wid,max)
	test	rdx, rdx                    #   if (wid==0)
	jz	funnelsnore.ret             #     return buf
	refo	rcx, rdi                    #   p = PTR(inp)+iof
	mov	r9, 8                       #   tmp = 8
	sub	r9, rdx                     #   tmp = 8 - wid
	sub	rcx, r9                     #   p -= tmp
	mov	rcx, [rcx]                  #   inp = *p
	shl	r9, 3                       #   tmp *= 8 (bits)
	shrx	rcx, rcx, r9                #   inp >>= tmp (slide back)
	xor	edi, edi                    #   iof = 0
	jmp	smolsnore.bounded           #   goto smolsnore.bounded
funnelsnore.ret:
	ret
\end{asm}

\subsubsection{bufsnore}

bufsnore handles the cases where the output buffer is indirect.

If the input is direct, that is handled by \lstinline|pokesnore|,
otherwise this is just a memcopy after the width has been capped to be
in-range and the offsets have been applied to the input and the output.

We don't have to do anything special in the case of an out-of-bounds read
and write, because this will result in a zero copy-width which will mean
that the memcopy does nothing.

\begin{asm}
	# rdi/iof rsi/bof rdx/wid rcx/inp rax/r8/buf
	# buf is indirect
bufsnore:                                   # bufsnore:
	bytsiz_	r10, r8                     #   max  = buffer byte-size
	dec	r10                         #   max  = writeable size
	sub0	r10, rsi                    #   max -= buffer-offset
	cap	rdx, r10                    #   constrain wid
	jdirect	rcx, pokesnore              #   indirect input -> bigsnore
	bytsiz_	r10, rcx                    #   max  = inp.bytes
	sub0	r10, rdi                    #   max -= input-offset
	cap	rdx, r10                    #   constrain width to input size
	ptr	rcx                         #
	add	rcx, rdi                    #   rcx = PTR(inp)+iof
	ptr	r8                          #
	add	r8, rsi                     #   r8 = PTR(buf)+bof
	mov	rdi, r8                     #   dst = r8
	mov	rsi, rcx                    #   src = rcx
	mov	rcx, rdx                    #   cnt = wid
	rep	movsb                       #   memcpy
	ret                                 #   return buf
\end{asm}

\subsubsection{pokesnore}

When we are copying a slice of a direct atom into an indirect atom,
we can do this using a single 64-bit load followed by a single 64-bit
store, and we only have one predictable branch to handle the case where
the input slice is completely out-of-bounds.

We have to be careful not to read past the *end* of the nat, because it
is possible that it is at the end of a memory region.  However, every nat
has a 64-bit GC header, so it is always safe to read a few bytes *before*
the slice.

Once we constain the slice size so that it fits within the input and
output, a non-zero width guarentees that the read will be in-bounds,
so we can eliminate that case by returning immediatly if it is zero.

Next, we shift the pointer backwards, so that we read a word where all
of the bytes that we want to replaces are the high bytes of the word.

We then zero out these bytes with \lstinline|bzhi|, shift the input into
the appropriate bytes, and then combine the to with an \lstinline|or|,
and write the word back out to the same location that we loaded it from.

This is much simpler and faster than other approaches because we don't
need to branch on the size, and we don't need a loop.

\begin{asm}
	# rdi/iof rsi/bof rdx/wid rcx/inp rax/r8/buf
	# buf is indirect, inp is direct
pokesnore:                                  # pokesnore:
	dbytsz_	r9, rcx                     #   tmp  = inp.bytes
	sub0	r9, rdi                     #   tmp -= iof
	cap	rdx, r9                     #   wid = min(wid, tmp)
	test	rdx, rdx                    #   if (wid==0)
	jz	pokesnore.ret               #     return buf
	shl	rdi, 3                      #   iof*8
	shrx	rcx, rcx, rdi               #   inp >>= (iof*8)
	refo	r8, rsi                     #   p = PTR(buf)+bof
	mov	r9, 8                       #
	sub	r9, rdx                     #   tmp = [8-wid]
	sub	r8, r9                      #   p -= tmp
	mov	r10, [r8]                   #   wor = *p
	shl	r9, 3                       #   tmp <<= 3 (bit clif)
	bzhi	r10, r10, r9                #   clear high bits from wor
	shlx	rcx, rcx, r9                #   inp <<= tmp (slide position)
	or	r10, rcx                    #   wor |= inp
	mov	[r8], r10                   #   *p = wor
pokesnore.ret:                              # pokesnore.ret:
	ret                                 #   return buf
\end{asm}

\section{Closure Primops}

\subsection{Closure Size}

\begin{asm}
opsz:                                       # opsz: rdi=arg
	EVAL	rdi                         #   eval arg
fastsz:                                     # fastsz:
	xor	rax, rax                    #   res = 0
	jnclz	rdi, 1f                     #   if !isapp(arg) return
	clzsz_	rax, rdi                    #   res = arg.sz()
1:	ret                                 #   return
\end{asm}


\subsection{Closure Head}

\begin{asm}
ophd:                                       # ophd: rdi=arg
	mov	rax, rdi                    #   res = input
	ERAX                                #   eval res
	jnclzt	rax, 1f                     #   if (!isapp(res)) return
	dref	rax                         #   res = res.hd()
1:	ret                                 #   return
\end{asm}

\subsection{Last: Final Element}

Because of the way the formalism is defined, getting the last element
of a closure is a core operation.

Implementing this is easy.  Since all closures have at least on parameter,
the only edge-case is the cases where we are passed something besides
a closure.

\begin{asm}
oplast:                                     # oplast: (rdi=thunk)
	EVAL	rdi                         #   eval rdi
	jnclz	rdi, ret0                   #   if (!rdi.isclz) return 0
	slots_	rcx, rdi                    #   get slots (sz+1)
	ptr	rdi                         #   p = PTR(rdi)
	mov	rax, [rdi+8*rcx-8]          #   res = p[slots-1]
	JRAX                                #   return eval(res)
\end{asm}


\subsection{Init: Drop Final Element}

There are two cases.  If given a single-element closure, return the head.
Otherwise copy the closure to create a new one with one-fewer arguments.

\begin{asm}
opinit:                                     # opinit: (rdi=thunk)
	EVAL	rdi                         #   eval rdi
	jnclz	rdi, ret0                   #   if (!rdi.isclz) return 0
	dref_	rax, rdi                    #   hd = rdi.hd()
	clzsz_	rcx, rdi                    #   n = rdi.size
	cmp	rcx, 1                      #   if (n != 1)
	jne	opinit.copy                 #     goto copy
	ret                                 #   return hd
opinit.copy:                                # copy: rcx/n rdi/clz rax/hd
	ppush	rdi                         #   save oldclz
	mov	rdi, rax                    #   arg1 = hd
	lea	rsi, [rcx-1]                #   arg2 = n-1
	call	closure                     #   rax = newclz
	ppop	rsi                         #   src = restore(oldclz)
	ptr	rsi                         #   src = oldclz.ptr
	ptr_	rdi, rax                    #   dst = newclz.ptr
	mov	r8b, [rsi-8]                #   r8 = oldclz.heaprecord.type
	mov	[rdi-8], r8b                #   newclz.heaprecord.type = r8
	rep	movsq                       #   memcpy(newclz.ptr, oldclz.ptr, n)
	ret                                 #   return newclz
\end{asm}


\subsection{Row: Closure from List}

Given a head, a size, and a list, construct a row from the list.

\begin{verbatim}
    Row 3 4 [1 [2 [3 [4 0]]]]  ==>  3[1 2 3 4]
\end{verbatim}

This operation is designed to be as operationally simple as possible,
which is why the size is explicit and the head is cast to a nat.  Without
a known size, we need to \lstinline|realloc()|.  If the head might be
a function or closure we would need to deal with a bunch of edge-cases.
The operation behavior it thus:

\begin{verbatim}
    hed = NAT(hed)
    row = mkrow(hed, n)
    for (int i=0; i<n; i++):
        eval list
        row[i] = Ix0(list)
        list = Ix1(list)
    return row
\end{verbatim}

Note that Ix1 returns 0 for non-closures, so if the list is smaller
than the size, the remaining elements are zeros.  Since we have to
zero-initialize the memory in order to maintain GC invariants, we take
advantage of this to simply break out of the loop if given a list which
is too slow.

\begin{asm}
oprow:                                      # oprow:
	ENAT	rdi, rsi, rdx               #   eval+cast hd
	ENAT	rsi, rdi, rdx               #   eval+cast n
	mov	rax, rdi                    #   rax/hd
	test	rsi, rsi                    #   if (n == 0)
	jz	oprow.ret                   #     return hd
	call	closure                     #   row = closure(hd, sz)
	mov	r8, rax                     #   r8/row
	mov	r9, rdi                     #   r9/hd
	xor	eax, eax                    #   rax = 0
	ptr_	rdi, r8                     #   rdi = PTR(row)
	mov	[rdi], r9                   #   row.hd = hd
	add	rdi, 8                      #   rdi = &row.slot(0)
	mov	rcx, rsi                    #   count = n
	rep	stosq                       #   memset
	xor	ecx, ecx                    #   i = 0
	mov	rdi, rdx                    #   rdi/list
	mov	rdx, rsi                    #   rdx/n
oprow.loop:                                 # loop: r8=row rdx=n rcx=i rdi=list
	cmp	rcx, rdx                    #   if (i >= n)
	jae	oprow.break                 #     break
	EVAL	rdi, rdx, rcx, r8           #   evaluate list
	jnclz	rdi, oprow.break            #   if notapp(list) break;
	drefo_	rax, rdi, 8                 #   res = rdi.ix(0)
	ptr_	r9, r8                      #   r9      = PTR(row)
	mov	[r9 + rcx*8+8], rax         #   r9[i+i] = item
	slots_	rax, rdi                    #   get gcsz
	cmp	rax, 3                      #   if (gcsz < 3)
	jb	oprow.break                 #     break
	drefo	rdi, 16                     #   list = list.ix(1)
	inc	rcx                         #   i++
	jmp	oprow.loop                  #   continue
oprow.break:                                # break:
	mov	rax, r8                     #   res = buf
oprow.ret:                                  # ret:
	ret                                 #   return res
\end{asm}


\section{Effectful Primops}

\subsection{PeekOp}

TODO: there is no point in having an offset here, the caller can do the
add themselves.

\begin{asm}
PeekOp:                                     # rdi=[bufptr off sz]
	mov	rsi, 3                      #
	call	unpackop.sized              #
	ENAT	rdi, rsi, rdx               #   rdi = c dst pointer as nat
	ENAT	rsi, rdi, rdx               #   rsi = offset inside rdi
	ENAT	rdx, rdi, rsi               #   rdx = length in bytes to copy
	# fallthrough to fastpeekop
\end{asm}

\subsubsection{fastpeekop}

\begin{verbatim}
    ARG rdi = c src pointer as nat
    ARG rsi = offset inside rdi
    ARG rdx = length in bytes to copy
\end{verbatim}

\begin{asm}
.global fastpeekop
fastpeekop:                                 # fastpeekop: rdi=src rsi=off rdx=n
	add	rdi, rsi                    #   pointless offset logic
	lea	rcx, [rdx+8]                #
	shr	rcx, 3                      #   words = (bytes+8)/8
	ppush	rdi, rdx, rcx               #   save dest/bytes/words
	mov	rdi, rcx                    #
	call	reserve                     #   rax=buf=reserve(words)
	ppop	rsi, rcx, r8                #   rsi=ptr rcx=bytes r8=words
	mov	byte ptr [rax+rcx], 1       #   set high 1 byte
	mov	rdi, rax                    #   dst=buf
	rep	movsb                       #   copy bytes (src=ptr rcx=bytes)
	mov	rdi, r8                     #   rdi=words
	jmp	claim                       #   return claim(words)
\end{asm}

\subsection{PokeOp}

And here is the actual primop which unpacks, evaluates, and casts all
of the inputs.

TODO: the offset should be from the nat, instead of into the destination.
The caller can trivially add to the pointer before they call us, but they
can't easily get a pointer into the middle of a nat (we can).

\begin{asm}
PokeOp:                                     # PokeOp:
	mov	rsi, 4                      #   input = 0[ptr nat off count]
	call	unpackop.sized              #   load 4 operands
	ENAT	rdi, rsi, rdx, rcx          #   rdi = c dst pointer as nat
	ENAT	rsi, rdi, rdx, rcx          #   rsi = source nat
	ENAT	rdx, rdi, rsi, rcx          #   rdx = offset inside bufAddr
	ENAT	rcx, rdi, rsi, rdx          #   rcx = bytes to copy
	# fallthrough to fastpokeop
\end{asm}

\subsubsection{fastpokeop}

And here's the actual routine, minus the actualy copying logic.

\begin{asm}
.global fastpokeop
fastpokeop:                                 # fastpokeop:
	ppush	rbx, rbp                    #   save rbx, rbp
	add	rdi, rdx                    #   dst = bufAddr + offset
	mov	rbx, rdi                    #   rbx=ptr
	mov	rdi, rsi                    #   arg1=src
	call	fastbytsz                   #   rax=bytes (clobbers r8)
	mov	rbp, rcx                    #   rbp=extras
	sub0	rbp, rax                    #   extra=(n-sz) (floor to 0)
	sub	rcx, rbp                    #   count -= extras
	mov	rdi, rbx                    #   dst=ptr
	add	rbx, rcx                    #   rbx=tailptr = ptr+count
	call	pokeraw                     #   pokeraw()
	mov	rdi, rbx                    #   dst=tailptr
	mov	rcx, rbp                    #   cnt=extras
	xor	eax, eax                    #   val=0
	rep	stosb                       #   memset(dst,cnt,0)
	ppop	rbx, rbp                    #   restore rbx/rbp
	ret                                 #   return 0
\end{asm}

\subsubsection{pokeraw}

And here's a simple subroutine which just copies n bytes from a macro
into a destination address.

\begin{verbatim}
    ARG rdi=ptr (clobbered)
    ARG rsi=nat (clobbered)
    ARG rcx=count (clobbered)
\end{verbatim}

\begin{asm}
pokeraw:                                    # pokeraw:
	jdirect	rsi, pokeraw.direct         #   if direct, then goto direct()
pokeraw.indirect:                           # indirect:
	ptr	rsi                         #   src=ptr(nat)
	rep	movsb                       #   memcopy(dst=ptr,str,n=rcx)
	ret                                 #   return
pokeraw.direct:                             # direct:
	ppush	rsi                         #   write to memory
	mov	rsi, r15                    #   src = sp (addr of word)
	rep	movsb                       #   movsb(dst=ptr,src,n=rcx)
	ppop	rsi                         #   remove from stack
	ret                                 #   return
\end{asm}

\section{Heaps and Process Layout}

At the highest level, the PLAN runtime mmaps two regions:

\begin{enumerate}
\item
A heap which is managed using a buddy allocator. The per-thread heaps, other
memory allocated by the interpreter and all of the unpersisted pins.

\item
In binaries created by \lstinline|boot|, a file backed persistence heap which
is stored in the program binary itself, after the ELF region of the executable
and which is mmapped on startup.
\end{enumerate}

This memory is garbage collected in different ways:

\begin{enumerate}
\item
Each per-thread heap is a first generation, and is collected with a
thread-local moving GC using Cheney's algorithm. A new per-thread heap is
allocated from the buddy heap, live values are copied to it, and the old heap
is explicitly freed from the buddy heap.

\item
The buddy allocator heap is also garbage collected. On a collection of this
second generation, each thread submits a list of roots to the buddy allocator,
including interpreter objects like the heap and any pin objects referenced by
the stack or the first generation heap.

\item
The persistence file is also garbage collected. On a collection of this third
generation, live regions are calculated, and then we request the kernel to
punch a hole on pages between the live regions.

Right now, this collection only happens offline, but we hope to make this
online soon.
\end{enumerate}

\section{The Per Thread Heap}

Each thread has its own heap, where the heap itself is allocated from the main
buddy allocation arena.

\subsection{Cheney Sizes}

The Cheney heap can be in only one of a set of prechosen heap sizes. We
maintain a mapping from ``heap rank'' to sizes. The first 20 entries are the
Fibonacci sequence starting with 12 and 38 starting at fib number 7 in that
sequence, representing the start of the ramp up phase of the Cheney heap. After
the first 20 entries, we switch to 20\% growth per term up to a heap of ~583
megabytes, at which you should just crash. We allocate space for this table:

\begin{asmdata}
.set HEAP_MAX_RANK,	51

.global heap_sizes
.align 8
heap_sizes:
	.zero HEAP_MAX_RANK * 4
\end{asmdata}

\subsubsection{build\_heap\_table}

When we startup, we must build a table of the appropriate heap bucket
sizes. These are the sizes that we allocate during Cheney collection.

We follow the precedent of the Erlang BEAM VM by sizing our heap allocations by
a modified Fibonacci sequence, starting in the same place: 12 and 38
words. Also following BEAM, we switch from doing Fibonacci to doing a straight
multiplication when Fibonacci grows too quickly.

\begin{asm}
.set HEAP_MAX_RANK,	51
build_heap_table:			    # build_heap_table:
	mov	DWORD PTR [heap_sizes], 12  #   hs[0]=12, hs[1]=38
	mov	DWORD PTR [heap_sizes + 4], 38
	mov	r8, 2			    #   i = 2
build_heap_table.fibloop:		    # fibloop:
	mov	esi, [heap_sizes + r8*4-4]  #   esi = heap_sizes[i - 1]
	mov	edi, [heap_sizes + r8*4-8]  #   edi = heap_sizes[i - 2]
	add	esi, edi		    #   esi += edi
	mov	[heap_sizes + r8*4], esi    #   hs[i] = hs[i - 1] + hs[i - 2]
	inc	r8			    #   i++
	cmp	r8, 21			    #   if (i < 21)
	js	build_heap_table.fibloop    #   then keep doing fibonacci
build_heap_table.mulloop:		    # mulloop:
	mov	esi, [heap_sizes + r8*4-4]  #   esi = heap_sizes[i - 1]
	lea	eax, [rsi + rsi*2]	    #   eax = rsi * 3
	add	eax, eax		    #   eax = eax * 2 (or: rsi * 6)
	xor	edx, edx		    #   clear dividend
	mov	ecx, 5			    #   div by 5
	div	ecx			    #   eax = (6 * rsi) / 5 = 1.2 * rsi
	mov	[heap_sizes + r8*4], eax    #   heap_sizes[i] = 1.2 * hs[i - 1]
	inc	r8			    #   i++
	cmp	r8, HEAP_MAX_RANK	    #   if (i < HEAP_MAX_RANK)
	js	build_heap_table.mulloop    #   then loop
	ret				    #   return
\end{asm}

\subsection{Cheney State}

Each execution thread (currently only one) maintains information about its own
heap.

\begin{asmdata}
.global heap_addr, mapd_size, heap_size, heap_rank
heap_addr:	.quad 0
mapd_size:	.quad 0
heap_size:	.quad 0
heap_rank:	.quad 0
\end{asmdata}

\subsection{Cheney Collection}

Each thread has its own Cheney heap.

\subsection{asmgc}

This is the core loop that drives garbage collection. It sets up the current
Cheney heap in rbx/rbp, calculates out how large the new heap has to be to
accommodate the failed needed allocation, allocates the new heap, drives the
copying, and then calculates the target size to run the next collection.

Most of the work is in that final part:

\begin{itemize}
\item
If the sum of the surviving words and the needed words is less than 25\% of
capacity, then shrink the target heap rank by one.
\item
If the sum of the surviving words and the needed words are over 50\% of
capacity of the current rank, then grow the target heap rank by at least one,
increasing up until the sum fits, since the program could ask for much more
memory than the current size.
\item
Otherwise, we're sized correctly between 25\% and 50\% of capacity and we keep
the same target.
\end{itemize}

\begin{asm}
asmgc:					    # asmgc(rdi=needed):
	mov	rbx, [heap_addr]	    #   fromspc.top = heap_addr
	mov	rbp, [mapd_size]	    #   old heap mapped size in words
	shl	rbp, 3			    #   convert to bytes
	add	rbp, [heap_addr]	    #   fromspc.end = start+bytes
	mov	esi, [heap_rank]	    #   get current rank
	mov	r8d, [heap_sizes + esi*4]   #   get current size
	add	r8d, edi		    #   requested = current + needed
asmgc.incrank:				    # incrank:
	inc	esi			    #   increase rank
	mov	eax, [heap_sizes + esi*4]   #   get new rank size
	cmp	eax, r8d		    #   if (new rank size < requested)
	jb	asmgc.incrank		    #	then increase rank again
	mov	[mapd_size], rax	    #   mapd_size = new_heap_size
	mov	[heap_size], rax	    #   heap_size = mapd_size
	push	rdi			    #   preserve incoming needed
	mov	rdi, rax		    #   rdi = large enough size
	call	heap_new		    #   heap_new(size to hold requested)
	mov	[heap_addr], rax	    #   heap_addr = heap_new(...)
	mov	r14, rax		    #   heap_next = heap_addr
	mov	rax, [heap_size]	    #   rax = heap_size
	lea	r13, [r14+rax*8]	    #   heap_end = heap_addr[heap_size]
	call	copystack		    #   copystack()
	call	copyheap		    #   copyheap()
	call	free_old_heap		    #   free_old_heap()
	pop	rdi			    #   restore incoming needed
asmgc.calcnextrank:			    # calculate next rank size:
	mov	r12, [heap_addr]	    #   r12 = heap_addr
	mov	rax, r14                    #   used_bytes = heap_next
	sub	rax, r12                    #   used_bytes = heap_next-heap_addr
	shr	rax, 3			    #   bytes to words
	add	rax, rdi		    #   bytes += needed
	mov	r9d, [heap_rank]	    #   get original rank
	mov	r8d, [heap_sizes + r9d*4]   #   get original size
	mov	rdi, rax		    #   rdi = used size
	shl	rdi, 2			    #   rdi = used size * 4
	cmp	rdi, r8			    #   if (used < 0.25*orig)
	jb	asmgc.shrink		    #   then shrink a rank
	mov	rcx, r8			    #   rcx = original size
	lea     rcx, [rcx + rcx*1]	    #   rcx = original size * 2
	cmp	rdi, rcx		    #   if (use >= 0.50*orig)
	jae     asmgc.grow		    #  	then grow ranks to fit
asmgc.keeprank:				    # keep rank:
	mov	[heap_size], r8d	    #   heap size is current rank size
	jmp	asmgc.done		    #   completed size setting
asmgc.shrink:				    # shrink rank:
	## TODO: Understand where Erlang BEAM sets "bigness" and how
	## to translate that to our world. 6 is a shot in the dark.
	cmp	r9d, 6			    #   if (rank < 6) (minimum)
	jb	asmgc.keeprank		    #   then keep current rank
	dec	r9d			    #   decrease rank
	mov	qword ptr [heap_rank], r9   #   set the decreased rank.
	mov	r8d, [heap_sizes+r9d*4]	    #   get shrunk rank size
	mov	qword ptr [heap_size], r8   #   set shrunk size
	jmp	asmgc.done		    #   goto done
asmgc.grow:				    # grow rank:
	inc	r9d			    #   rank++
	mov	r8d, [heap_sizes+r9d*4]	    #   get increased rank size
	cmp	rax, r8			    #   if (used size > rank++ size)
	ja	asmgc.grow		    #   then increase again
	mov	qword ptr [heap_rank], r9   #   set the increased rank.
	mov	qword ptr [heap_size], r8   #   set grown size.
asmgc.done:				    # done:
	mov	r13, [heap_addr]	    #   heap_end = heap_addr
	lea	r13, [r13+8*r8]		    #   heap_end = &heap_addr[heap_size]
	ret				    #   return
asmgc.toobig:				    # too big:
	ud2				    #   cannot fulfill this request
\end{asm}







\section{The Buddy Allocator}

The runtime's main allocator is a buddy allocator. We choose the buddy
allocator because it's well understood and simple to implement. It is also
relatively easy to extend with garbage collection.

In a buddy allocator, the memory arena is represented as a binary tree where a
leaf node represents an allocation. At start, there's a single leaf node that
represents the entire arena. The high level idea is that if you ask for a small
4k allocation, this one large node will split to an inner node with two
unallocated leaf nodes, and will do this recursively until it has tightly sized
unallocated node which it hands to the user.

\subsection{Buddy Configuration}

Our allocator sizes are configurable, and there are two main tunables:

\begin{asmdata}
.global	min_alloc_log2, max_alloc_log2
min_alloc_log2: .quad 0
max_alloc_log2: .quad 0
\end{asmdata}

The minimum and maximum allocation size are given in terms of logs of two, such
that \lstinline|9| is \lstinline|512|, and \lstinline|12|
is \lstinline|4096|. We input logs of two because a buddy tree must have a
power of two nodes. Please note that the time to run one garbage collection
pass increases as \lstinline|max_alloc_log2 - min_alloc_log2| increases.

These two tunables are used to calculate the following variables:

\begin{asmdata}
.global min_alloc, max_alloc, bucket_count
min_alloc: .quad 0
max_alloc: .quad 0
bucket_count: .quad 0
\end{asmdata}

Where \lstinline|min_alloc| and \lstinline|max_alloc| are the limits in bytes,
and \lstinline|bucket_count| is the maximum depth of the tree between the two
log2 representations.

Using all this information, we allocate one mmap for one giant arena that
starts with space for state tracking, and has the raw arena managed by the
buddy allocator right after it.

\begin{asmdata}
.global buddy_mmap_ptr, base_ptr, base_end_ptr
buddy_mmap_ptr: .quad 0
base_ptr: .quad 0
base_end_ptr: .quad 0
\end{asmdata}

We'll document the different other structures as we go along.  There are two
main structures in the buddy allocator: the buckets and the nodes.

\subsection{Buddy Bucket Structure}

The bucket structure is an array of intrusive doubly linked lists which are
quick freelists for each allocation size. This array
is \lstinline|bucket_count| long and the largest bucket index represents the
smallest allocation.

\begin{asmdata}
.global buckets_ptr
buckets_ptr: .quad 0
\end{asmdata}

The ``bucket number'' of an allocation is the depth of the buddy tree where
this allocation would go, and is used as a general identifier of size, even
outside the bucket list.

\subsubsection{bucket\_for\_request}

Given a requested allocation size, round it up to the bucket that can hold that
allocation.

TODO: This can actually be done without a loop with bsr.

\begin{asm}
.global bucket_for_request
bucket_for_request:
	shl	rdi, 3		# convert to bytes
	mov	rax, [bucket_count]
	dec	rax
	mov	rsi, [min_alloc]
bucket_for_request.loop:
	cmp	rsi, rdi	#
	jnb	bucket_for_request.done
	dec	rax
	shl	rsi, 1
	jmp	bucket_for_request.loop
bucket_for_request.done:
	ret
\end{asm}

\subsection{GC Headers}

TODO: Figure out a better place for this to live.

Every node has a GC header. The tag bit on that header can be:


\begin{verbatim}
0b0000 - 0x0 - Nat
0b0001 - 0x1 - Pin
0b0010 - 0x2 - Law
0b0011 - 0x3 - Clz (nf)
0b0100 - 0x4 - Clz (whnf)
0b0101 - 0x5 - Thunk
0b0110 - 0x6 - Megapin

0b0111 - 0x7 - Heap
0b1000 - 0x8 - Hash Table
0b1001 - 0x9 - Bump Slab
\end{verbatim}





\subsection{Buddy Node Structure}

The implementation of the buddy allocator used in the Linux kernel uses an
optimized representation where each node is represented by a single bit, but we
cannot use this representation because it is not traversable. For garbage
collection purposes, we need to be able to traverse the buddy tree. We instead
must represent three states:

\begin{asm}
.set	UNALLOCATED,	0
.set	INNER_NODE,	1
.set	ALLOC_GC,	2
.set	ALLOC_MANUAL,	3
\end{asm}

This means that we need two bits to represent each node in the tree. We use a
linearized binary tree, packing four items into each byte. We store this tree
in the mmaped arena at \lstinline|node_state_ptr|:

\begin{asmdata}
.global node_state_ptr
node_state_ptr: .quad 0
\end{asmdata}

There are two ways to refer to a node:

\begin{itemize}
\item
    Directly by pointer to the first byte of a memory range.
\item
    By raw index into the linearized node tree.
\end{itemize}

You can switch between these two representations as long as you have the
``bucket number'', the index which represents the depth of this allocation in
the tree.

\subsubsection{nodeget}

Since nodes in our tree can be in one of three states, our lookup function must
parse out those two bits from the byte.

\begin{asm}
.global nodeget
nodeget:			            # nodeget(rdi=index):
	mov	ecx, edi                    #   copy index for shift
	and	ecx, 0x03                   #   ecx = index % 4
	shl	ecx, 1                      #   shift = (index % 4) * 2
	shr	rdi, 2                      #   byteIndex = index / 4
	mov	rax, [node_state_ptr]       #   load current base pointer
	movzx	eax, byte ptr [rax + rdi]   #   load byte
	shr	eax, cl                     #   offset byte
	and	eax, 0x03                   #   mask for answer
	ret                                 #   return
\end{asm}

\subsubsection{nodeset}

The core primitives of our buddy tree must be thread safe, so we write them
using atomics. Individual values in the tree are owned by different threads,
but the packing means that setting one of the four values in a byte needs to
not stomp other thread's work, so we use a \lstinline|cmpxchg| loop to ensure
we don't overwrite the value a different thread set.

\begin{asm}
.global nodeset
nodeset:                                    # nodeset(rdi=index, rsi=tristate):
	mov	ecx, edi                    #   copy index for shift
	and	ecx, 0x03                   #   ecx = index % 4
	shl	ecx, 1                      #   shift = (index % 4) * 2
	shr	edi, 2                      #   byteIndex = index / 4
	mov	r9d, 0x03                   #   r9d = 2 bit mask
	shl	r9d, cl                     #   shift mask into place
	and	esi, 0x03                   #   constrain to tristate
	shl	esi, cl                     #   shift state into place
	mov	rcx, [node_state_ptr]       #   load current base pointer
nodeset.try:                                # try:
	mov	al, [rcx + rdi]             #   load current full byte
	mov	r8b, al                     #   working copy
	and	r8b, r9b                    #   r8b = original & mask
	xor	r8b, al                     #   r8b = original with cleared bits
	or	r8b, sil                    #   r8b = new value inserted
	lock cmpxchg [rcx + rdi], r8b       #   CAS xchange al/[rcx+rdi]/r8b
	jnz	nodeset.try                 #   retry if contended
	ret                                 #   return
\end{asm}

\subsubsection{ptr\_for\_node}

Given a node index and a bucket number, return the pointer to the actual node.

\begin{asm}
.global ptr_for_node
ptr_for_node:			# ptr_for_node: (rdi=index, rsi=bucket)
	mov	rax, 1		#   base one
	mov	rcx, rsi	#   shl requires rcx
	shl	rax, rcx	#   rax = (1 << bucket)
	inc	rdi  		#   + 1
	sub	rdi, rax	#   index = index - (1 << bucket) + 1

	mov	rcx, [max_alloc_log2]
	sub	rcx, rsi
	shl	rdi, rcx	# offset = index << (MAX_ALLOC_LOG2 - bucket)

	mov	rax, base_ptr
	add	rax, rdi	# rax = base_ptr + offset
	ret
\end{asm}

\subsubsection{node\_for\_ptr}

Given a pointer and a bucket number, calculate the node index.

\begin{asm}
.global node_for_ptr
node_for_ptr:			# node_for_ptr: (rdi=ptr, rsi=bucket)
	mov	rax, rdi
	sub	rax, base_ptr	# rax = ptr - base_ptr

	mov	rcx, [max_alloc_log2]
	sub	rcx, rsi
	shr	rax, rcx	# rax =>> MAX_ALLOC_LOG2 - bucket

	mov	rdx, 1
	mov	rcx, rsi
	shl	rdx, rcx	# rdx = 1 << bucket

	add	rax, rdx

	dec	rax
	ret
\end{asm}

\subsection{List Utilities}

Buddy tree free lists are represented inline, where the memory for the free
space linked list node lives right at where the allocation will be in the
future. The free list is a doubly linked list for fast insertion and removal.

Each list starts with the following struct:

\begin{verbatim}
typedef struct list_t {
  struct list_t* prev;
  struct list_t* next;
} list_t;
\end{verbatim}

\subsubsection{list\_init}

The empty list points to itself.

\begin{asm}
.global list_init
list_init:				    # list_init(rdi=list*):
	mov	[rdi], rdi		    #   list->prev = list;
	mov	[rdi+8], rdi		    #   list->next = list;
	ret				    #   return
\end{asm}

\subsubsection{list\_push}

\begin{asm}
.global list_push
list_push:				    # list_push(rdi=list*,rsi=entry):
	ablptr	rdi                         #   input list pointer valid
	ablptr	rsi                         #   input new entry pointer valid
	mov	r8, [rdi]		    #   list_t* prev = list->prev;
	ablptr	r8                          #   input prev pointer valid
	mov	[rsi], r8		    #   entry->prev = prev;
	mov	[rsi + 8], rdi		    #   entry->next = list;
	mov	[r8 + 8], rsi		    #   prev->next = entry;
	mov	[rdi], rsi		    #   list->prev = entry;
	ret				    #   return
\end{asm}

\subsubsection{list\_remove}

Unlinks a list entry from its list.

\begin{asm}
.global list_remove
list_remove:				    # list_remove(rdi=entry*):
	ablptr	rdi                         #   input entry pointer valid
	mov	r8, [rdi]		    #   list_t* prev = entry->prev;
	ablptr	r8                          #   prev pointer valid
	mov	r9, [rdi + 8]		    #   list_t* next = entry->next;
	ablptr	r9                          #   next pointer valid
	mov	[r8 + 8], r9		    #   prev->next = next;
	mov	[r9], r8		    #   next->prev = prev;
	ret				    #   return
\end{asm}

\subsubsection{list\_pop}

\begin{asm}
.global list_pop
list_pop:				    # list_pop(rdi=list*) -> rax:
	mov	r8, [rdi]		    #   list_t* back = list->prev;
	cmp	r8, rdi			    #   if back == list
	je	ret0			    #   then return null
	ablptr	r8                          #   assert pointer is in buddy heap
	mov	rdi, r8			    #   remove back from list
	call	list_remove		    #   list_remove(back)
	mov	rax, rdi		    #   set back as return value
	ret				    #   return
\end{asm}

\subsection{Aligned Bit Trees}

We perform liveliness marking against an aligned bit tree, which mirrors the
structure of the buddy tree.

We have the logical structure where bit[0] is the head of a binary tree and its
left child is \lstinline|2*i + 1| and its right child is \lstinline|2*i + 2|.
The naive problem with this structure is that it is never word
aligned. Meanwhile, we want to use this structure to be quadword aligned as
much as possible so that we can do optimized walks over the data.

The compromise is that we just add one. Logical 0 is physical 1, logical 1 is
physical 2, etc. This means all of level 0-5 sits in the first quadword, level
6 sits exactly in the second quadword, level 7 sits in the third and fourth
quadword and so forth. All we have to do to achieve alignment is add a single
bit.

\subsubsection{aligned\_bit\_set}

Atomically sets a specific bit, returning the previous value.

\begin{asm}
.global aligned_bit_set
aligned_bit_set:			    # aligned_bit_set(rdi=bit,rsi=v*):
	inc	rdi			    #   logical bit -> physical bit
	mov	rax, rdi		    #   rax = copy logical bit
	shr	rax, 6			    #   rax = byte_index
	and	rdi, 63			    #   rdi = bit_offset
	lock bts QWORD PTR [rsi+rax*8], rdi #   atomically set bit
	setc	al			    #   original value in CF to al
	movzx	eax, al			    #   al to return value
	ret				    #   return
\end{asm}

\subsubsection{aligned\_bit\_get}

\begin{asm}
.global aligned_bit_get
aligned_bit_get:			    # aligned_bit_get(rdi=bit,rsi=v*):
	inc	rdi			    #   logical bit -> physical bit
	mov	rax, rdi		    #   rax = copy logical bit
	shr	rax, 6			    #   rax = byte_index
	and	rdi, 63			    #   rdi = bit_offset
	bt	QWORD PTR [rsi+rax*8], rdi  #   test bit in byte
	setc	al			    #   original value in CF to al
	movzx	eax, al			    #   al to return value
	ret				    #   return
\end{asm}


\subsection{Buddy Collection}

While our buddy allocator provides explicit \lstinline|buddy_malloc| and
\lstinline|buddy_free| calls for the cases where a process can control the
lifecycle of data, since user pins are placed on the heap, the entire buddy
allocator must provide a garbage collector. Our collection is concurrent: the
collector runs on its own thread, and 

\subsubsection{Data Structures}

When the concurrent garbage collector wants to collect, it pauses for every
thread to mark its roots in the explore tree. Individual threads use
\lstinline|shallow_mark| to mark the bottom of this tree, and when everything
is submitted, the sweep thread then uses \lstinline|project_explore_tree| to
make a binary bit tree used for exploration.

\begin{asmdata}
.global explore_tree_ptr
explore_tree_ptr: .quad 0
\end{asmdata}

Because we don't want to perform a linear scan that of the entire explore tree,
which will cause the kernel to back those pages, we instead keep track of the
highest index in the final bucket layer so we can bound scanning.

\begin{asmdata}
.global explore_max_idx
explore_max_idx: .quad 0
\end{asmdata}

Once the \lstinline|explore_tree| is filled, it "explores", meaning it looks at
completed explore structure to find live items and then recurse into its
dependencies. The sweep tree should have every live allocation reachable from
the items marked in the \lstinline|explore_tree|.

\begin{asmdata}
.global sweep_tree_ptr
sweep_tree_ptr: .quad 0
\end{asmdata}

But we need a second area to mark freshly malloced items that were
generated while the exploration and sweeping were in progress. We don't
want to overload the \lstinline|sweep_tree_ptr| because you then confuse marking
new items with whether you've recursively iterated through them, leading
to undermarking.

\begin{asmdata}
.global new_allocation_ptr
new_allocation_ptr: .quad 0
\end{asmdata}

These three bitmaps take a lot of space, and we don't want to have to zero them
out, paging them into memory. They are very sparse, so we keep record the size
of the bitmap in bytes so we can pass it to the kernel for clearing.

\begin{asmdata}
.global bitmap_bytes
bitmap_bytes: .quad 0
\end{asmdata}

Finally, once we've filled the sweep tree, we perform a sweep where we move
step by step through the tree. The actual state of the sweeper is visible as a
global variable because sometimes the \lstinline|buddy_free| process must
interact and change the sweeping state if the two are accessing the same
location. \lstinline|sweep_level| will be -1 when not running.

\begin{asmdata}
.global sweep_level
sweep_level: .quad -1

.global sweep_node_ptr
sweep_node_ptr: .quad 0

.global sweep_action_ptr
sweep_action_ptr: .quad 0
\end{asmdata}

\subsubsection{Locks}

Garbage collection of the buddy heap is concurrent, so we need some
locks.

First is the \lstinline|collector_lock|. A reader-biased reader/writer lock
where malloc is a reader and the collector is a writer. This is the main lock
that has to be taken whenever reading or writing to the \lstinline|mark_state|.

The design is that the collector should only progress through the tree while
there are no mallocs, because a malloc could be writing to the
\lstinline|mark_state| while calculating dependencies, which it is responsible
for doing if the collector thread is sweeping.

\begin{asmdata}
.global collector_lock
collector_lock:
	.zero	8*7
\end{asmdata}

\lstinline|bucket_locks_ptr| will be allocated when as part of initialization
to the right size of one u32 mutex for every bucket. The bucket locks are a
per-bucket level lock needed to be taken before a thread reads or writes to the
\lstinline|buckets| or the \lstinline|node_state| for that level.

When acquiring multiple locks, they should be acquired from the smallest
allocation size (the highest bucket number) to the largest allocation size
in the case of splitting.

\begin{asmdata}
.global bucket_locks_ptr
bucket_locks_ptr: .quad 0
\end{asmdata}

The locking strategy is that you either take a write lock on
\lstinline|collector_lock|, or take a read lock on \lstinline|collector_lock|
and take a fine grained bucket lock. The sweeping process will take a coarse
write \lstinline|collector_lock|, while malloc will take a read
\lstinline|collector_lock| and will take fine grained locks for the buckets it
needs to minimize contention inside malloc.

\subsubsection{atomic\_max}

Given multiple threads recording indexes at the same time, we need a way to
locklessly update a maximum number.

\begin{asm}
.global atomic_max
atomic_max:                                 # atomic_max(rdi=*, rsi=new_value):
	mov	rax, [rdi]                  #   load current value
atomic_max.retry:                           # retry:
	cmp	rsi, rax                    #   if new_value <= current
	jle	atomic_max.done             #   then don't change
	lock cmpxchg [rdi], rsi             #   if [rdi]==rax then [rdi]=rsi
	jne	atomic_max.retry            #   if ZF=0, exchange failed. retry.
atomic_max.done:                            # done:
	ret                                 #   return
\end{asm}

\subsubsection{project\_or}

Given a 64-bit word of child mark bits (packed as [child0, child1,
child2, ... child63], where the children for parent i are at positions
\lstinline|2*i| and \lstinline|2*i+1|), compute for each parent: parent's
bit = \lstinline|child_left| OR \lstinline|child_right|.

This is done by:

\begin{itemize}
\item Shifting a copy of the 64-bit word
\item ORing it with the original 64-bit word.
\item Using PEXT to extract even-indexed bits (mask 0x5555...5555)
\end{itemize}

\lstinline|project_or| expects its output to be pre-zeroed. It takes advantage
of this fact by skipping running any bit manipulation code on any zero input
since it'll result in a zero output.

\begin{asm}
.equ ODD_MASK, 0x5555555555555555
.global project_or
project_or:				    # project_or(rdi=dst,rsi=src,rdx=l):
	push	r15			    #   save r15
	mov	r15, ODD_MASK		    #   extraction mask
	xor	rcx, rcx		    #   i = 0
project_or.loop:			    # loop:
	mov	rax, [rsi+8*rcx]	    #   load the 64-bit mark word
	test	rax, rax                    #   if mark word is zero
	jz	project_or.next             #   then skip the expensive pext
	mov	r8, rax			    #   copy the word
	shr	r8, 1			    #   shift right by 1 bit
	or	rax, r8			    #   combine pairs
	pext	r9, rax, r15		    #   extract every other bit
	mov	[rdi+4*rcx], r9d	    #   store the result
project_or.next:                            # next:
	inc	rcx			    #   i++
	cmp	rcx, rdx		    #   if equals length
	jne	project_or.loop 	    #   then loop
	pop	r15			    #   restore r15
	ret				    #   return
\end{asm}


\subsection{Buddy Sweeping}

Sweeping is the second part of mark/sweep process for the buddy heap.


\subsubsection{recursively\_prune}

Recursively walk down the tree starting from bucket \lstinline|i| in
\lstinline|bucket| level, resetting the entire substructure. We handle the
three states as such:

\begin{itemize}
\item \lstinline|UNALLOCATED|: We just remove the item from the bucket list.
\item \lstinline|INNER_NODE|: We just recurse both ways.
\item \lstinline|ALLOC_GC|: This can be a no op. During pruning, we just abandon
  already allocated spans because they don't have any presence in the bucket
  list. But we may do various memory checks here 
\item \lstinline|ALLOC_MANUAL|: It is always an error to encounter a manually
  allocated item while recursively pruning.
\end{itemize}

At the end of this, {i, bucket} and every child node of it will be UNALLOCATED
in the \lstinline|node_state| and will not appear in any freelists for
allocation. (The caller is responsible for doing something with {i, bucket}.)

TODO: The assembly version of this has been temporarily removed and reverted to
the C \lstinline|recursively_prune_c| while we rework the collector to be
concurrent, and add more checking code to the process.


\section{The Persistence Heap}

The runtime may provide a durable persistence heap which is written inside the
current binary. This presents two XPLAN operations to the
programmer: \lstinline|precommit| and \lstinline|commit|. While these two
operations always are available, they only do something with the runtime has
the persistence system compiled into it.

The primary point of the persistence system is to durably commit to a function
to be run on process restart. As your XPLAN program runs, your XPLAN program
can \lstinline|commit| to a \lstinline|function : Row Args -> ()| which will be
run on the restart of the binary, which should resume the current state of the
program. This is the core primitive that lets processes live forever from a
committed checkpoint. The commit operation only writes data that hasn't been
persisted.

The secondary point is to move durable data off of the buddy heap onto a heap
where the Linux kernel can manage the paging of currently unneeded
data. The \lstinline|precommit| operation copies a tree of pins onto the
persistence heap, returning the persisted root value. This doesn't change the
committed startup function.

The purpose of \lstinline|precommit| is to allow larger writes to happen
in the background without blocking smaller commits which advance the
actual state.  The intended use-case here is that you persist states with
a snapshot and an event log, and resume by loading the latest snapshot
and then replaying all of the events since then.  Writting a new event is
fast, and will use \lstinline|commit|.  Snapshots are a lot bigger, and
will use \lstinline|precommit| on a background thread in order to commit
a full state to disk without blocking the commit of additional events.
Once a snapshot is fully written, it will be included in the next event,
where each event references a snapshot whose state lags behind the
event itself.

\subsection{Low Level Page Storage}

\subsubsection{Mounting The Current Binary With Mmap}

Normally, the Linux kernel will prevent a binary that's being executed from
being opened for writing. If you try to open a writeable file descriptor, the
kernel will return an \lstinline|ETXTBSY| error. This means we have to go
through some hoops that have to be performed right during a
program's \lstinline|_start| entry point.

The high level idea is that during \lstinline|_start|, we create an anonymous
file with \lstinline|memfd_create|, copy the ELF portion of our binary to it,
change the ELF entrypoint to \lstinline|_restart| and then execveat this
anonymous file.

The first thing we do is we open the current executable
(\lstinline|/proc/self/exe|) as read only, both so we can read the file, but
also so we have a file descriptor to the current binary that we can use on the
other side of the exec which is TOCTOU safe.

We then read current ELF header to figure out how much to copy, create a memfd,
truncate it to the size of the ELF portion of the input file, copy the ELF
data, change the ELF entrypoint to \lstinline|_restart| which is the second
stage execution, disable ASLR (since we need address space predictability for
all of this to work), and then execveat the new anonymous program we just
created.

\begin{asm}
.global boot_start
boot_start:
	## Stack layout (32 bytes total):
	##    [rbp - 8]   : exe file fd
	##    [rbp - 16]  : memfd fd
	##    [rbp - 24]  : ELF binary size
	##    [rbp - 32]  : temporary buffer for patching binary
	mov	rbp, rsp
	add	rsp, -32

	mov	rax, SYS_OPEN
	lea	rdi, [proc_self_exe]	# open current executable
	mov	rsi, O_RDONLY		# no O_CLOEXEC so hold across exec
	syscall
	mov	[rbp - 8], rax		# store exe file fd; should always be 3

	mov	rdi, rax     	 	# fd = exe file fd
	lea	rsi, [elf_header]	# buf
	mov	rdx, 64			# count = 64
	mov	r10, 0			# offset = null
	mov	rax, SYS_PREAD64	# Read ELF header into memory.
	syscall

	call	elf_binary_end
	mov	[rbp - 24], rax		# Set ELF binary size.

	lea	rdi, [plan_interpreter] # Anonymous memfd name
	mov	rsi, MFD_CLOEXEC	# Close the memffd on exec
	mov	rax, SYS_MEMFD_CREATE	# Create anonymous file descriptor
	syscall
	mov	[rbp - 16], rax		# store memfd fd

	mov	rdi, [rbp - 16]	 	# rdi=memfd fd
	mov	rsi, [rbp - 24]		# rsi=elf binary size
	mov	rax, SYS_FTRUNCATE	# Truncate memfd to ELF binary size
	syscall

	mov	rdi, [rbp - 16]		# out_fd = memfd
	mov	rsi, [rbp - 8]		# in_fd = exe file fd
	mov	rdx, 0	    		# offset = 0
	mov	r10, [rbp - 24]		# count = total elf binary size
	mov	rax, SYS_SENDFILE	# Copy ELF exe portion to memfd
	syscall

	mov	rdi, [rbp - 16]	  	# out_fd = memfd
	lea	rsi, [_restart]		# address of new entry point
	mov	[rbp - 32], rsi		# write address to buffer
	lea	rsi, [rbp - 32]		# buf/rsi = buffer with _restart
	mov	rdx, 8			# pointer is 8 bytes
	mov	r10, 24			# e_entry is at 24 byte offset
	mov	rax, SYS_PWRITE64	# Write patched entry point to memfd
	syscall

	mov	rdi, ADDR_NO_RANDOMIZE
	mov	rax, SYS_PERSONALITY	# Disable ASLR for _restart
	syscall

	lea	rbx, [rsp + 32]		# rbx = original stack frame pointer
	mov	rax, [rbx]  		# rax = argc

	mov	rdi, [rbp - 16]		# rdi = memfd fd
	lea	rsi, [empty_str]	# rsi = pointer to ""
	lea	rdx, [rbx + 8]		# rdx = argv
	lea	r10, [rbx+8+(rax+1)*8]  # r10 = envp
	mov	r8, AT_EMPTY_PATH	# r8 = flags
	mov	rax, SYS_EXECVEAT	# Restart program at _restart in memfd
	syscall
	## If we return here, execveat() failed.

	add	rsp, 32			# Restore the stack pointer.
	mov	rdi, 1			# exit code 1
	call	syscall_exit_group      # and call exit
\end{asm}

Once we call execveat, the kernel releases the execution lock on the original
file, while still maintaining file descriptor 3 which points at it. Since Linux
maintains symlinks to all open files in a process, and execveat doesn't close
files without the \lstinline|O_CLOEXEC| flag, we still have a read only file
descriptor to the original file accessible
with \lstinline|/proc/self/fd/3|. Since there's no longer an execution lock on
this file since we're now running from the memory fd, we can now open the file
for writing.

As some book keeping, we need storage for our writable file descriptor:

\begin{asmdata}
.global self_fd
self_fd:	.quad 0
\end{asmdata}

Then we just do normal process initialization before we call \lstinline|main|.

\begin{asm}
_restart:
	lea	rax, boot_main
	mov	[chosen_main], rax

	lea	rdi, [proc_fd_path]	# rdi=/proc/self/fd/3
	mov	rsi, O_RDWR		# open for reading and writing
	mov	rax, SYS_OPEN		# Open the file for writing
	syscall
	mov	[self_fd], rax		# selfd = open(...)

	mov	rdi, rax		# fd = exe file fd
	lea	rsi, [elf_header]	# rsi = elf_header buffer
	mov	rdx, 64			# count = 64
	mov	r10, 0			# offset = null
	mov	rax, SYS_PREAD64	# Reread the ELF header.
	syscall

	jmp	sharedbegin
\end{asm}

Finally, here's the helper function that parses the size of the ELF file. The
definition is based off the elf.h headers, which we don't want to bundle into
our project as a build dependency, and just calculates the length of the ELF
portion of the binary:

\begin{asm}
.global elf_binary_end
elf_binary_end:
	movzx   eax, WORD PTR elf_header[rip+58]
	movzx   edx, WORD PTR elf_header[rip+60]
	imul	eax, edx
	add     eax, DWORD PTR elf_header[rip+40]
	ret
\end{asm}

Finally, we have to store some read only constants used only in initialization:

\begin{asmdata}
.section .data
### ELF header of the executable loaded first thing on execution.
.align 16
elf_header:
	.zero	64

.section .rodata
proc_self_exe:
    .string "/proc/self/exe"
proc_fd_path:
    .string "/proc/self/fd/3"
plan_interpreter:
    .string "plan_interpreter"
empty_str:
    .asciz ""
\end{asmdata}

This execution trampoline is \textit{only} used in binaries that have the
persistence system compiled in. It is not included in rpn and the plan shell
because it confuses gdb when you try to restart a binary. Those binaries use a
small shim where \lstinline|_start| just immediately
calls \lstinline|_sharedbegin|.

\subsubsection{Persistence Format}

Our persistence heap starts at the first 4096 aligned page after the end
of the ELF data. Our file format is two pages for superblocks, and then a
series of data pages. We hardcode a page size of 4096 because that's the most
common page size, and we don't want the file format to diverge on systems where
that isn't true.

We allocate two pages, each dedicated to one superblock structure. Each
superblock is on its own page since filesystems only guarantees that individual
pages get committed. We then mmap the entire area of the file past this point.

\begin{clang}
const long page_size = 4096;
const long page_size_words = page_size / 8;

#define MAGIC_NUMBER 0x31764e414c50 // "PLANv1" in ASCII

#pragma pack(push, 1)       // Save alignment and set to 1 byte alignment
struct superblock {
    uint64_t magic;         // Magic number to identify our file type
    uint8_t sequence;       // Sequence number, wraps around 8-bit
    uint32_t val_checksum;  // Expected checksum for val.
    Val val;                // Tagged pointer into persistence heap.
    size_t write_offset;    // Current write position within the data area
    uint32_t checksum;      // Checksum of the superblock (excluding this field)
};
#pragma pack(pop)           // Restore previous alignment

struct superblock active_sb = {0};
off_t active_offset = 0;    // Binary location of current superblock
off_t inactive_offset = 0;  // Binary location of next superblock
\end{clang}

We have two superblocks with sequence numbers and CRC32 checksums, so that if
we fail during writing of a new superblock, we can detect that case on startup
and use the previous superblock.

We compute that by taking the checksum of the rest of the packed struct.

\begin{clang}
uint32_t calculate_superblock_checksum(struct superblock *sb) {
  return calculate_crc32c((const char*)sb,
                          sizeof(struct superblock) - sizeof(uint32_t));
}
\end{clang}

\subsubsection{Persistence System Initialization}

Each \lstinline|main| function has the choice to start the persistence
subsystem or not. If a binary does want to initialize the system, they
call \lstinline|init_persistence| with the file descriptor of the file they
want to mount. This file descriptor is usually the \lstinline|self_fd|
descriptor that was opened during \lstinline|_restart|, but can also be a
target file when building an image for the first time.

\lstinline|init_persistence| will calculate the locations of the superblocks,
mount the region past the superblocks in an mmap, figure out which is the
active superblock (if any), and record the bounds of the current persistence
image.

\begin{clang}
void init_persistence(int fd) {
  persistence_fd = fd;

  struct stat st;
  if (syscall_fstat(persistence_fd, &st) < 0) {
    syscall_close(persistence_fd);
    die("Failed to get file size");
  }
  filesize = st.st_size;

  off_t binary_end = elf_binary_end();
  off_t sb1_offset = round_up_align(binary_end, page_size);
  off_t sb2_offset = sb1_offset + page_size;

  // Data starts at the third page
  size_t data_start = sb1_offset + 2 * page_size;
  headersize = data_start;

  // Mount the data into memory. We do this before processing the superblocks
  // so we can do a top level item validation to make sure it was written
  // correctly.
  //
  // TODO: Lay out memory maps so they use all address space, collaborating
  // with the buddy allocator.
  const u64 rw = 3; // PROT_READ | PROT_WRITE
  const u64 map_flags = MAP_SHARED | MAP_FIXED;
  const u64 base_size = 1ULL << 39;
  persistence_start = syscall_mmap((u64*)base_addr,
                                   base_size, rw,
                                   map_flags, persistence_fd,
                                   data_start);

  // Read and check the validity of the two superblocks.
  struct superblock sb1, sb2;
  bool have_sb1 = false, have_sb2 = false;
  if (filesize >= sb1_offset + sizeof(struct superblock)) {
    have_sb1 = read_superblock(persistence_fd, sb1_offset, &sb1);
  }
  if (filesize >= sb2_offset + sizeof(struct superblock)) {
    have_sb2 = read_superblock(persistence_fd, sb2_offset, &sb2);
  }

  find_active_superblock(sb1, have_sb1, sb1_offset,
                         sb2, have_sb2, sb2_offset);

  persistence_cur = persistence_start + active_sb.write_offset;
  persistence_end = persistence_start + (base_size / 8);
}
\end{clang}

\subsubsection{Reading a Superblock}

When we read or write a superblock, we do it by reading directly from the file,
instead of reading or writing to a memory mapped page. (We are following what
LMDB does here.)

We verify that we did read the right amount of data, verify it has the right
magic number, verify that the checksum of the superblock is right, and verify
that the checksum for the value the superblock points to is correct.

Returns true if everything is valid, false if anything is invalid.

\begin{clang}
bool read_superblock(int fd, off_t offset, struct superblock *sb) {
  ssize_t bytes_read = syscall_pread64(
      fd, (char*)sb, sizeof(struct superblock), offset);

  if (bytes_read != sizeof(struct superblock)) {
    if (bytes_read > 0 || bytes_read == -22 /* EINVAL */) {
      // This might be a new file or truncated file
      return false;
    }
    die("Failed to read superblock");
  }

  if (sb->magic != MAGIC_NUMBER) {
    printf("Invalid superblock (wrong magic number) at offset %ld\n",
           (long)offset);
    return false;  // Block is not valid
  }

  uint32_t expected_checksum = sb->checksum;
  uint32_t actual_checksum = calculate_superblock_checksum(sb);
  if (expected_checksum != actual_checksum) {
    printf("Superblock corruption detected (checksum mismatch) at offset %ld\n",
           (long)offset);
    return false;  // Block is not valid
  }

  expected_checksum = sb->val_checksum;
  actual_checksum = crc32_checksum_for(sb->val);
  if (expected_checksum != actual_checksum) {
    printf("Superblock corruption detected (content mismatch) at offset %ld\n",
           (long)offset);
    return false;  // Block is not valid
  }

  return true;
}
\end{clang}

\subsubsection{Writing a Superblock}

Writing a superblock calculates and sets the checksum of the rest of the block,
and then writes and fsyncs the file descriptor to make sure the data is
persisted.

\begin{clang}
void write_superblock(int fd, off_t offset, struct superblock *sb) {
    sb->checksum = calculate_superblock_checksum(sb);

    ssize_t bytes_written = syscall_pwrite64(
        fd, (char*)sb, sizeof(struct superblock), offset);
    if (bytes_written != sizeof(struct superblock)) {
      die("Failed to write superblock");
    }

    if (syscall_fsync(fd) == -1) {
      die("Failed to fsync superblock");
    }

    // Swap offsets.
    off_t tmp = active_offset;
    active_offset = inactive_offset;
    inactive_offset = tmp;
}
\end{clang}

\subsubsection{Determining Which Superblock To Use}

On startup, in \lstinline|init_persistence|, we have to initialize the state of
which superblock is active and which one isn't. We look at which superblocks
exist in the file, and initialize the state to the newest valid superblock,
initializing a new superblock if there's no persistence data in the file.

\begin{clang}
void find_active_superblock(struct superblock sb1,
                            bool have_sb1,
                            off_t sb1_offset,
                            struct superblock sb2,
                            bool have_sb2,
                            off_t sb2_offset) {
  if (have_sb1 && have_sb2) {
    // Handle wraparound by checking if the difference (considering
    // unsigned overflow) is less than 128
    if ((uint8_t)(sb1.sequence - sb2.sequence) < 128) {
      active_sb = sb1;
      active_offset = sb1_offset;
      inactive_offset = sb2_offset;
    } else {
      active_sb = sb2;
      active_offset = sb2_offset;
      inactive_offset = sb1_offset;
    }
  } else if (have_sb1) {
    active_sb = sb1;
    active_offset = sb1_offset;
    inactive_offset = sb2_offset;
  } else if (have_sb2) {
    active_sb = sb2;
    active_offset = sb2_offset;
    inactive_offset = sb1_offset;
  } else {
    // Write the initial superblock
    active_sb.magic = MAGIC_NUMBER;
    active_sb.sequence = 1;
    active_sb.val_checksum = natural_crc32(0);
    active_sb.val = 0;
    active_sb.write_offset = 0;
    write_superblock(persistence_fd, sb1_offset, &active_sb);

    active_offset = sb1_offset;
    inactive_offset = sb2_offset;
  }
}
\end{clang}

\subsubsection{Allocating Raw Pages}

The low level primitive in the persistence file is the allocation of pages.
Right now, we allocate a number of pages at the end of the file. This function
only makes sure the file is large enough to contain the requested allocation
(and calling ftruncate if it isn't) and

This function does no syncing. It just hands out a pointer to persistence
backed memory.

\begin{clang}
u64* allocate_persistence_space_for(size_t sz) {
  size_t required_size = headersize
                       + ((persistence_cur - persistence_start) * 8)
                       + (sz * 8);

  if (filesize < required_size) {
    size_t new_size = round_up_align(required_size, page_size);
    if (syscall_ftruncate(persistence_fd, new_size) == -1) {
      die("Failed to extend file");
    }

    filesize = new_size;
  }

  // Our current pointer is always page aligned.
  u64* now = persistence_cur;
  persistence_cur += round_up_align(sz, page_size_words);
  return now;
}
\end{clang}

\subsubsection{Msyncing pages}

The caller to \lstinline|allocate_persistence_space_for| is responsible for
msyncing the pages once they are written to.

\begin{clang}

#define MS_SYNC 4

void msync_region(u64* begin, u64 word_size) {
  // Round word_size turned into bytes to the nearest page.
  u64 rounded_size = round_up_align(word_size * 8, page_size);

  // Sync the region
  if (syscall_msync(begin, rounded_size, MS_SYNC) == -1) {
    die("Failed to msync data");
  }
}
\end{clang}

\subsection{Persisting PLAN values}




\subsection{Persistence Collection}



\subsection{Scrubing and Validation}

All pins persisted have the crc32 bit set, meaning they have checksums to make
sure they point to the right data and haven't been corrupted either by the
system or by the garbage collector.

We calculate the checksums with the x86 crc32 instruction, which does
Castagnoli CRC.

\subsubsection{Calculating CRC32 Checksums}

For natural numbers, we just do a single CRC instruction on the direct natural
number.

\begin{clang}
u64 natural_crc32(u64 nat) {
  __int64_t crc = 0xFFFFFFFF;
  crc = _mm_crc32_u64(crc, nat);
  crc ^= 0xFFFFFFFF;
  return crc;
}
\end{clang}

For general PLAN values, note that we run the CRC32 loop over the heap header
one space before the pointer to the item:

\begin{clang}
u64 crc32_checksum_for(Val item) {
  if ((item >> 63) == 0) {
    return natural_crc32(item);
  } else {
    u64* ptr = PTR(item);
    u64 bitsz = get_bitsz(item);
    u64 wordsz = (bitsz + 63) / 64;

    return calculate_crc32c((char*)(ptr - 1), (wordsz + 1) * 8);
  }
}
\end{clang}

All usages of CRC32 are bound at load time by the following implementation,
which is unrolled to operate in qwords for efficiency.

\begin{asm}
.global calculate_crc32c
calculate_crc32c:			     # crc32c(rdi=data, rsi=bytesz):
	mov	eax, -1			     #   crc initialized to all ones
	test	rsi, rsi		     #   if size == 0
	jz	ret0			     #   then return zero
	mov	rcx, rsi		     #   rcx = len
	shr	rcx, 3			     #   blocks = len / 8
	and	rsi, 7			     #   rsi = len % 8 (tail size)
	test	rcx, rcx		     #   if blocks == 0
	jz	calculate_crc32c.tail4	     #   then jump past loop8
calculate_crc32c.loop8:			     # loop8:
	crc32	rax, qword ptr [rdi]	     #   process 8 bytes
	add	rdi, 8			     #   src += 8
	dec	rcx			     #   blocks--
	jnz	calculate_crc32c.loop8	     #   if remaining blocks, loop
calculate_crc32c.tail4:			     # tail4:
	test	rsi, 4			     #   if len & 4 == 0
	jz	calculate_crc32c.tail2	     #   then goto tail2
	crc32	eax, dword ptr [rdi]	     #   process 4 bytes
	add	rdi, 4			     #   src += 4
calculate_crc32c.tail2:			     # tail2:
	test	rsi, 2			     #   if len & 2 == 0
	jz	calculate_crc32c.tail1	     #   then goto tail1
	crc32	eax, word ptr [rdi]	     #   process 2 bytes
	add	rdi, 2			     #   src += 2
calculate_crc32c.tail1:			     # tail1:
	test	rsi, 1			     #   if len & 1 == 0
	jz	calculate_crc32c.done	     #   then goto done
	crc32	eax, byte ptr [rdi]	     #   process 1 byte
calculate_crc32c.done:			     # done:
	not	eax			     #   eax = eax xor 0xffffffff
	ret
\end{asm}

\subsubsection{Validating Each Item}

For each item, we check that its checksum matches or we kill the process.  If
the item is a tagged pointer, we also validate that the embedded crc is
unchanged.

\begin{clang}
void validate_item(Val item, u64 expected) {
  __int64_t crc = crc32_checksum_for(item);

  if ((item >> 63) == 1) {
    u64* ptr = PTR(item);
    if (ptr[-2] != crc) {
      printf("Item pointer CRC32 differs for %lx: ptr[-2]=%lx, actual=%lx\n",
             item, ptr[-2], crc);
      die("scrub failed");
    }
  }

  if (expected != crc) {
    printf("CRC32 differs for %lx: expected=%lx, actual=%lx\n",
           item, expected, crc);
    die("scrub failed");
  }
}
\end{clang}

\subsubsection{Scrubbing An Entire Pin Tree}

Our superblock structure records both a PLAN value and the expected crc32 of
the value. So to scrub a whole recursive pin structure, we set up an empty
red-black tree with a bump allocator so that we can track the regions of memory
we've already checked, and we then start the recursive process:

\begin{clang}
void scrub(Val item, u64 expected) {
  bump_alloc_t alloc;
  bump_alloc_init(&alloc, sizeof(rb_node));

  rb_tree tree;
  rb_init(&tree);

  scrub_item(item, expected, &tree, &alloc);

  bump_alloc_free(&alloc);
}
\end{clang}

\lstinline|scrub_item| is the recursive step, where we check each item in a
tree of pins. It uses the same \lstinline|mark_item_in_tree| tracking system
used by the persistence collector, because pin dependencies are massively
duplicated in practice.

This structure checks the validity of a single item, by checking that its own
crc32 that proceeds it matches the expected checksum, that the item's bytes
match the checksum, and then recurring based on whether this is a pin or a
megapin. This leverages megapin structure so that we don't have to recur
through the pin tree and can just linearly walk the megapin index.

\begin{clang}
void scrub_item(Val item, u64 expected, rb_tree* tree, bump_alloc_t* a) {
  if ((item >> 63) == 0) {
    // Check a direct reference.
    u64 crc = natural_crc32(item);
    if (crc != expected) {
      printf("Header hash differs from expected for %lx\n", item);
      die("scrub failed");
    }

    return;
  }

  if (mark_item_in_tree(item, tree, a)) {
    return;
  }

  u64* ptr = PTR(item);
  if (ptr[-2] != expected) {
    printf("Header hash differs from expected for %lx\n", item);
    die("scrub failed");
  }

  validate_item(item, expected);

  u64 pin_count = ptr[2];
  if (ismegapin(item)) {
    u64 megapin_count = ptr[6 + 2 * pin_count];

    u32* pin_crcs = (u32*)(ptr + 6 + 2 * pin_count + 1 + 2 * megapin_count);

    // nonrecursively scrub the pins
    for (size_t i = 0; i < pin_count; ++i) {
      if (mark_item_in_tree(ptr[6 + i], tree, a))
        continue;

      validate_item(ptr[6 + i], pin_crcs[i]);
    }

    // recursively scrub the megapins.
    for (size_t i = 0; i < megapin_count; ++i) {
      scrub_item(ptr[6 + 2 * pin_count + 1 + i],
                 pin_crcs[pin_count + i],
                 tree, a);
    }
  } else {
    u32* pin_crcs = (u32*)(ptr + 6 + pin_count);
    for (size_t i = 0; i < pin_count; ++i) {
      scrub_item(ptr[6 + i], pin_crcs[i], tree, a);
    }
  }
}
\end{clang}

\section{Naive Compiler}

The native runtime will include an extremely simple.  See
\lstinline|judge.pdf| for the details of that proposal.

However, in the meantime, we still need to recognize primop wrappers,
because otherwise primops have a ton of overhead.

\subsection{Wrappers}

A primop wrapper is a function which take a certain number of arguments,
and then directly calls a single primop with them.  These are used
everywhere, since the primop ABI is uncurried, but we want currying
because it's faster and cleaner.

Here's a simple example, a law which invokes the Add operation:

\begin{verbatim}
    (a b)&(<15> (%Add a b)) =>

        {0 2 (0 <15> ((0 "Add" 1) 2)}
\end{verbatim}


\subsubsection{Recognizing Wrappers}

In order to recognize a wrapper, we need a routine which recognizes this
pattern, and tells use the op, arity, and hd.  Or just tells us that it
didn't match.

If it is a wrapper, we can then look that up and see if it is a known
operation, and then replace the pin/law executioner to directly call
that operation.

These laws all have a simple shape:

\begin{verbatim}
    {_ arity body(arity)}

    body(arity) = (0 <op:Nat> adt(arity))

    adt(0) = hd:Nat (where n>arity)
    adt(0) = [hd:Nat]
    adt(n) = (0 adt(n-1) n)
\end{verbatim}

\subsubsection{match}

\begin{verbatim}
ARG rsi -- The law to inspect
RET rax -- The primop method
RET rdx -- The primop class
RET r8  -- The law arity (zero if no match)
\end{verbatim}

\begin{asm}
match:                                      # match: rdi/law
	arity_	r8, rdi                     #   r8  = law.arity
	body_	rax, rdi                    #   rax = law.body
	jnkal	rax, match.none             #   if (body !~ 0[a b]) goto none
	ix0_	rdx, rax                    #   rdx = body.ix(0)
	jnpin	rdx, match.none             #   if (rdx !~ <i>) goto none
	unpin	rdx                         #   rdx = rdx.item
	jheap	rdx, match.none             #   if !direct(rdx) goto none
	ix1	rax                         #   x = body.ix(1)
	mov	rcx, r8                     #   i = arity
match.loop:                                 # loop: rdx/op r8/args rax/x rcx/i
	test	rcx, rcx                    #   if (i == 0)
	jz	match.head                  #     head
	jnkal	rax, match.none             #   if !(x ~ 0[a b]) goto none
	ix1_	r9, rax                     #   r9=b
	cmp	r9, rcx                     #   if (b != i)
	jne	match.none                  #     goto none
	ix0	rax                         #   x=a
	dec	rcx                         #   i--
	jmp	match.loop                  #   goto loop
match.head:                                 # head: rdx/op r8/arity rax/head
	jdirect	rax, match.noquote          #   if direct(x) goto noquote
	jnquo	rax, match.none             #   if !(x ~ 0[k]) goto none
	dref	rax                         #   x = k
match.found:                                # found:
	ret                                 #   return
match.noquote:                              # noquote: rdx/op r8/arity rax/hd
	cmp	rax, r8                     #   if (x > arity)
	ja	match.found                 #     then goto found
match.none:                                 # none:
	xor	r8d, r8d                    #   arity=0
	xor	eax, eax                    #   key=0
	xor	edx, edx                    #   op=0
	ret                                 #   return
\end{asm}

\section{Profiling}

\subsection{Profile Format}

The native runtime has built in lightweight profiling events. A thread can use
an XPLAN operation to start and stop recording profiling events, and another to
return the raw data to the caller.

Each profiling record is two values, where the first is a 63-bit direct number
with three bits of event tag information and 60-bits of unix time at 8ns
resolution. The second is event type specific, but for begin/end events, is the
actual law value being run. The trick here is that the native runtime is trying
to lay out memory directly in a form that is a) fast to produce, b) directly
handable to the programs.

That memory is laid out as a linked list of closures, where the most recent
page is at the start of the list. Therefore, the format for each profiling page
is:

\begin{verbatim}
     5[prev_page a1 a2 b1 b2 c1 c2 ...]
\end{verbatim}

This format moves most complexity of profiling from the native runtime to the
running program.

The only global variables needed are a pointer to the current allocation, a
pointer to where in that allocation to write to next, and an end pointer to
track when we need to overflow and allocate a new page.

\begin{asmdata}
.section .data
.global profile_base, profile_next, profile_end
profile_base:	.quad 0
profile_next:	.quad 0
profile_end:	.quad 0
\end{asmdata}

\subsection{Time Format}

We choose a 60-bit 8ns resolution because that's good up until the year 2262
while being close to a resolution that's drowned out by kernel overhead.

We use the following macro to turn a timespec pointer filled by the kernel into
our time format:

\begin{asm}
.macro PTIME60 dst, src
	mov	\dst, [\src+0]		     #   rax = tv_sec (time_t)
	mov	r12, [\src+8]		     #   r12 = tv_nsec (long)
	imul	\dst, \dst, 125000000	     #   ticks = tv_sec * (1e9 / 8)
	shr	r12, 3			     #   tv_nsec >>= 3
	add	\dst, r12		     #   sum ticks
	mov	r12, 60			     #   literal 60
	bzhi	\dst, \dst, r12		     #   mask 60 bits.
.endm
\end{asm}

\subsection{Push Profile Page}

Since the profiling results are a linked list of pages, pushing a profile page
is both the main initialization function and what gets called repeatedly to add
a new item to the linked list.

\begin{asm}
.set	PROFILE_SIZE,	4088
push_profile:				    # push_profile:
	mov	rdi, PROFILE_SIZE	    #   rdi = PROFILE_SIZE
	mov	rsi, 0x004		    #   rsi = closure tag
	call	buddy_malloc		    #   buddy_malloc(PROFILE_SIZE, 4)
	test	rax, rax		    #   if malloc returned null
	js	push_profile.oom	    #   then handle oom
push_profile.prepareclosure:		    # prepareclosure:
	mov	r8, [profile_base]	    #   check current profile_base
	test	r8, r8			    #   if profile_base == NULL
	jz	push_profile.write	    #   then skip cleanup
	push	rax			    #   save rax
	call	zero_remaining_profile	    #   zero out remaining profile page
	pop	rax			    #   restore rax
	mov	r9, 0xD050000000000000	    #   write correct header
	or	r8, r9			    #   or header with profile_base ptr
push_profile.write:			    # write
	mov	qword ptr [rax], 5	    #   Set hd to 5
	mov	[rax + 8], r8		    #   5[profile_base a1 a2 b1 b2...]
	mov	[profile_base], rax	    #   save new profile_base
	mov	rcx, rax		    #   end = start
	add	rcx, PROFILE_SIZE*8	    #   end = start + size
	mov	[profile_end], rcx	    #   set end
	add	rax, 16			    #   increment to first entry
	mov	[profile_next], rax	    #   set next write location
	ret
push_profile.oom:
	ud2
\end{asm}

\subsection{Zero Remaining Profile Page}

We don't want to initialize a full profile linked list closure at allocation
time because that will destroy cache performance. We only zero the remaining
unused structure when we have to either push another profile page to the
reverse linked list structure, or when we have to hand this chain of pages to
the program.

\begin{asm}
.global zero_remaining_profile
zero_remaining_profile:			    # zero_remaining_profile:
	mov	rdi, [profile_next]	    #   rdi = profile_next
	mov	rcx, [profile_end]	    #   rcx = profile_end
	sub	rcx, rdi		    #   rcx = remaining space
	shr	rcx, 3 			    #   bytes to words
	xor	eax, eax		    #   store 0
	rep	stosq			    #   memset
	ret				    #   return
\end{asm}

\subsection{recordevent}

This is the low level recording implementation for everything else. This takes
an input of the tag and the 2nd value to write to the log.

\begin{asm}
### TODO: This calls the SYS_clock_gettime call. YOU REALLY DO NOT WANT TO DO
### THIS IN PRODUCTION CODE. This introduces SYSCALL OVERHEAD to a common
### action we're doing all the time. The correct thing to do is to call the
### VDSO version of this, but that's going to require parsing the auxiliary
### vector at startup, then do minimal ELF parsing to find lookup the pointer
### to `__vdso_clock_gettime`. That's all hard. So for getting from 0 to 1, we
### punt and do the wrong thing. But the data is going to be quasi-invalid
### until we do that.
.global recordevent
recordevent:                                # recordevent(rdi=tag, rsi=2nd val):
	sub	rsp, 32			    #   stack space for timespec*
	mov	[rsp + 16], rdi		    #   save tag
	mov	[rsp + 24], rsi		    #   save 2nd val
recordevent.check:			    # check free space:
	mov	rax, [profile_next]	    #   get current next pointer
	add	rax, 16			    #   increment by 2 64bit words.
	mov	rcx, [profile_end]	    #   get end pointer
	cmp	rax, rcx		    #   if (next >= profile_end)
	jae	recordevent.overflow	    #   then handle overflow
recordevent.askfortime:			    # ask kernel for time:
	mov	rax, 228		    #   SYS_clock_gettime
	mov	rdi, 1			    #   CLOCK_MONOTONIC
	mov	rsi, rsp		    #   rsi = &timespec
	syscall				    #   call kernel
recordevent.record:			    # record record:
	PTIME60	rax, rsp		    #   timespec to 60-bit 8ns time.
	shl	rax, 3			    #   make room for tag
	mov	rdi, [rsp + 16]		    #   rdi = input tag
	or	rax, rdi		    #   OR time with tag
	mov	rcx, [profile_next]	    #   get current pointer
	mov	[rcx], rax		    #   low word: time+type tag
	mov	rdi, [rsp + 24]		    #   retrieve 2nd val
	mov	[rcx + 8], rdi		    #   high word: law pointer
	lea	rcx, [rcx + 16]		    #   advance next pointer by 16
	mov	[profile_next], rcx	    #   save updated pointer
	add	rsp, 32			    #   restore stack frame
	ret                                 #
recordevent.overflow:                       # overflow:
	ppush	rdi, rsi                    #   save input during allocation
	call	push_profile		    #   new profile frame
	ppop	rdi, rsi                    #   restore input
	jmp	recordevent.askfortime	    #   continue in before path
\end{asm}

\subsection{opprofile}

The main generator of profiling events is the opprofile function, which is the
jet of \lstinline|Profile| function. The \lstinline|Profile| function is
defined as just returning its second argument.

In the fast path, when profiling is disabled (ie there are no profiling frames
to write to), the overhead is just one fused compare-and-jump and one register
move.

We must absolutely prevent references to GC1 from leaking into GC2 because that
breaks our garbage collection semantics. So we handle this by ignoring any
\lstinline|Profile| call that contains data on a thread's Cheney heap. All
long term Law objects should be on GC2 or GC3.

\begin{asm}
opprofile:				    # opprofile(rdi=law rsi=ret val):
	cmp	qword ptr [profile_next], 0 #   if profiling is enabled
	jne	opprofile.start		    #   then jump to profile handling
opprofile.default:                          # default:
	mov	rax, rsi		    #   setup 2nd argument
	JRAX				    #   evaluate 2nd argument
opprofile.start:			    # start:
	jdirect	rdi, opprofile.savespace    #   if direct, start immediately
	ptr_	rcx, rdi                    #   rcx = PTR(rdi)
	cmp	rcx, r14                    #   if (1st < heap_end)
	jb	opprofile.checkincheney     #   then check if in cheney heap
opprofile.savespace:                        # save space:
	sub	rsp, 16			    #   stack space for two
	mov	[rsp], rdi		    #   save law pointer
	mov	[rsp + 8], rsi		    #   save eval thunk
opprofile.recordbegin:                      # record begin:
	mov	rsi, rdi		    #   rsi = "2nd val"
	mov	rdi, 0			    #   rdi = type tag 0 (begin)
	call	recordevent		    #   recordevent(0, law ptr)
opprofile.eval:				    # eval:
	mov	rsi, [rsp + 8]		    #   get saved thunk
	EVAL	rsi			    #   evaluate it
	mov	[rsp + 8], rsi		    #   put saved evaluation
opprofile.recordend:                        # record end:
	mov	rsi, [rsp]		    #   rsi = "2nd val"
	mov	rdi, 1			    #   rdi = type tag 1 (end)
	call	recordevent		    #   recordevent(0, law ptr)
opprofile.cleanup:                          # cleanup:
	mov	rax, [rsp + 8]		    #   retrieve result
	add	rsp, 16			    #   undo stack space
	JRAX				    #   evaluate 2nd argument
opprofile.checkincheney:                    # check in cheney:
	cmp	rcx, [heap_addr]            #   if (PTR(rdi) >= heap_addr)
	jae	opprofile.default           #   then can't record value
	jmp	opprofile.savespace         #   otherwise continue recording
\end{asm}

\subsection{SetProfEnabledOp}

The profiling system is explicitly enabled and disabled by the using code. This
XPLAN operation takes 0 or 1 for whether it's enabled.

\begin{asm}
SetProfEnabledOp:			    # SetProfEnabledOp(rdi=enabled):
	test	rdi, rdi		    #   if (rdi == 0)
	jz	SetProfEnabledOp.disable    #   then disable requested
SetProfEnabledOp.enable:		    # enable:
	cmp	qword ptr [profile_next], 0 #   if profiling already enabled
	jne	ret0			    #   then just return 0
	call	push_profile		    #   push a profile frame to start
	ret				    #   done
SetProfEnabledOp.disable:		    # disable:
	mov	qword ptr [profile_base], 0 #   profile_base = null
	mov	qword ptr [profile_next], 0 #   profile_next = null
	mov	qword ptr [profile_end], 0  #   profile_end = null
	ret				    #   let gc collect dead frames
\end{asm}

\subsection{GetProfOp}

The calling program must periodically call this XPLAN operation to return the
current profiling samples.

\begin{asm}
GetProfOp:				    # GetProfOp(rdi=()):
	cmp	qword ptr [profile_next], 0 #   if profiling is enabled
	jne	GetProfOp.enabled	    #   then jump to profile handling
	mov	rax, 0			    #   else return 0
	ret				    #   return
GetProfOp.enabled:			    # enabled:
	call	zero_remaining_profile	    #   zero the remaining profile
	mov	rax, [profile_base]	    #   get current closure base ptr.
	mov	r8, 0xD050000000000000	    #   write correct header
	or	rax, r8			    #   make rax tagged pointer
	mov	qword ptr[profile_base], 0  #   reset profile_base stack
	push	rax			    #   save result
	call	push_profile		    #   push a new profile buffer
	pop	rax			    #   restore result
	ret				    #   return
\end{asm}


\section{Seed}

\subsection{The Boot Seed}

The first argument to the \lstinline|plan| program is a seed to run
which contains an XPLAN program.

xseed programs have a calling convention similar to C programs: The
seed file is loaded, the function is passed the rest of the command-line
arguments as an array of nats, the result is evaluated, cast to a nat,
and then passed to the \lstinline|exit(2)| syscall.

This section will document the seed format, and the implementation of
the seed loader.

\subsection{The Seed Format}

\subsubsection{Introduction}

This will attempt to communicate the basic idea of Seed without getting
caught up in implementation concerns. The idea is that, if you
understand the basic format layout, that should give you pretty strong
intuitions about the how encoding and decoding work.

I will avoid talk about /why/ choices were made, since many choices are
informed by implementation concerns, and I want to avoid getting caught
up in that. Instead, I will only describe the format itself.

A seed file implements a template for a PLAN value, with a special
number of parameters.  These parameters will need to be passed into the
loader routine.

Because PLAN values often contain a lot of shared structure, this is
represented explicitly in the format.  The format contains a number of
fragments which can refer to each-other, and the final fragment is the
actual resulting value.

\subsubsection{The Header}

The first 40 bytes of a Seed is a header which contains 5 64-bit words.
All words in the format are represented in least-significant bit order.

\begin{itemize}
\item The number of parameters for the template.
\item The number of multi-word numbers.
\item The number of 64-bit numbers.
\item The number of 8-bit numbers.
\item The number of fragments.
\end{itemize}

\subsubsection{Numbers}

After the header, the format contains all of the numbers, in descending order.

\begin{itemize}
\item The first thing is the width of each multi-word number.
\item And then the actual data for each multi-word number
\item And then the data for each single-word number.
\item And then the data for each single-byte number.
\end{itemize}

\subsubsection{Scopes and References}

At each point in the seed, everything that came before is "in scope".
All of the arguments, all of the numbers, and every preceding tree
binding.

For a certain number of bindings in scope, we require a certain number of bits.
For example, if there are four things in scope, each reference will require two
bits.

\subsubsection{Fragments}

Trees are encoded as a bit-packed recursive structure.  Each node is a
reference applied to zero or more parameters.  This is encoded as:

\begin{itemize}
\item The size of the size in unary.
\item The size in binary.
\item The actual reference bits
\item Zero or more following nodes, one per parameter.
\end{itemize}

Because the high-bit of the actual size is always 1, so the high bit is
omitted in order to save space.  The only edge-case is the case where
there are zero parameters, but this is easy to handle.

Here are some examples of sizes and how they are encoded.  Something
tricky here is that these examples are listed LSB-first, which means
that the actual binary values need to be read in reverse.

\begin{verbatim}
    0 -> 1.
    1 -> 01.
    2 -> 001.0
    3 -> 001.1
    4 -> 00010.00
    5 -> 00010.11
    6 -> 00010.01
    7 -> 00010.11
    8 -> 00011.000
\end{verbatim}

\subsubsection{Encoding Pins and Laws}

The format only includes numbers and applications, so there actually
isn't any way to encode pins and laws.

However, the resulting template does not need to be in normal form, so
we can just take MkPin and MkLaw as arguments.  And we actually do not
need to take MkLaw as an argument, because it can be constructed using
just MkPin.

The specific conventions around which argument are passed in depends on
the context.

When loading a boot seed, we pass in MkPin and we construct MkLaw
ourselves.  Since the calling convention for MkLaw is going to get more
complicated, we likely should pass in MkLaw as well.

When loading a single node of a Pin DAG, we currently pass in MkPin but
only use it to construct laws, since all of the sub-pins are passed in
as references.

Since the calling convention for MkLaw is going to get more complicated,
we likely should pass in both MkPin and MkLaw during boot, and only
MkLaw when working with individual DAG nodes.


\subsubsection{Important Properties}

Seed has a number of properties which make it a good format for our uses:

\begin{itemize}
\item Decoding is fast, and requires very little code.
\item Decoding requires no significant data structures, just an array.
\item Loading numbers is a trivial memory read, as they are stored in
      aligned memory, in a machine-native format.
\end{itemize}

\subsection{The Seed Loader}

The built-in loaded is designed to be as small as possible, and does
no input validation.

As a result, it is not safe.  This is designed only for use with known
seeds during bootstrap.  If given an invalid seed, this will segfault.

There is no additional security concern here, since this is the first
thing that the process does, and we are already reading this as an XPLAN
program and running it, which can perform arbitrary system calls.

\subsubsection{mmapfile}

Given a path and a pointer to a statbuf (structure produced by fstat
syscall), this opens a file, gets the size, and then loads the entire
file into memory using mmap.

Returns the buffer, uses SysV calling conventions.

\begin{asm}
.global mmapfile
mmapfile:                                   # rdi=path rsi=statbuf
	push	r12
	push	rbx
	mov	r12, rsi                    # r12=statbuf
	xor	esi, esi                    # flags=0
	xor	edx, edx                    # mode=0
	call	syscall_open_chk            # open(path, flags, mode)
	mov	rbx, rax                    # fd to global
	mov	rdi, rax                    # fd
	mov	rsi, r12                    # statbuf = &seedstat
	call	syscall_fstat_chk           # fstat(fd, statbuf)
	xor	edi, edi                    # addr=0
	mov	rsi, [r12+48]               # len=file.size
	mov	edx, 1                      # prot=1
	mov	ecx, 2                      # flags=2
	mov	r8, rbx                     # fd = *seedfd;
	xor	r9d, r9d                    # off=0
	call	syscall_mmap_chk            # mmap(ptr,len,prot,flags,fd,off)
	mov	r12, rax                    # save buffer
	mov	rdi, rbx                    # fd = *seedfd
	call	syscall_close_chk           # close(rd)
	mov	rax, r12                    # restore result=buffer
	pop	rbx
	pop	r12
	ret                                 # return
\end{asm}

\subsubsection{seedfile}

Given a filename, load and return the seed, following SysV calling
conventions.

\begin{asm}
.global seedfile # RPN testing system calls this.
seedfile:
	sub	rsp, 160                 # 144+padding
	mov	rsi, rsp                 # rsi = &stat (stack local)
	call	mmapfile                 # rax = ptr
	mov	[rsp+152], rax           # save buf
	mov	rdi, rax                 # buf
	call	seed                     # seed(buf)
	mov	rdi, [rsp+152]           # ptr = buf
	mov	rsi, [rsp+48]            # len = stat.st_size
	mov	[rsp+48], rax            # save result
	call	syscall_munmap_chk       # munmap(ptr, len)
	mov	rax, [rsp+48]            # restore result
	add	rsp, 160                 # restore stack
	ret                              # return
\end{asm}

\subsubsection{seed}

Given a pointer to a seed buffer, load the encoded value and return it
via rax.

This just loads the header, loads all the sizes, and then calls into
\lstinline|frags|.

TODO: Document this.

\begin{asm}
seed:
	mov	r12, rdi
	push	r12
	call	seed.inner
	pop	r12
	ret
seed.inner:
	mov	rcx, [r12]
	lea	rsi, [r12 + 40]
seed.holeloop:
	test	rcx, rcx
	jz	seed.sizes
	xor	edi, edi
	push	rsi
	push	rcx
	push	r12
	call	mkpin
	pop	r12
	pop	rcx
	pop	rsi
	ppush	rax
	dec	rcx
	jmp	seed.holeloop
seed.sizes:
	mov	rcx, [r12 + 8]
seed.sizeloop:
	test	rcx, rcx
	jz	seed.bigs
	mov	r8, [rsi]
	sub	r15, 8
	mov	[r15], r8
	add	rsi, 8
	dec	rcx
	jmp	seed.sizeloop
seed.bigs:
	mov	rcx, [r12 + 8]
seed.bigloop:
	test	rcx, rcx
	jz	seed.words
	dec	rcx
	mov	r8, [r15 + rcx*8]
	mov	rdi, r8
	push	rcx
	push	r8
	push	r12
	call	reserve
	pop	r12
	pop	r8
	mov	rdi, rax
	mov	rcx, r8
	rep	movsq
	push	rsi
	mov	rdi, r8
	call	claim
	pop	rsi
	pop	rcx
	mov	[r15 + rcx*8], rax
	jmp	seed.bigloop
seed.words:
	mov	rcx, [r12 + 16]
seed.wordloop:
	test	rcx, rcx
	jz	seed.bytes
	mov	rax, [rsi]
	push	r12
	call	mkword
	pop	r12
	sub	r15, 8
	mov	[r15], rax
	add	rsi, 8
	dec	rcx
	jmp	seed.wordloop
seed.bytes:
	mov	rcx, [r12 + 24]
seed.byteloop:
	test	rcx, rcx
	jz	seed.frags
	movzx	r8, byte ptr [rsi]
	sub	r15, 8
	mov	[r15], r8
	inc	rsi
	dec	rcx
	jmp	seed.byteloop
seed.frags:
	mov	r8, rsi
	shr	r8, 3
	shl	r8, 3
	mov	rdx, [r12 + 24]
	mov	rdi, rdx
	add	rdi, [r12 + 0]
	add	rdi, [r12 + 8]
	add	rdi, [r12 + 16]
	shl	rdx, 61
	shr	rdx, 58
	mov	rcx, [r12 + 32]
	call	frags
	mov	rax, [r15]
	mov	rcx, [r12]
	add	rcx, [r12 + 8]
	add	rcx, [r12 + 16]
	add	rcx, [r12 + 24]
	add	rcx, [r12 + 32]
	lea	r15, [r15 + rcx*8]
	ret
\end{asm}

\subsubsection{frags}

Basically, the just loads a certain number of tree fragments, and
appends each one to the environment.

\begin{verbatim}
    rdi - ARG - The total size of the environment table.
    rsi - ARG - The environment (an array of plan values).
    rdx - ARG - The number of used bits in the current input word.
    rcx - ARG - The number of tree fragments to load.
    r8  - ARG - A pointer to the current input word.
    r10 - TMP - The number of fragments which have been read
    r9  - TMP - Misc
\end{verbatim}

This basically just calls \lstinline|frag()| in a for loop, but the
reference width may grow each time, so we need to recalculate that for
each fragment.

\begin{asm}
frags:                                      # rdi=en rsi=e rdx=boff r8=wp rcx=n
	push	rbx
	push	rbp
	push	r12
	xor	rbx, rbx                    #   i=0 in rbx
	mov	rbp, rdi                    #   en in rdi
	mov	r12, rcx                    #   n in r12
frags.loop:                                 # continue:
	cmp	rbx, r12                    #   if (i >= n)
	jae	frags.ret                   #     break
	lea	rdi, [rbp-1]
	lzcnt	r9, rdi
	mov	rdi, 64
	sub	rdi, r9                     #   bitsz = 64 - lzcnt(sz-1)
	mov	rsi, r15                    #   rsi = sp
	call	frag                        #   frag(bitsz, rsi)
	inc	rbp                         #   i++
	inc	rbx                         #   en++
	jmp	frags.loop                  #   contiue
frags.ret:                                  # break:
	pop	r12
	pop	rbp
	pop	rbx
	ret                                 #   return
\end{asm}

\subsubsection{frag}

This is a recursive function that just loads a single bit-packed tree
from the input. It will consume a certain number of bits, and the result
will be pushed on the PLAN stack.

All this function does is decode these inputs and uses them to construct
a lazy function call:

\begin{verbatim}
    (((head arg1) arg2) arg3)
\end{verbatim}

The logic here is pretty straightforward, and it's not that much code,
but it is quite a lot to understand since there is a lot of state.

\begin{verbatim}
    ARG rdi - refSz (bitSz of a reference into the environment)
    ARG rsi - table (environment)
    ARG rdx - ub (the number of bits that have been used in the current word).
    ARG r8  - fp (pointer into the current word of the input buffer).
    TMP rcx - temporary, or width-of-width
    TMP r9  - temporary, or number of arguments.
    TMP r10 - the next word from the input.
\end{verbatim}

The arguments are essentially state that is threaded through the entire
process. We don't write these to the stack except to avoid clobbing when
calling mkapp().

The tricky bits here are just the decoding of the bit-packed input.
The best way to understand this code is to first understand the format,
and then work through this logic on paper with some small examples.

A couple of finer points to keep in mind:

\begin{itemize}
\item
    Closures sizes up to 2**32 (or something like that) fit in a single
    word.  By not supporting that, the initial data for each node always
    fits in 64 bits, which means that all of our bit-decoding logic can
    work with registers.
\item
    However, we need to read 64 bits of data at a bit-offset.  In order
    to keep things simple and have less state, we just do this every time:

\begin{verbatim}
    word = (p[0]<<o) | (p[1]<<o)
\end{verbatim}

    This means that we are re-reading the same words multiple times,
    but this isn't a major concern because of caching, and it keeps the
    complexity tractable.
\item
    The handling of the zero-parameters case adds little code, but is
    a little bit tricky.
\end{itemize}

\begin{asm}
# state: rdi=refsz rsi=env rdx=used r8=ptr
# local: r10=word rcx=tmp r9=params
frag:                                       # frag:
	push	r12                         #   avoid clobbering r12
	call	frag.recur                  #   enter recursive logic
	pop	r12                         #   restore r12
	ret                                 #   return
frag.recur:                                 # recur:
	sub	rsp, 32                     #   frame=(params, ptr, refsz, env)
	mov	[rsp+16], rdi               #   save refsz
	mov	[rsp+24], rsi               #   save env
	mov	r10, [r8]                   #   word = ptr[0]
	test	rdx, rdx                    #   if (used == 0)
	jz	frag.size                   #   goto size
	mov	rcx, rdx                    #   (shr only words with rcx)
	shr	r10, rcx                    #   word <<= used
	mov	rcx, 64                     #
	sub	rcx, rdx                    #   remain = 64-used
	mov	r9, [r8 + 8]                #   next = ptr[1]
	shl	r9, rcx                     #
	or	r10, r9                     #   word = (word | next<<remain)
frag.size:                                  # size:
	xor	r9, r9                      #   params = 0
	tzcnt	rcx, r10                    #   szsz = ctz(word)
	shr	r10, rcx                    #
	shr	r10, 1                      #   word >>= szsz+1 (advance)
	inc	rdx                         #   used++ (for 0 case)
	test	rcx, rcx                    #   if (!szsz)
	jz	frag.head                   #     goto head (0 case has no size)
frag.args:                                  # args:
	dec	rdx                         #   used-- (undo inc)
	add	rdx, rcx                    #
	add	rdx, rcx                    #   used += szsz*2
	dec	rcx                         #   swid = (szsz-1)
	xor	r9, r9                      #
	bts	r9, rcx                     #
	dec	r9                          #
	and	r9, r10                     #   params = word&((2**swid)-1)
	bts	r9, rcx                     #   set high bit
	shr	r10, rcx                    #   word>>swid
frag.head:                                  # head: (r10=word r9=params)
	xor	rcx, rcx                    #
	bts	rcx, rdi                    #
	dec	rcx                         #
	and	rcx, r10                    #   rix = word & ((2**refsz)-1)
	mov	rcx, [rsi + rcx*8]          #   ref = env[rix]
	ppush	rcx                         #   push rcx
	add	rdx, rdi                    #   used += refsz
	cmp	rdx, 64                     #   if (used < 64)
	jb	frag.loop                   #     goto loop
	sub	rdx, 64                     #   used -= 64
	add	r8, 8                       #   ptr++
frag.loop:                                  # loop:
	test	r9, r9                      #   if (params == 0)
	jz	frag.break                  #     break
	mov	[rsp], r9                   #   save params
	call	frag.recur                  #   recur()
	mov	[rsp+8], r8                 #   save ptr
	call	mkapp                       #   mkapp()
	mov	r9, [rsp]                   #   restore params
	mov	r8, [rsp+8]                 #   restore ptr
	mov	rdi, [rsp+16]               #   restore refsz
	mov	rsi, [rsp+24]               #   restore env
	dec	r9                          #   params--
	jmp	frag.loop                   #   continue
frag.break:                                 # break:
	add	rsp, 32                     #   pop frame
	ret                                 #   return
\end{asm}

\section{Parallelism}

PLAN has no dependency on a C library. That means we need to implement all the
basic threading and parallelism primitives ourselves.

\subsection{Mutex}

Mutexes are a single 32 bit integer in memory and mutex handling functions take
a pointer to that memory address. That piece of memory can be in one of three
states:

\begin{asm}
.equ	MUTEX_UNLOCKED, 0		    # Mutex is available
.equ	MUTEX_LOCKED, 1			    # Mutex is locked, no waiters
.equ	MUTEX_LOCKED_WAITERS, 2		    # Mutex is locked with waiters
\end{asm}

\subsubsection{mutex\_init}

A mutex is initialized by zeroing out its int32 value.

\begin{asm}
.global mutex_init
mutex_init:				    # mutex_init(rdi=int*):
    mov DWORD PTR [rdi], MUTEX_UNLOCKED	    #   *rdi = MUTEX_UNLOCKED
    ret					    #   return
\end{asm}

\subsubsection{mutex\_lock}

Locking a mutex has a fast path, where you can locally just atomically change a
mutex's value from unlocked (0) to locked (1). When the mutex is uncontested,
taking the mutex is very, very fast.

When you try to take a mutex that's already locked, you have to worry about
whether if you're the first contestor. The implementation falls through to
\lstinline|wait_loop| which checks the \lstinline|cmpxchg|ed value of the
mutex and makes sure it's \lstinline|MUTEX_LOCKED_WAITERS| before calling the
kernel futex implementation.

In \lstinline|check_unlocked|, after futex wakeup or failed state change from
\lstinline|wait_loop|, you try to acquire the lock again. You always try to set
to \lstinline|MUTEX_LOCKED_WAITERS| because we know other threads might be
involved and we'll have to signal to them.

\begin{asm}
.global mutex_lock
mutex_lock:				    # mutex_lock(rdi=int*):
	mov	eax, MUTEX_UNLOCKED	    #   eax = expected value (0)
	mov	ecx, MUTEX_LOCKED	    #   ecx = new value (1)
	lock cmpxchg DWORD PTR [rdi], ecx   #   try atomically change 0 to 1
	je	mutex_lock.success	    #   if success, we're done
	push	r12			    #   slow path when locked
	mov	r12, rdi		    #   save the mutex pointer
mutex_lock.wait_loop:			    # wait_loop:
	cmp	eax, MUTEX_LOCKED_WAITERS   #   if already marked with waiters
	je	mutex_lock.futex_wait	    #   then go wait for kernel futex
	mov	eax, MUTEX_LOCKED	    #   eax = expected value (1)
	mov	ecx, MUTEX_LOCKED_WAITERS   #   ecx = new value (2)
	lock cmpxchg DWORD PTR [r12], ecx   #   try atomically change 1 to 2
	jne	mutex_lock.check_unlocked   #   if failed, check if unlocked
mutex_lock.futex_wait:			    # futex_wait:
	mov	rdi, r12		    #   restore rdi
	mov	edx, MUTEX_LOCKED_WAITERS   #   edx = 2
	call	futex_wait_private	    #   futex_wait(mutex*, WAITERS)
mutex_lock.check_unlocked:		    # check unlocked:
	mov	eax, MUTEX_UNLOCKED	    #   eax = expected value (0)
	mov	ecx, MUTEX_LOCKED_WAITERS   #   ecx = new value (2)
	lock cmpxchg DWORD PTR [r12], ecx   #   try atomically change 0 to 2
	jne	mutex_lock.wait_loop	    #   if failed, go back to wait loop
	pop	r12			    #   lock acquired, restore
mutex_lock.success:			    # success:
	ret				    #   return
\end{asm}

\subsubsection{mutex\_unlock}

When we unlock, we check whether the old value was \lstinline|MUTEX_LOCKED| so
we can tell the kernel to wake a different waiter if necessary. Note that we
unlock the mutex completely before calling into futex; that's expected for this
algorithm. Compare with how \lstinline|mutex_lock.check_unlocked| expects a 0
value when woken up.

\begin{asm}
.global mutex_unlock
mutex_unlock:				    # mutex_unlock(rdi=int*):
	mov	eax, -1			    #   -1 to subtract 1
	lock	xadd DWORD PTR [rdi], eax   #   xadd returns old value in eax
	cmp	eax, MUTEX_LOCKED	    #   if old value was 1 (now 0)
	je	mutex_unlock.done	    #   then no waiters to wake
	mov DWORD PTR [rdi], MUTEX_UNLOCKED #   unlock completely
	mov	edx, FUTEX_WAKE_ONE	    #   call futex to wake one waiter
	call	futex_wake_private	    #   rdi already contains mutex ptr
mutex_unlock.done:
	ret
\end{asm}

\subsection{Condition Variables}

A condition variable is a ``struct'' of a u64 pointer to a mutex and a sequence
variable.

\begin{table}[h]
\begin{tabular}{|l|l|}\hline
Offset & Description   \\\hline
+0     & mutex pointer \\
+8     & sequence var  \\\hline
\end{tabular}
\end{table}

The sequence variable is a 32-bit value. Technically, this could wrap around
but requires \lstinline|2^32| signals in the tiny window between seq load and
the \lstinline|futex_wait| kernel call, which is physically impossible in our
environment.

\subsubsection{condition\_init}

Our condition variables don't follow the \lstinline|pthread_cv| interface: a
paired mutex is selected at condition variable initialization time. This
actually simplifies the implementation.

\begin{asm}
.global condition_init
condition_init: 			    # condition_init(rdi=cv*, rsi=mx*):
	mov	qword ptr [rdi], rsi	    #   [rdi] = mutex pointer
	mov	dword ptr [rdi + 8], 0	    #   [rdi+8] = sequence var, 0 or 1
	ret
\end{asm}

\subsubsection{condition\_signal}

\begin{asm}
.global condition_signal
condition_signal:			    # condition_signal(rdi=cvar*):
	lock	inc dword ptr [rdi + 8]	    #   increment number of signals
	lea	rdi, [rdi + 8]		    #   wake the sequence var location
	mov	edx, FUTEX_WAKE_ONE	    #   wake only one waiter
	call	futex_wake_private	    #   call futex wake
	ret				    #   done
\end{asm}

\subsubsection{condition\_broadcast}

\begin{asm}
.global condition_broadcast
condition_broadcast:			    # condition_broadcast(rdi=cvar*):
	mov	r9d, 1			    #   increment by one
	lock xadd dword ptr [rdi + 8], r9d  #   prev value returned in eax
	inc	r9d			    #   r9d is now the new value.
	mov	r8, [rdi]		    #   requeue to the mutex
	lea	rdi, [rdi + 8]		    #   wake on the sequence var
	jmp	futex_cmp_requeue_private   #   tail call futex requeue
\end{asm}

\subsubsection{condition\_wait}

\begin{asm}
.global condition_wait
condition_wait:				    # condition_wait(rdi=cvar*):
	push	r12			    #   save register
	mov	r12d, [rdi + 8]		    #   save current sequence var
	push	rdi			    #   save input cvar*
	mov	rdi, [rdi]		    #   set up mutex pointer
	call	mutex_unlock		    #   mutex_unlock(cvar->mutex)
	pop	rdi			    #   restore cvar*
	push	rdi			    #   save cvar*
	add	rdi, 8			    #   rdi = cvar->seq
	mov	edx, r12d		    #   edx = sequence var expected val
	call	futex_wait_private	    #   call futex wait
	pop	rdi			    #   restore cvar*
	mov	rdi, [rdi]		    #   rdi is now mutex for rest of fun
condition_wait.wakeup_loop:		    # wakeup_loop:
	mov	eax, MUTEX_LOCKED_WAITERS   #   prepare locked+waiters state
	lock xchg DWORD PTR [rdi], eax	    #   grab mutex, mark contended
	test	eax, eax		    #   if mutex was unlocked before
	jz	condition_wait.done	    #   then we now hold mutex
	push	rdi			    #   save mutex pointer
	mov	edx, MUTEX_LOCKED_WAITERS   #   expect mutex to be locked
	call	futex_wait_private	    #   call futex wait
	pop	rdi			    #   restore mutx pointer
	jmp	condition_wait.wakeup_loop  #   continue
condition_wait.done:			    # done:
	pop	r12			    #   restore r12
	ret				    #   return
\end{asm}

\subsection{Read/Write Locks}

Our reader writer locks are one 56-byte structure with the reader and writer
structs inlined into the structure.

\begin{table}[h]
\begin{tabular}{|l|l|l|}\hline
Offset & Size & Description \\\hline
+0   & unsigned int, 4 bytes & mutex \\
+4   & 4 bytes & padding (for 8-byte alignment) \\
+8   & pointer, 8 bytes & readers\_cv.mutex pointer \\
+16  & unsigned int, 4 bytes & readers\_cv.sequence var \\
+20  & 4 bytes & padding (for 8-byte alignment) \\
+24  & pointer, 8 bytes & writers\_cv.mutex pointer \\
+32  & unsigned int, 4 bytes & writers\_cv.sequence var \\
+36  & 4 bytes & padding (for 8-byte alignment) \\
+40  & unsigned int, 4 bytes & active\_readers \\
+44  & unsigned int, 4 bytes & writer\_active \\
+48  & unsigned int, 4 bytes & waiting\_writers \\
+52  & 4 bytes & padding (for 8-byte alignment) \\\hline
\end{tabular}
\end{table}

Our reader/writer lock currently has a strong reader preference; this could
lead to writer starvation. This is fine for now, because the only usage of
rwlocks is communication between the buddy allocator and the concurrent garbage
collector where the allocator allocating should starve forward progress in
collection compared to starving allocation completion on work threads.

\subsubsection{rwlock\_init}

\begin{asm}
.global rwlock_init
rwlock_init:				    # rwlock_init(rdi=rwlock*):
	push	r12			    #   save r12
	mov	r12, rdi		    #   save rwlock pointer
	call	mutex_init		    #   initialize front mutex
	lea	rdi, [r12 + 8]		    #   rdi = &rwlock->readers_cv
	lea	rsi, [r12]		    #   rsi = &rwlock->mutex
	call	condition_init		    #   initialize readers cvar
	lea	rdi, [r12 + 24]		    #   rdi = &rwlock->writers_cv
	lea	rsi, [r12]		    #   rsi = &rwlock->mutex
	call	condition_init		    #   initialize writers cvar
	mov	DWORD PTR [r12+40], 0	    #   active_readers = 0
	mov	DWORD PTR [r12+44], 0       #   writer_active = 0
	mov	DWORD PTR [r12+48], 0	    #   waiting_writers = 0
	mov	rdi, r12		    #   restore rwlock pointer
	pop	r12			    #   restore r12
	ret				    #   return
\end{asm}

\subsubsection{rwlock\_read\_lock}

\begin{asm}
.global rwlock_read_lock
rwlock_read_lock:			    # rwlock_read_lock(rdi=rwlock*):
	push	r12			    #   save r12
	mov	r12, rdi		    #   save rwlock pointer
	call	mutex_lock		    #   lock our mutex
rwlock_read_lock.wait_loop:		    # wait_loop:
	cmp	DWORD PTR [r12 + 44], 0     #   if writer_active == 0
	jz	rwlock_read_lock.no_writers #   then advance to next step
	lea	rdi, [r12 + 8]		    #   rdi = rwlock->readers_cv
	call	condition_wait		    #   wait on readers_cv
	mov	rdi, r12		    #   restore rwlock in rdi
	jmp	rwlock_read_lock.wait_loop  #   try again
rwlock_read_lock.no_writers:		    # no_writers:
	inc	DWORD PTR [r12 + 40]	    #   increment readers atomically
	mov	rdi, r12		    #   rdi = rwlock*
	call	mutex_unlock		    #   mutex_unlock(rwlock*)
	pop	r12			    #   restore r12
	ret				    #   return
\end{asm}

\subsubsection{rwlock\_read\_unlock}

\begin{asm}
.global rwlock_read_unlock
rwlock_read_unlock:			    # rwlock_read_unlock(rdi=rwlock*):
	push	r12			    #   save r12
	mov	r12, rdi		    #   save rwlock pointer
	call	mutex_lock		    #   mutex_lock(rwlock*)
	dec	DWORD PTR [r12 + 40]	    #   decrement active_readers
	jnz	rwlock_read_unlock.done     #   if active_readers, then done
	cmp	DWORD PTR [r12 + 48], 0     #   if waiting_writers == 0
	jz	rwlock_read_unlock.done     #   then done
	lea	rdi, [r12 + 24]		    #   rdi = &mutex->writers_cv
	call	condition_signal	    #   signal one writer
rwlock_read_unlock.done:		    # done:
	mov	rdi, r12		    #   rdi = rwlock*
	call	mutex_unlock		    #   unlock the mutex
	pop	r12			    #   restore r12
	ret
\end{asm}

\subsubsection{rwlock\_write\_lock}

\begin{asm}
.global rwlock_write_lock
rwlock_write_lock:			    # rwlock_write_lock(rdi=rwlock*):
	push	r12			    #   save r12
	mov	r12, rdi		    #   save rwlock pointer
	call	mutex_lock		    #   lock the mutex
	inc	DWORD PTR [r12 + 48]	    #   increment waiting writers
rwlock_write_lock.wait_loop:		    # wait_loop:
	cmp	DWORD PTR [r12 + 40], 0     #   if there are active readers
	jnz	rwlock_write_lock.wait	    #   then wait
	cmp	DWORD PTR [r12 + 44], 0     #   if there's no writer active
	jz	rwlock_write_lock.done	    #   then finish up and take lock
rwlock_write_lock.wait:			    # wait:
	lea	rdi, [r12 + 24]		    #   rdi = &rwlock->writers_cv
	call	condition_wait		    #   wait on writers_cv
	mov	rdi, r12		    #   restore rdi
	jmp	rwlock_write_lock.wait_loop #   try again
rwlock_write_lock.done:			    # done:
	mov	DWORD PTR [r12 + 44], 1     #   mark writer as active
	dec	DWORD PTR [r12 + 48]	    #   decrement waiting writers
	mov	rdi, r12		    #   unlock mutex
	call 	mutex_unlock		    #   mutex_unlock(rdi)
	pop	r12			    #   restore r12
	ret				    #   return
\end{asm}

\subsubsection{rwlock\_write\_unlock}

\begin{asm}
.global rwlock_write_unlock
rwlock_write_unlock:			    # rwlock_write_unlock(rdi=rwlock*):
	push	r12			    #   save r12
	mov	r12, rdi		    #   save rwlock pointer
	call	mutex_lock		    #   lock the mutex
	mov	DWORD PTR [r12 + 44], 0	    #   mark writer as inactive
	cmp	DWORD PTR [r12 + 48], 0     #   if waiting_writers == 0
	jz	rwlock_write_unlock.signal  #   then signal readers
	lea	rdi, [r12 + 24]		    #   rdi = &rwlock->writers_cv
	call	condition_signal	    #   signal one writer
	jmp	rwlock_write_unlock.done    #   we woke the next writer
rwlock_write_unlock.signal:		    # signal:
	lea	rdi, [r12 + 8]		    #   rdi = &rwlock->readers_cv
	call	condition_broadcast	    #   broadcast to all readers
rwlock_write_unlock.done:		    # done:
	mov	rdi, r12		    #   rdi = rwlock*
	call	mutex_unlock		    #   unlock the mutex
	pop	r12			    #   restore r12
	ret
\end{asm}


\subsection{Barriers}

A barrier waits on a value to become 0. Used for count down tasks, usually
where there's one waiter who farms out tasks to threads.

\subsubsection{barrier\_set}

\begin{asm}
.global barrier_set
barrier_set:				    # barrier_set(rdi=barrier*,rsi=cnt):
    lock xchg [rdi], rsi		    #   *barrier = count
    ret					    #   return
\end{asm}

\subsubsection{barrier\_done}

\begin{asm}
.global barrier_done
barrier_done:				    # barrier_done(rdi=barrier*):
	mov	eax, -1			    #   decrement by 1
	lock xadd dword ptr [rdi], eax	    #   locked perform subtract
	cmp	eax, 1			    #   if we aren't last thread
	jne	barrier_done.done	    #   then don't wake anyone
	mov	edx, FUTEX_WAKE_ALL	    #   otherwise wake everyone
	call	futex_wake_private	    #   call futex
barrier_done.done:
	ret
\end{asm}

\subsubsection{barrier\_wait}

\begin{asm}
.global barrier_wait
barrier_wait:				    # barrier_wait(rdi=barrier*):
	mov	edx, [rdi]		    #   current number of waiters
	test	edx, edx		    #   if this is zero
	jz	barrier_wait.done	    #   then we have don't have to wait
	push	rdi			    #   save barrier*
	call	futex_wait_private	    #   call futex_wait
	pop	rdi			    #   restore barrier*
	jmp	barrier_wait		    #   try again
barrier_wait.done:
	ret
\end{asm}

\subsection{Threads}

We use the following structure for a thread:

\begin{table}[h]
\begin{tabular}{|l|l|l|}\hline
Offset & Size & Description \\\hline
+0   & unsigned int, 4 bytes & tid \\
+4   & 4 bytes & padding \\
+8   & unsigned int, 4 bytes & futex exit\_flag \\
+12  & 4 bytes & padding \\
+16  & pointer, 8 bytes & function pointer to run \\
+24  & void*, 8 bytes & argument to pass to function \\
+32  & void*, 8 bytes & return value storage \\
+40  & void*, 8 bytes & stack pointer \\
+48  & size\_t, 8 bytes & stack size \\\hline
\end{tabular}
\end{table}


\subsubsection{thread\_create}

Thread create takes three arguments: A pointer to empty memory for the thread
struct in rdi, a function pointer to run on the thread in rsi, and an argument
to pass to that function pointer in rdx.

The one thing that's subtle here is the handling of the start path. So
\lstinline|clone3| will return 0 to the child thread, and the pid to the parent
thread. When we jump into the child thread, we have to immediately set up the
stack by creating a base stack frame and make sure we're stack aligned.

\begin{asm}
.equ    CLONE_FLGS, CLONE_VM | CLONE_FS | CLONE_FILES | CLONE_SIGHAND | CLONE_THREAD
.equ    EIGHTMB,    8 * 1024 * 1024

.global thread_create
thread_create:                              # thread_create(rdi, rsi, rdx):
	push	rbp			    #   frame header
	mov	rbp, rsp		    #   establish frame
	push	r12                         #   save callee-saved
	mov	r12, rdi                    #   r12 = thread*
	mov	qword ptr [r12], 0          #   tid = 0 (init structure)
	mov	qword ptr [r12 + 8], 0      #   futex exit_flag = 0
	mov	qword ptr [r12 + 16], rsi   #   function pointer to run
	mov	qword ptr [r12 + 24], rdx   #   argument to function pointer
	mov	qword ptr [r12 + 32], 0     #   return value storage
	mov	qword ptr [r12+48], EIGHTMB #   stack size
	mov	rax, SYS_MMAP		    # 	 set up MMAP call for stack
	mov	rdi, 0                      #   addr = NULL
	mov	rsi, [r12 + 48]             #   size = stack_size
	mov	rdx, PROT_READ|PROT_WRITE   #   prot = read/write
	mov	r10, MAP_PRIVATE|MAP_ANONYMOUS|MAP_STACK
	mov	r8, -1                      #   no file descriptor backing
	mov	r9, 0                       #   offset = 0
	syscall				    #   mmap(...)
	cmp	rax, 0			    #   if return mmap < 0
	jl	thread_create.error	    #   then handle error
	mov	[r12 + 40], rax             #   stack_base = mmap addr
thread_create.do:			    # do:
	mov	rsi, [r12 + 40]		    #   base child stack pointer
	add	rsi, [r12 + 48]             #   base + size = top of stack
	mov	rax, SYS_CLONE		    #   SYS_CLONE
	mov	rdi, CLONE_FLGS             #   set clone flags
	xor	rdx, rdx                    #   ptid = NULL
	xor	r10, r10                    #   ctid = NULL
	xor	r8, r8                      #   newtls = 0
	syscall				    #   clone(flags, child_stack, ...)
	cmp	rax, 0			    #   test clone return value
	jl	thread_create.error	    #   rax < 0: error
	je	thread_create.start         #   child path (clone returns 0)
	mov	[r12], rax		    #   parent path:store rax=chld tid
	pop	r12			    #   restore r12
	mov	rax, 0			    #   return success
	leave				    #   restore stack
	ret				    #   return
thread_create.error:			    # error:
	leave				    #   restore stack
	ud2				    #   kill process
thread_create.start:			    # start:
	xor	rbp, rbp		    #   clear frame pointer for new thrd
	and	rsp, -16		    #   ensure 16-byte stack alignemnt
	mov	rdi, [r12 + 24]		    #   rdi = arg to function pointer
	push	r12			    #   save r12
	call	[r12 + 16]		    #   call function pointer
	pop	r12			    #   restore r12
	mov	[r12 + 32], rax		    #   save function result
	mov	eax, 1			    #   set exit mark
	lock xchg dword ptr [r12 + 8], eax  #   mark thread as exited
	lea	rdi, [r12 + 8]		    #   rdi = atomic exit_flag
	mov	edx, FUTEX_WAKE_ONE	    #   wake a single waiter
	call	futex_wake_private	    #   futex_wake()
	mov	rdi, rax		    #   function return val as exit code
	call	syscall_exit     	    #   and call exit
\end{asm}

\subsubsection{thread\_join}

Thread joining takes a pointer to a thread structure and optionally a pointer
to memory to return a value from the thread.

\begin{asm}
.global thread_join
thread_join:				    # thread_join(rdi, rsi):
	cmp 	dword ptr [rdi + 8], 0      #   if futex exit_flag is not set
	jz	thread_join.futex_wait	    #   then wait for thread
	test	rsi, rsi		    #   if no result pointer
	jz	thread_join.munmap_stack    #   then skip setting result
	mov	rax, [rdi + 32]		    #   get result
	mov	[rsi], rax		    #   set result
thread_join.munmap_stack:		    # munmap_stack:
	mov	rdi, [rdi + 40]		    #   rdi = stack pointer
	mov	rsi, [rdi + 48]		    #   rsi = stack size
	mov	rax, SYS_MUNMAP		    #   munmap
	syscall				    #   munmap(rdi, rsi)
	ret				    #   return
thread_join.futex_wait:			    # futex_wait:
	push	rdi			    #   save rdi
	push	rsi			    #   save rsi
	lea	rdi, [rdi + 8]		    #   wait on futex exit_flag
	mov	edx, 0			    #   expect 0
	call	futex_wait_private	    #   futex_wait()
	pop	rsi			    #   restore rsi
	pop	rdi			    #   restore rdi
	jmp	thread_join		    #   try again after futex wake
\end{asm}

\subsection{Futex Helpers}

Calling the futex syscall is repetitive across our parallelism code. We use
helpers that otherwise work with kernel syscall calling conventions.

\subsubsection{futex\_wait\_private}

\lstinline|rdi| is a u32 memory location to wait on, while \lstinline|edx| is
the expected value of that memory location for early futex return.

\begin{asm}
futex_wait_private:			    # futex_wait_private(rdi, edx):
	mov	rax, SYS_FUTEX		    #   rax = futex syscall
	mov	esi, FUTEX_WAIT|FUTEX_PRIVATE_FLAG
	xor	r10, r10		    #   timeout = NULL (wait forever)
	xor	r8, r8			    #   r8 unused
	xor	r9, r9			    #   r9 unused
	syscall                             #   rdi and edx already set
	ret                                 #   return
\end{asm}

\subsubsection{futex\_wake\_private}

\lstinline|rdi| is a u32 memory location to wake, while \lstinline|edx| is
the number of waiters to wake up.

\begin{asm}
futex_wake_private:			    # futex_wake_private(rdi, edx):
	mov	rax, SYS_FUTEX		    #   rax = futex syscall
	mov	esi, FUTEX_WAKE|FUTEX_PRIVATE_FLAG
	xor	r10, r10		    #   r10 ununused
	xor	r8, r8			    #   r8 unused
	xor	r9, r9			    #   r9 unused
	syscall				    #   rdi and edx already set
	ret				    #   return
\end{asm}

\subsubsection{futex\_cmp\_requeue\_private}

Requeues waiters from one memory location to another.

\lstinline|rdi| is a u32 memory location to wake. \lstinline|r8| is the memory
location to requeue all waiters other than the one we wake. \lstinline|r9d| is
the check sequence number.

\begin{asm}
futex_cmp_requeue_private: 		    # futex_cmp_requeue_private(rdir8r9)
	mov	rax, SYS_FUTEX		    #   rax = futex syscall
	mov	esi, FUTEX_CMP_REQUEUE|FUTEX_PRIVATE_FLAG
	mov	edx, 1			    #   wake exactly one waiter
	mov	r10, 0x7fffffff		    #   requeue all other waiters
	syscall				    #   r8 and r9d already set
	ret				    #   return
\end{asm}

\section{Proposals}

\subsection{Seed Format Changes}

\subsubsection{Constant Reference Width}

Right now, after each fragment is loaded, it is pushed to the stack,
which increases the maximum possible reference number, which potentially
increases bit-width of each reference.

Instead of doing this, we could just always operate with the maximum
reference size required by the file.  This would increase the size of
the seed file somewhat, but it would simplify decoding, and it might
actually improve the size of compressed seeds, since it makes repeated
patterns have a more predictable shape.

\subsubsection{Encoding Graphs}

If we use a constant reference width, that opens up the potential
for seed to be used to encode recursive values, like with LETREC.

To implement this, you would just pre-populate each fragment as a
blackhole'd thunk, and then you would update the thunk with the actual
value as each fragment is read.

The biggest downside here would be needing to validate that all references
are backwards, when loading seeds which are meant to be DAGs.  This is
fairly easy, however, since a validating loader already needs to keep
track of the maximum valid reference.

\subsubsection{Smaller Words}

The current format encodes fixed-width numbers as either Word64 or Word8,
but this wastes a lot of space.

We can actually add 32-bit and 16-bit words to this set without changing
the size of the header, or breaking alignment.

This would decrease the size of seeds, and add only a tiny bit more
complexity to the decoder.

The trick is to use a smaller word to encode the count of the smaller
words.  There are 2**32 possible 32-bit words, which is one too big to
fit in a 32-bit word, but if you exclude 16-bit words from that set,
then it always fits.  The same logic applies for 16-bit words, and then
we use a 16-bit words to represent the number of bytes.

A 32-bit word and two 16-bit words still weights 64 bits, so the rest
of the structure is unchanged.

\begin{verbatim}
struct header {
    uint64_t numArgs;
    uint64_t numBigs;
    uint64_t n64;
    uint32_t n32;
    uint16_t n16;
    uint16_t n8;
    uint64_t numFrags;
}
\end{verbatim}

\subsection{Pointer Tagging Optimizations}

There are a number of potential changes to this scheme which are worth
exploring.

\begin{enumerate}
\item
  The second bit being set for all non-nat values is no longer necessary,
  since we have a sufficiently efficient way to ask if a number is a
  natural using rorx and cmp.
\item
  The tag bits can be shifted all the way to the left, to provide a
  bit more extra data.
\item
  The tag on natural numbers should maybe include a size. This would
  remove the need for the size-metadata on the heap object for small
  numbers, it would make fetching the size cheaper, and would make
  comparisons of small bignats faster.  OTOH, it would make constructing
  nats a little bit more expensive and complicated.  This requires
  careful evaluation before proceeding.
\item
  Natural numbers should include a tag bit which indicates whether or
  not they refer to a nat that lives within a frozen pin.  This will
  give us the ability to keep unfrozen nats off of the moving heap, which
  avoids needing to copy them all the time, and also makes their
  locations stable.
\item
  The size of the tag and size information on closures could be expanded
  to 5 bits (max direct tag/size would be 30, this would hurt legibility
  in a debugging but would make switching on bigger ADTs possible without
  an indirection and would remove the need to store size metadata on
  bigger closures.
\end{enumerate}

\subsubsection{First Alternative}

The result of these changes may end up looking something like this,
but the specifics are TBD:

\begin{verbatim}
    u63 0$$$$$$$$$$$$$$$
    NAT 100001fsssssssss (f=frozen, s=size)
    PIN 10001000000000cm (c=hascrc32, m=ismegapin)
    LAW 1001000000000000
    CLZ 101000tttttzzzzz (t=tag, z=size)
    THK 1100000000000000
\end{verbatim}

The idea with size of the nat tag is to include 9 bits of size
information, which is enough to have known sizes for bignats up to
64 bytes.

-   Since bit-sizes 0-63 are always direct, this allows us to have
    directly known sizes for sizes up to 574 bits.

-   However, we need a special value to indicate "too big".  We will
    use all ones for that case.

-   Therefore, the procedure for extracting the size information would be
    to extract the size bits (via a mask), and then add 64.

-   Otherwise, you can find the bitsize by extracting the s bits, and
    then subtracting 63

\subsubsection{Second Alternative}

Another alternative, which continue to allow the high-byte to uniquely
define heap reference types, could be:

\begin{verbatim}
    u63 0$$$$$$$$$$$$$$$
    NAT 10000001ssssssss small indirect nat, (s=bitsize)
    BAT 10000010ssssssss big indirect nat, (s=bitsize)
    PAT 10000100ssssssss pinned nat, (s=bitsize)
    PIN 10001000000000cm (c=hascrc32, m=ismegapin)
    LAW 1001000000000000
    CLZ 10100000ttttzzzz (t=tag, z=size)
    THK 1100000000000000
\end{verbatim}


\subsection{Buffers / FATs}

Right now, large, unpinned natural numbers live directly on the moving
heap, but this has a number of downsides:

\begin{itemize}
\item
  This requires that this data be copied on every GC which is
  expensive for large buffers.
\item
  We want to be able to store generated code for laws in natural numbers,
  and this code will need to not be moved by garbage collection.
\item
  We want to be able to allocate temporary buffers in order to work
  with syscalls, and we need to be sure that those are not moved.
\end{itemize}

Here is a proposed sequence of proposals for introducing this concept
and then integrating it deeper and deeper into the system.

\subsubsection{xbuf}

GC2 already supports top-level nat allocations, and the bucket-marking
system used for all GC2 references should already work for marking
such nats, so a basic version of this should be possible with very
little change.

The heap layout and pointer tagging scheme would be totally conventional,
just a normal NAT allocation, just allocated in GC2.  Just introduce a
few XPLAN primitives:

\begin{itemize}
\item
    \lstinline|xbuf : Nat -> Nat| constructs a buffer of byte-size(n)
    as a Nat which is allocated as a top-level allocation at a fixed
    location.

    This should crash if asked to allocate a size smaller than 8, the
    actual allocation should be one byte bigger than requested and the
    high byte should be set to 1 (making this a Bar), so that normal
    in-place NAT operations can be used to manipulate this.
\item
    \lstinline|xptr : Nat -> Nat| converts from a number in fixed memory
    (either a fixed nat, or a nat that lives inside of a pin), and
    converts it into a pointer.

    This can be used with system calls to read data into a nat (even
    at an offset), or to write data from a nat (for example, to serialize
    data directly from pinned nat without needing to make a copy.

    This can also be used if we dynamically generate code, to get a code
    pointer that can be attached as the judgement for a pin or a law.
\item
    \lstinline|xtouch: a -> b -> b| is used to guarantee that a temporary
    buffer is not freed until we are doing using a raw pointer into it.
    This is equivalent to Seq, but with the guarentee that it will not
    be optimized away, even if an optimizer can prove that the value
    has already been evaluated. (right now, there is not difference
    since we do not optimizer Seq, but we will eventually).
\end{itemize}

Significant RPN tests should be added which demonstrate that this all
works correctly and is correctly collected.

This should be sufficient for basic usage, but it has a number of
operational problems which we become a problem under heavier production
use.

\subsubsection{GC1 Collection}

Heavy use of simple fixed buffers will result in a much higher allocation
rate in GC2, and keeping this allocation rate low (through pin compaction)
is a central pillar of our design.

However, fixed nats are not pins and cannot be referenced from pins or
shared between actors, so they should in theory be collectable directly
by GC1.

Here's an outline of how such a design could work:

\begin{itemize}
\item
    First, give fixed nats (FATs) their own pointer tag.  There is enough
    left space for this in the high byte, and the \lstinline|jinat/jnnat|
    checks are already range-checks, so they should be able to support
    multiple types at once without any change.
\item
    Second, change the layout of FATs so that they have a mark bit
    and an intrinsic linked list.  GC2 still requires a GC header on
    the outside, so the layout should be something like this:

    \begin{verbatim}
    | GC2HDR | mark | next | GC1HDR | word | word | word\end{verbatim}
\item
    Reserve the first stack slot of the shaddow (r15) stack to be a
    pointer to the FAT list.  Every FAT allocation prepends itself to
    this list.
\item
    During GC1 collection, when we encounter a reference to a FAT,
    set the mark.
\item
    After GC1 collection, traverse the linked list of FATs, and
    free+delete any node which is not marked.
\end{itemize}

This scheme should add very little overhead, and the result is that FATs
are freed promptly instead of clobbing up the heap between GC2 collections.

It's important the think through the synchronization issues between GC1
and GC2 collections very carefully, but intuitively this seems like it
shouldn't be a problem.

\begin{itemize}
\item When GC2 is not running there is no sync issue.
\item During initial mark collection, this scheme changes nothing.
\item After initial mark collection, all old/reachable things are marked.
\item During sweep, all allocations are allocated marked.
\item The sweep traversal uses a lock that is shared with free().  So,
      if GC1 frees a FAT while GC2 is sweeping, that should not be a problem.
\end{itemize}

\subsubsection{BigNats always FAT}

In addition to the explicit \lstinline|xbuf| operation, all large numbers
should be allocated as FATs.  This keeps the GC1 heaps small, which
improves GC frequency, and avoids copying these large binary objects
on every collection.

The conventional cut-off for this is 64 bytes.

This is conceptually trivial, but will be somewhat annoying in practice
because it breaks the \lstinline|reserve/claim| protocol.  However, if
we wait until we add support for nats that can be truncated in place,
this problem should go away, since that change will avoid the need for
this special dance altogether.

\subsection{Actors}

Our GC architecture is designed to be thread-safe, but we currently only
use it with one thread at a time.  Getting things to *actually* work
with multiple threads at once will likely require quite a lot of effort.

\subsection{Specialized Executioners}

In addition to the unknown executioners or thunks, we should create
specialized executioners for saturated (and oversaturated) calls to
known operations.

Something like this, for example:

\begin{verbatim}
    {THUNK(5), xknown2, &addop, Add, x, y}
\end{verbatim}

\begin{verbatim}
    {THUNK(7), xknown2over, &addop, Add, 0, 1, 2, 3}
\end{verbatim}

This would eliminate a lot of overhead, and wouldn't add all that much
complexity.  However, we will need a compiler before we can make use
of this.

\subsection{Freelists in Persistence}

Right now, our persistence file format is append only, with garbage collection
of regions only punching holes in the file, zeroing them and freeing the
space. But this doesn't allow reuse of this address space, and very long
running persistence files will run out.

After a garbage collection pass, we have a map of what regions of the file are
free. We should write this map to the file and allocate from this freelist.

\subsection{Register Reform}

First, treating the C stack as GC roots will allow us to get rid of the
r15 stack, freeing up another register.  It might be worth using rbp
as the heap pointer and rbx as the heap end, which would free up all of
the numbered registers.

Second, the convention of using r12 as a scratch registers conflicts
very badly with it's usual role as a callee-saved register, and another
register should replace it for that purpose: maybe r11?

Third, we gain a lot of value from using custom calling conventions
in a lot of places, but having reliable callee-saved registers would
make a lot of things easier.  These rules must be respected everywhere,
or else you can't rely on them anywhere.

Together, this changes would make r12, r13, r14, and r15 available as
callee-saved registers.

\subsection{Killing the Shaddow Stack}

Killing the shadow stack would free up a register, and it would also
mean that each actor has one less region to worry about.

In particular, the Erlang process model has a single allocation area
per-actor which contains the stack and the heap, and they grow in opposite
directions.  This makes allocating a new actor super cheap, and makes
he per-actor memory footprint very small for small/helper actors.

However, right now we use the RPN/shadow stack quite heavily.  Most uses
are just for register spilling, but we also use it as an actual stack
in quite a few places. All of these systems would need to be reworked
somehow.

\begin{itemize}
\item
    In the actual RPN debugger, used for the whole test suite.  For the
    basic tests, we can probably find some reasonable way to write these
    tests directly in C/C++.

    For more complex tests, we can probably just use seeds / planlisp
    / Sire.
\item
    The seed loader uses the stack to represent environments, and also
    as a logical stack for mkapp.  The former could just use a single
    explicit stack-frame, and the latter can be rewritten to use normal
    register conventions.
\item
    The graph reduction engine uses the shadow stack to handle
    oversaturation.  I'm not sure what the alternative would be, but it
    should be possible.
\item
    In Judge, the environment and all nested sub-expressions are
    explicitly laid out on the stack.  This will all go away with the new
    template expansion approach.
\item
    There are likely a few more uses that have been overlooked by the
    above list.
\end{itemize}

One other major complication is that scanning the C stack is not at
all safe.  And this is going to make the interface between C/asm even
trickier.  We somehow need to avoid scanning C frames.  We will either
need an explicit roots system or to avoid GC1 allocation from C code
at all.

\subsection{Faster Evaluation Preludes}

At present, almost all primops begin with a prelude which evaluates all
of the strict arguments.  Something like this:

\begin{verbatim}
    # opcopy: rdi=cnt rsi=sof rdx=dof rcx=src r8=dst
    opcopy:                                     # opcopy:
            NAT     rdi, rsi, rdx, rcx, r8      #   eval+cast cnt
            ENAT    rsi, rdi, rdx, rcx, r8      #   eval+cast sof
            ENAT    rdx, rsi, rdi, rcx, r8      #   eval+cast dof
            EVAL    rcx, rsi, rdx, rdi, r8      #   eval src
            EVAL    r8,  rsi, rdx, rcx, rdi     #   eval dst
    fastcopy:
            ...
\end{verbatim}

These evaluation macros \lstinline|EVAL| and \lstinline|ENAT| are
optimized to do as little work as possible in the cases where the routine
is not given a thunk.  And they accomplish this by only flushing to the
stack when *actually* given a thunk.

This works by examining each argument one by one.  If it is a thunk
everything is flushed to the stack, the relevant values are evaluated,
and everything is restored from the stack.  So, for the above example,
if we are given 5 thunks, we will do something like 20 memory writes
and 20 memory reads.

Unfortunately, in practice, the routines are almost always given thunks.
The only time when we would be given something *besides* a thunk is when
a law body passes in a hard-coded constant to a function.  This happens,
but it is very much not the common case.

Once we have an optimizing law compiler, law judgment itself will do a
lot of evaluation directly before deferring to the graph reduction engein,
and this will greatly increase the set of places where things besides
thunks are passed in.

But such an optimizing law compiler should *also* be able to just perform
the requisite evaluations itself, and then directly call into the fast
path (\lstinline|fastcopy| in this case), avoiding the entire prelude.

Given this combination of factors, it might be significantly more
efficient to just flush everything to the stack, evaluate each item
one-by-one, and the restore everything.  In this case, the memory options
would be reduced to 10 writes and 10 reads, which is significantly
less work:

\begin{enumerate}
\item Save all five arguments to the stack.
\item Evaluate each argument, replacing the stack slot with the result.
\item Restore all of the registers from the stack.
\end{enumerate}

And then, when using a more sophisticated judgment strategy, we would
instead do the evaluation in the caller, and then directly invoke the
fastpath, skipping this entire prelude.

\subsection{Exit Segfaults}

The code seems to segfault on \lstinline|CTRL-C|, but only on the
first run.  What causes this?

\subsection{GC2 Under Pressure}

What's the best way to handle heavy load of GC2?  Since our abstraction
hides the memory hierarchy, having any sort of hard-limits on the size
of data in each region is not great.

How can we avoid needing to make the buddy allocator resizable?

How can we make sure that heavy use of GC2 swaps, instead of killing
the whole process?

\subsection{GC2/GC3 Mutation}

One of the invariants that we rely on in order to have a concurrent
garbage collector without any write barriers is the complete lack
of mutation in GC2 and GC3.

Because of the design of our system, this invariant is fairly easy
to maintain.  However, there are a few cases where it would be nice to
violate it.

The big question is, can we sneak in these exceptions without losing
the properties which make concurrent GC tractile?

Here are the edge-cases that would be nice to introduce:

\begin{itemize}
\item
    Redirections: If a GC2 pin is persisted, we would like to redirect
    all other references to the on-disk variant.
\item
    Redirections: If we discover via hashing that two pins are identical,
    in either GC2 or GC3, we would like be able to redirect one to the
    other in order to avoid the duplication.
\item
    Lazy Hashing: It would be nice to be able to wait to calculate the
    cryptographic hash of pins in GC2 until they are actually needed.
\item
    Codegen: Since pins in GC2 and GC3 have stable locations, we can
    avoid needing to have a "constants" table for generated code, and
    instead generate machine code which directly references values and
    code, by raw pointer.

    For example, to constants or other code within the same pin.  But if
    we need to generate the code *before* the pin is constructed, then
    we don't can't know the memory-locations of these values in advance.

    It would make more sense to figure out in advance how big the generate
    code would be, then allocate the pin with the code zeroed, then
    actually fill in the code.
\item
    Moving code from GC2 to GC3.  Once we attach generated code to pins
    and laws, the generated code will somehow need to be updated in
    order to patch-up the pointers to point to new locations.
\item
    Compaction: It would be nice to be able to re-organize the heap on the
    fly, to reclaim parts of the address space and reduce fragmentation.

    However, this would require some combination of the above things in
    order to work well.
\end{itemize}

Some of this seems like it should be possible, but these things all need
to be worked through in incredibly detail before we can conclude that
this is viable.

For example, If we just constructed a pin in GC2 and no other code
has reference to it, then mutating NAT that contains the code should
not cause any issues.  Since only binary data is being mutated, no new
references can be introduced into the GC graph, so the concurrent GC
should simply not care at all.

Similarly, If we pre-populate all cryptographic hashes within GC2 with
a placeholder, then filling in that value later is also just a binary
change.  There are synchronization issues to think through, however.

This is *NOT TRUE* for GC3. Since we need to care about synchronization
and atomicity invariants as well, even in-place binary changes are a
big problem.

When inserting a redirection from (GC2 -> GC2) or (GC2 -> GC3) the thread
performing the indirection necessarily has a reference to both things,
so no new references can be "discovered" through this process.  This also
does not introduce any risk of introducing references from GC3 -> GC2,
since we would simply ban such redirections.

This two facts *seems* like they could be sufficient to make doing this
safe, but we need to tread super, super carefully here.

\subsection{Separate Code Section}

I've completely punted on this for now, since it introduces a lot of
complexity.  And that punting should continue for the foreseeable future.

However, ideally, we would have a separate area in GC2 and in GC3 for
code.  This way we don't have to map all of our data into executable
pages.  Also, code can be more compact if it can take advantage of the
fact that it is collocated in order to use relative jumps/references, etc.

However!  This may introduce so much complexity that it is simply not
worth it, despite the significant upsides.

\subsection{Lazy Pins}

We can't do this yet because this will definitely break compatibility
with the Haskell runtime, but one major change which should result in
a massive performance improvement is lazy pins and laws.

Right now, whenever we construct a Pin, we normalize the pin and
immediately copy everything to GC2.

However, when doing something like inserting a page of JSON into a hitch
database in the fulltag demo, we are constructing a huge pile of temporary
pins which immediately become garbage.

In the new PLAN standard, pins are *not* guaranteed to be normalized.
Also, we can separate out pin construction in GC1 from pin freezing
(normalization and movement to GC2).  The result will be that all of these
ephemeral pins never make it to GC2 and never even need to be copied.

At the end of each step, when we are ready to merge our final change
into the resulting state, *only then* do we we recursively freeze all of
the pins in the result.

This will massively decrease the amount of garbage allocated into GC2.

And it also means that we don't need to copy all of the data within a
pin each time we construct a pin.  The new version of a pin can share
data with the previous version.

For example, if you do a single insert into the hitchhikers of a hitch
node, you do not have to copy the whole table, just the insert into the
hitchhikers table and a new root.

I'm not sure if this is advisable in this context, but it even becomes
possible to use lazy evaluation to avoid work when performing an
iterative series of changes to a pin DAG.

\end{document}

%%% Local Variables:
%%% indent-tabs-mode: t
%%% tab-width: 8
%%% End:
