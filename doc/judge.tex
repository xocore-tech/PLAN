\documentclass[11pt]{article}

\usepackage{fancyhdr}
\pagestyle{empty}
\fancypagestyle{plain}{\pagestyle{empty}}

\usepackage{listings}
\usepackage[T1]{fontenc}
\usepackage{textcomp}
\usepackage{lmodern} % Latin Modern family (vectorized Computer Modern)
\usepackage{microtype}
\usepackage[paperwidth=210mm,paperheight=2000mm,margin=15mm]{geometry}
\usepackage{parskip}
\usepackage{needspace}
\usepackage[colorlinks=true,linkcolor=blue]{hyperref}

\lstset{
  basicstyle=\ttfamily\small,
  breaklines=true,
  frame=single,
  xleftmargin=1em,   % indent from left
}

\lstdefinelanguage{asm}{
  morecomment=[l]{\#},
}

\lstset{
  language=asm,
  basicstyle=\ttfamily\small,
  commentstyle=\color{black},
  tabsize=8,
  breaklines=false,
  captionpos=b
}

\lstdefinelanguage{Haskell}{
  morekeywords={case,of,if,then,else,data,type,class,where,do,let,in},
  sensitive=true,
  morecomment=[l]--,
  morecomment=[n]{\{-}{-\}},
  morestring=[b]",
}

\lstdefinestyle{haskellstyle}{
  backgroundcolor=\color{backcolour},
  commentstyle=\color{codegreen},
  keywordstyle=\color{blue},
  numberstyle=\tiny\color{codegray},
  stringstyle=\color{codepurple},
  basicstyle=\ttfamily\footnotesize,
  breaklines=true,
  numbers=left,
  numbersep=5pt,
  showstringspaces=false,
  tabsize=2,
}

\lstnewenvironment{asm}[1][] % optional args pass through
  {\lstset{language=asm,#1}}
  {}

\lstnewenvironment{asmdata}[1][] % optional args pass through
  {\lstset{language=asm,#1}}
  {}

\lstnewenvironment{clang}[1][]
                  {\lstset{language=C,#1}}
                  {}


\title{Native Law Judgements}
\author{Sol}

\begin{document}
\maketitle
\begin{abstract}

This document describes the design and implementation of the default
function evaluation engine of the native PLAN runtime system.

\end{abstract}
\section{Introduction}

In order to achieve a system which truly has zero dependencies, our
PLAN runtime system is written in assembly.  This makes the code quite
expensive to maintain.  Worse, the runtime system is designed to have
uptimes in the decades, which makes it impossible to do a live upgrade
of runtime system itself.  This constraint adds a number of difficult
and unconventional engineering challenges.

Our high level approach to these challenges is to keep the runtime system
as small as possible, and to instead move as much code into XPLAN,
which can be written in a much higher-level language and which can be
upgraded dynamically.

One of the most significant subsystems which we move into XPLAN is the
actual optimization logic used to run functions.  However, a number of
non-trivial tasks must still be possible using only the basic evaluation
system:

\begin{enumerate}
\def\labelenumi{\arabic{enumi}.}
\item
    We need to be able to run a significant body of code in order to
    test the runtime system itself.
\item
    The optimization engine will be written in source code in a high-level
    language.  We will need to be able to load a compiler and use it to
    build that optimization engine.
\item
    The compiler for this language will also be written in a high level
    language, so we will need to be able to run it against itself for
    bootstrapping.
\item
    In order to produce the actual seed-file that we need for
    bootstrapping, we need to be able to serialize the bootstrapping
    compiler.
\end{enumerate}

This tension between the need to minimize mechanical complexity, and
the need to achieve a usable performance baseline results in a somewhat
unique implementation strategy, which we will document in detail in the
rest of this document.

\section{Lazy Graph Reduction}

PLAN is a lazy graph reduction engine built using super-combinators.
A system like this, in theory, doesn't do any actual computation when
a function is called.  Functions just work something like a template
expansion, and the graph reduction engine does all of the actual work.

Here's a concrete example:

\begin{lstlisting}[language=Haskell]
foo :: Bool -> Int -> Int -> Int
foo x y z = if x then y+y else z+z
\end{lstlisting}

If \lstinline|foo| is invoked as \lstinline|(foo True 3 4)|, the actual
result of running the function is not 6, but tree of thunks, which is then
processed by the graph reduction engine in order to find a result:

\begin{lstlisting}[language=Haskell]
(((foo True) 3) 4)
((foo[True] 3) 4)
(foo[True 3] 4)
(((If True) ((Add 3) 3)) ((Add 4) 4)) -- expansion of foo
((If[True] ((Add 3) 3)) ((Add 4) 4))
(If[True ((Add 3) 3)] ((Add 4) 4))
((Add 3) 3)                           -- expansion of If
(Add[3] 3)
6                                     -- expansion of Add
\end{lstlisting}

Producing and consuming all of these thunks is highly inefficient, so
sophisticated runtime systems figure out which subset of the work will
always be performed, and just do all of that eagerly.  But this requires
somewhat sophisticated analysis and optimization work.

In order to keep the native runtime system simple, we avoid all of this
work and instead adopt the naive template-expansion approach.  But,
in order to produce something with a usable base-line of performance,
we also perform a number of rote optimizations which only require
trivial analysis.

\section{Optimizing Template Expansion}

Here's our example again:

\begin{lstlisting}
= (mapMaybe f o)
| If (Nil o) 0
| 0 (f (Ix 0 o))
\end{lstlisting}

Ignoring the issues around needing to make everything legible to GC,
the most stupid possible approach to template expansion would look
something like this:

\begin{lstlisting}[language=C]
Obj mapMaybe(Obj f, Obj o) {
    Obj result =
        App(App(App(If, App(Nil,o)),
                0),
            App(0,
                App(f,
                    App(App(Ix,0),
                        o))));
    tailcall tmp.execute()
}
\end{lstlisting}

This approach requires a very large number of small thunks to be
allocated, and each of those thunks must be separately processed by the
graph reduction engine.

To improve on this, we use a series of improvements on this basic output
in order to get much better performance.

\subsection{Optimization 1: Bigger Thunks}

We can do the same thing with fewer allocations, and make life a bit
easier for the graph reduction engine by constructing bigger thunks
instead of only use Apps.

\begin{lstlisting}[language=C]
Obj mapMaybe(Obj f, Obj o) {
    Obj result =
        App3(If, App(Nil,o),
                 0,
                 App(0, App(f, App2(Ix, 0, o))))
    tailcall tmp.execute()
}
\end{lstlisting}

\subsection{Optimization 2: Direct Closure Construction}

Notice that the \lstinline|(0 (f (Ix 0 o)))| expression is always
undersaturated, like an expression like \lstinline|(Add 3)| would be.
So the result will always be a closure.

Producing a thunk in this case results in a lot of wasted work.  The thunk
will examine the head, determine that the call is undersaturated,
allocate a closure, and update the thunk to point to the closure.

As long as the function call is to a known constant, it is trivial
to determine that the call is undersaturated, which allows us to skip
constructing this thunk, and instead directly construct a closure.

\begin{lstlisting}[language=C]
Obj mapMaybe(Obj f, Obj o) {
    Obj result =
        App3(If, App(Nil,o),
                 0,
                 Clz1(0, App(f, App2(Ix, 0, o))))
    tailcall tmp.execute()
}
\end{lstlisting}

\subsection{Optimization 3: Single Allocation}

Because the template expansion for each function is always the exact
same shape, we can avoid doing many small allocations, and instead just
do one big allocation.

Not only does this save work by avoiding repeated allocation, but it also
means that the work of actually filling in the template does not have to
make sure to make all of it's pointers available to the garbage collector.
No GC will happen during this initialization process.

\Needspace{10\baselineskip} % estimate lines needed
\begin{lstlisting}[language=C]
Obj mapMaybe(Obj f, Obj o) {
    save(f, o);
    void *buf = alloc(22);
    restore(f, o);

    fillapp1(&buf[0], Nil, o);        // a = (Nil o)    [weight=4]
    fillapp2(&buf[4], Ix, 0, o);      // b = (Ix 0 o)   [weight=5]
    fillapp1(&buf[9], f, b);          // c = (f b)      [weight=4]
    fillclz1(&buf[13], 0, c);         // d = (0 c)      [weight=3]
    fillapp4(&buf[16], If, a, 0, d);  // e = (If a 0 d) [weight=6]

    Obj *result = (Obj*) &result[17]; // 17 b/c first word is GC header
    tailcall result.execute()
}
\end{lstlisting}


\subsection{Optimization 4: Precomputed Template}

Not only is the size of the resulting graph constant, but so is it's
structure.  If we pre-compute the shape of the template, we can simply
copy the whole thing to our reserved space on the heap.

There are only a few changes that need to be made to this template after
it is copied:

-   Pointers between nodes internal to the graph need to be updated to
    point to the new node locations.

-   Reference to variables must be filled in.

-   Because references to constants (except direct nats) can be moved
    around by GC, we need to maintain a table of constants, and fill
    each of these in by replacing them with a reference into this table.

\Needspace{10\baselineskip} % estimate lines needed
\begin{lstlisting}[language=C]
Obj mapMaybe(Obj f, Obj o) {
    save(f, o);
    void *buf = alloc(22);
    restore(f, o);

    memcpy(template, buf);

    buf[2]   = Nil; // from constants array
    buf[3]   = o;   // from register
    buf[5]   = Ix;  // from constants array
    buf[7]   = o;   // from register
    buf[10]  = f;   // from register
    buf[11] += buf;  // convert offset to pointer
    buf[15] += buf;  // convert offset to pointer
    buf[17]  = If;   // from constants array
    buf[18] += buf;  // a: convert offset to pointer
    buf[20] += buf;  // d: convert offset to pointer

    Obj *result = (Obj*) &result[17]; // 17 b/c first word is GC header
    tailcall result.execute()
}
\end{lstlisting}


\subsection{Optimization 5: Code Generation}

As you can see from the previous example, the actual template-filling
logic is just a straight-line sequece of simple memory operations,
and none of this needs to be interleaved with garbage collection.

As a result, it is quite simple to just generate this machine code for
each law at law construction time.

In particular, we only need to be able to construct 4 assembly
expressions.  If constants=r11 and output=r12, then:

\begin{enumerate}
\item
  Load a constant: \lstinline|mov r10, [r12+o]|
\item
  Load from stack: \lstinline|mov r10, [r15+o]|
\item
  Write a register: \lstinline|mov [r14+o], reg|, (where reg can be:
  rax, r10, rdi, rsi, rdx, rcx, r8, or r9).
\item
  Write a argument: \lstinline|mov [r14+o], rdi|
\item
  Hydrate an internal: \lstinline|add [r14+o], r14|
\end{enumerate}

If we use 32-bit offsets for each output instead of trying to minimize
the size, we can use a simple 7-byte output for each instruction, which
allows a simple table to be used instead of complex encoding logic.

\begin{lstlisting}
Registers:

    r15           = stack
    r14, r13      = reserved (heap registers)
    r12           = temp
    r11           = constants
    rax           = self reference
    rdi,rsi..,r12 = arguments

Patterns:

    mov r12, [r11+x] ; 4D8BA3xxxxxxxx
    mov r12, [r15+x] ; 4D8BA7xxxxxxxx
    mov [r14+x], rax ; 498986xxxxxxxx
    mov [r14+x], rdi ; 4989BExxxxxxxx
    mov [r14+x], rsi ; 4989B6xxxxxxxx
    mov [r14+x], rdx ; 498996xxxxxxxx
    mov [r14+x], rcx ; 49898Exxxxxxxx
    mov [r14+x], r8  ; 4D8986xxxxxxxx
    mov [r14+x], r9  ; 4D898Exxxxxxxx
    mov [r14+x], r10 ; 4D8996xxxxxxxx
    mov [r14+x], r11 ; 4D899Exxxxxxxx
    mov [r14+x], r12 ; 4D89A6xxxxxxxx
    add [r14+x], r14 ; 4D01B6xxxxxxxx
\end{lstlisting}

TODO: figure out how to encode the prelude.  If we store the executable
code in a big nat outside of the moving heap, then we actually can do
the allocation directly, if that ends up being easier.

\begin{lstlisting}
judgement_prelude:
    if (r14+sz >= r13)
        gc(sz)
    r11 = self.metadata.template
    memcpy(r14, r11, sz*8)
    r11 = self.metadata.constants
    [[rest]]
\end{lstlisting}

TODO: figure out how to encode the tail logic:

\begin{lstlisting}
hp += 22
Obj *result = (Obj*) &result[17];
tailcall result.execute()
\end{lstlisting}

The tail logic should probably just be a tail call into a constants
function.

\begin{lstlisting}
tail_logic:
    hp += rdi
    result = &heap[rsi];
    tailcall result.execute()
\end{lstlisting}

\subsection{Optimization 6: Avoid Thunk Update}

Loops in function programming languages are encoded using tail recursion,
which makes tail recursion optimization an essential feature of any
purely functional programming language.

Tail call optimization is a bit different in a minimalist, purely
functional language like PLAN, because things like \lstinline|if| and
\lstinline|case| and just normal functions.

Lazy evaluation essentially works by repeatly simplifying the outer
expression until it is "done" and then repeating the process for the inner
expressions.  This iterative process automatically provides constant-space
tail recursion except for one tricky-edge case: thunk caching.

After each thunk is evaluated, it is modified so that subsequent
references return the pre-computed value, instead of running the
computation again.  But, in order to do this modification, we need to
run a piece of logic after the evaluation.  This requires a stack frame,
and breaks tail recursion.

The problem is that we end up with a long chain of tiny stack frame
which just update thunks.  If we could find some way to use an alternate
thunk structure which doesn't perform the update, all of these tiny
stack frames would go away.

Consider the following example:

\begin{lstlisting}
= (sillyAdd a b)
| Seq a
| Ifz b a
| sillyAdd Inc-a Dec-b

= (sillyDouble a)
| sillyAdd a a
\end{lstlisting}

When sillyDouble is run, we know that the call to sillyAdd will always be
evaluated immediately, so the thunk will definitely only be evaluated
once.  In this case, we can trivially use a non-updating thunk and
recover tail recursion.

In `sillyAdd`, that means that (Seq a ..) doesn't need to update, but
what about (Ifz b a ..)?  Well, we know that Seq evaluates each argument
exactly once, so any expression passed as an argument to `Seq` does not
need a thunk update.  And the same with `Ifz`.  Ifz will evaluate each of
it's arguments at most 1 time, so there is no need for the thunk update.

This chain of reasoning eliminates the thunk updates for the whole chain
of calls leading up the recursion, and the result is that we recover
tail call optimization.

Doing this analysis in general requires a significant amount of work.
However, in XPLAN, things like Seq and Ifz are treated as primitives,
so we can simply hard-code this information on functions which are
simple wrappers for primops (see the later section on recognizing primop
wrappers).

Once this information is available, the process is trivial.  Whenever
we see a saturated call to a function, remember which arguments are
evaluated at most once.  When generating a thunk for such an argument,
use the non-updating version.

\section{Template Layout}

First of all, we special case trivial functions which just return
an argument or a constant value.  These functions don't require any
allocations at all.

Computing the template is fairly easy, we just need to figure out what
all of the nodes are and where they belong in the tempate.  There are
only a few cases:

\begin{enumerate}
\item
  A saturated or unknown function call will be a thunk, and will require
  args+3 words.
\item
  An undersaturated function call will be a closure, and that requires
  args+2 words.
\item
  A constants value or a variable reference requires no space in the
  template, just a reference from the containing expression.
\end{enumerate}

Because of support for recursive let, the template is a graph, not a tree.
However, laying out the template remains easy, since the size of each
expression is static.  We just need to compute the size of each binding,
and we can use that to deterimne where all of the let-references should
point to.

There is one edge-case with let bindings, which is that each binding needs
to have an allocation in the resulting template which can be pointed to,
but constants as trivial aliases don't have that.  This can be solved by
simply creating a dummy thunk in that case (using the \lstinline|x_var|
executioner).

Before creating the template and code, we will need to know the sizes
and types for each top-level expressions (let-binding or body), as well
as the total space needed for the code.

\begin{verbatim}
    int cSz = 0
    int xSz = 0

    expr :: Expr -> ()
    expr e = case recognize e of
        RegRef{}      -> cSz++
        StackRef{}    -> cSz+=2
        Direct{}      -> ()
        Indirect{}    -> cSz+=2
        Closure{f,xs} -> xSz += (2 + xs.len); go(f, xs..)
        Thunk{f,xs}   -> xSz += (3 + xs.len); go(f, xs..)

    bindShape :: Expr -> Type
    bindShape e =
        expr e
        case recognize e of
            Closure{f,xs} -> ClosureType{tag=f, size=Sz(xs)}
            Thunk{f,xs}   -> ThunkType{}
            _             -> xSz+=3; ThunkType{}

    Then use this to get an array of sizes and types, once per top-level
    expression.

    size e = case recognize e of
\end{verbatim}

Once we have calculuated the size of each binding, and the body, we
layout the template using the following algorithm:

    At each point, we need to know:

        -   Where will the reference to this result be written?

        -   Where is the end of the currently-written section?

        -   Where is the end of the currently-written commands section?

    If the expression is a constant:
        If it's a direct atom:
            Write the atom as the reference.
        Otherwise:
            Add the constant to the references set.
            Emit a command to load the constant.
            Emit a command to write the constant to the output.

    If the expression is a reference:
        If the reference is an argument by register:
            Emit a command to write the register.
        If the reference is an argument by stack:
            Emit a command to load from the stack.
            Emit a command to write the value.
        If the reference is to a let-binding.
            Determine which type the binding holds.
            Write a tagged offset to the parameter
            Emit a command to hydrate the slot

    If the expression is an application:
        Collect all of the parameters.

        If it's undersaturated, write a closure.

            The size will be args+2.  Reserve this much space in the output.

            Write the GC header.

            Set the multiplicity word to 0 (all parameters are assumed to
            be shared).

            Write the function + arguments via recursive call passing in
            the appropriate slot as the output parameter.

        Otherwise, write a thunk.

            The size will be args+3.  Reserve this much space in the output.

            Write the GC header.

            Write the executioner.

\begin{verbatim}
                If the lowest mutipliciity bit is 1, it's
                \lstinliner|eval_unknown_noupdate|, otherwise it's
                \lstinline|eval_unknown|.
\end{verbatim}

            Set the multiplicity word to 0.

            Write the function (with a multiplicity word of 0).

            If the call is to a known function, load the multiplicity word
            from the metadata.

            Write all arguments via recursive calls.

                After each write, shift the multiplicity word right by one.

\section{Notes on Lazy Laws}

The newest version of PLAN does not force pins and laws on construction,
so they can contain thunks.  However, the whole *expression* is evaluated,
and all direct constant references are evaluated two WHNF.  This provides
enough data to be able to do law codegen at law construction time.

Doing codegen at law construction time also means that we can be sure
to always have the multiplicity analysis available for all of the laws
directly called within the law body.


\section{Recognizing Primops}

By convention, all invocations of primitives are wrapped using a pinned law:

    (Pin x)=(<0> (0 x))

These wrappers all have a predictable shape which can be easily recognized.

    {n 1 [<0> [[0] 1]]}
    {n 2 [<0> [[[0] 1] 2]]}
    {n 2 [<0> [[[[0] 1] 2] 3]]}

Or:

    {n 1 [<0> [7 1]]}
    {n 2 [<0> [[7 1] 2]]}
    {n 2 [<0> [[[7 1] 2] 3]]}

Whenever we find a valid wrapper for a primitive, we assign it's judgement
to be the code for that primitive, and we record a metadata word which
indicates which arguments are evaluated only once.

For example:

\begin{verbatim}
    (Add a b)=(<1> (2 a b))

    Add = <{"Add" 2 [<1> [[[2] 1] 2]}>
\end{verbatim}

This will be recognized as a wrapper by looking up this record in a table:

\begin{verbatim}
    {module=1, opcode=1, arity=2}
\end{verbatim}

The function pointer will then be assigned as a direct pointer to the
Inc primitive, and it's multiplicity will be 1 (the first argument
is unshared).

If you were to instead invoke this primop by hand, without the wrapper,
the generic primop machinery will be run, which will do:

-    Evaluate the argument.
-    Get the argument head and size.
-    lookup this information in the table.
-    Load all of the arguments into registers+stack.
-    Call the approriate primitives.

This is much more work, and the runtime will not be able to infer the
multiplicity information.


\section{Recording Multiplicity Information}

We can encode multiplicity information as a simple bit-mask.  The lowest
bit being set indicates that the first argument is unshared, etc.

However, where in the law allocation box should this information be stored?


\section{Storing the Code}

We can construct a NAT which stores the execution logic for a law,
and simply store it in one of the metadata slots for a LAW.

However, we do need to allocate once in each routine, and we don't
want the code to move around from underneath us, so we need some way to
store this outside of the GC heap.

Fortunately, I have been exploring ideas for moving large nats off-heap
anyways.

I believe that this can be achieved by adding a mark bit to the GC header,
and adding a bit to the NAT tag which indicates whether or not it lives
in a finalized pin or not (second or third GC generation).

Each thread will need to keep a table of large NATs (probably as a
backwards linked list stored in normal GC memory).  Whenever we see a
reference to a local bignat, we must mark it.  And, at the end of GC,
we need to traverse the list of known bignats and free each one which
has not been marked.

Erlang, like many other systems, does something like this, because
otherwise your big nats need to be copied around on every GC.

This system would also mean that the code that we generate for laws
would not move around during GC, which solves a lot of problems.

TODO: should the actual code-pointer point directly into this NAT.
Is that safe?

\section{Storing the Constants Table}

The code will also need access to a table of constants, and this will
need to be placed in the law somewhere as well.

I suppose we can do this all with a single metadata slot:

\begin{verbatim}
    0[multiplicity template-and-code constant1 constant2..]
\end{verbatim}

Where the code NAT directly includes the output template as a constant.

\end{document}
